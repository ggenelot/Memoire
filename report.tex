\documentclass{layout/epfl-report}

%% Set up the bibliography
\usepackage[style=authoryear, backend=biber, backref=true, url=false]{biblatex}
\addbibresource{references.bib}

\usepackage{lipsum}
\usepackage{printlen}

%% Index

\usepackage{imakeidx}
\makeindex[columns=2]

\usepackage{cleveref}
\usepackage[francais]{babel}

% Pour avoir des citations sympas 
\usepackage{csquotes}
%\SetBlockEnvironment{displayquote}
%\SetBlockEnvironmentText{\itshape}

%\usepackage{epigraph}


\usepackage{amsmath}
\usepackage{overpic}
\usepackage{pgfplots}
\pgfplotsset{compat=1.7}
\usepackage{subcaption} % Pour les sous-figures
\usepackage{lscape}
\usepackage{enumitem}
%\usepackage{adjustbox}

% changer les quotes :
\usepackage{etoolbox}

% Redéfinir l'environnement quote
%\renewenvironment{quote}{\begin{displayquote}}{\end{displayquote}}

% Pour pouvoir ajouter des fragments de code
\usepackage{listings}

\lstset{
  language=Python,         % Choisir le langage de programmation (Python ici, mais tu peux changer pour R, C++, etc.)
  basicstyle=\ttfamily,    % Style de base du texte (typewriter pour le code)
  keywordstyle=\color{blue},  % Couleur des mots-clés
  commentstyle=\color{gray},  % Couleur des commentaires
  stringstyle=\color{red},    % Couleur des chaînes de caractères
  numbers=left,            % Numérotation des lignes à gauche
  numberstyle=\tiny,       % Taille de la police pour les numéros de ligne
  stepnumber=1,            % Fréquence des numéros de ligne
  breaklines=true,         % Autoriser les retours à la ligne dans le code
  frame=single,            % Encadrer le code
}





\usepackage{tcolorbox}
\usepackage{ifthen}

\newcommand{\showPEEL}{false} % Changer "true" en "false" pour masquer l'encadré

% Commande pour l'encadré PEEL
\newcommand{\PEEL}[4]{%
    \ifthenelse{\equal{\showPEEL}{true}}{
        \begin{tcolorbox}[colback=gray!10, colframe=gray!50, title={PEEL}]
            \textbf{Point:} #1 \\
            \textbf{Evidence:} #2 \\
            \textbf{Explanation:} #3 \\
            \textbf{Link:} #4
        \end{tcolorbox}
    }{}
}

% Définition de la commande \blog

\usepackage{marginnote} % Pour insérer des éléments dans la marge
%\usepackage[marginparwidth=2cm]{geometry} % Pour ajuster la largeur de la marge si nécessaire

\newcommand{\blog}[1]{
    \marginnote{%
        \href{#1}{%
            \includegraphics[width=1.5cm]{figures/logos/development.png}%
        }
    }[0cm]
}




%% Legende 

% Justifier le texte
\usepackage{ragged2e}

% Définition de la commande \legende
\newcommand{\legende}[2]{
    \caption[#1]{\textbf{#1} \hspace{0.1em} \footnotesize#2} % Titre en gras et légende en italique sur la même ligne
}


\usepackage{glossaries}
\makeglossaries


%% Biblio

\hypersetup{
    colorlinks=true,
    linkcolor=gray,
    citecolor=blue,
    urlcolor=blue
}



%% Additional packages and commands
\setlist{itemsep=-2pt} % Reducing white space in lists slightly
\renewcommand{\deg}{\si{\degree}\xspace} % Use \deg easily, everywhere

\renewcommand{\listfigurename}{Liste des illustrations}
\renewcommand{\listtablename}{Liste des tableaux}

\usepackage{xcolor}
\usepackage{mdframed}
\usepackage{newfloat}

% Définition d'un nouvel environnement pour les encarts méthodologiques
\DeclareFloatingEnvironment[fileext=frm,placement={!ht},name=Encart]{method}
\captionsetup[method]{labelfont=bf}

% Couleur de fond pour les encarts
\definecolor{lightgray}{RGB}{240,240,240}

% Définition du style des encarts
\mdfdefinestyle{methodstyle}{
    backgroundcolor=lightgray,
    linewidth=0pt,
    roundcorner=5pt,
    innertopmargin=10pt,
    innerbottommargin=10pt,
    innerleftmargin=10pt,
    innerrightmargin=10pt
}



\newcommand{\methodtitle}[1]{%
    \refstepcounter{method}%
    \textbf{Méthode \arabic{method} : #1} \\
    \par
}

% Définition de l'environnement "methodbox"
\newenvironment{methodbox}[1][]
{\begin{mdframed}[style=methodstyle]\methodtitle{#1}\ignorespaces}
{\end{mdframed}}


%% ----------------------------------------------------------------------
%%    Begin of document + Frontmatter (Roman page numbering)
%% ----------------------------------------------------------------------

\linespread{1.3}



\begin{document}

\newglossaryentry{latex}
{
    name=latex,
    description={Is a markup language specially suited 
    for scientific documents \LaTeX}
}


\newglossaryentry{tippingpoint}
{
    name=tipping point, 
    description={Un \textit{tipping point} est un point où la dynamique du système change \cite{acemoglu_colonial_2001} }
}

\newglossaryentry{Risk tipping point}
{
    name=risk tipping points, 
    description={kljekljkl}
}

\newglossaryentry{Modèle}
{
    name=Modèle, 
    description={Un modèle est une représentation simplifiée de la réalité}
}

\newglossaryentry{iam}
{
    name=Modèle intégré, 
    description={Les modèles intégrés sont des \textit{représentations simplifiées de systèmes sociaux et physiques complexes, qui se concentrent sur les interactions entre l'économie, la société et l'environnement} \cite{intergovernmental_panel_on_climate_change_ipcc_annex_2023}  \\
    On peut distinguer deux grandes familles de modèles intégrés : les modèles de \textit{policy optimization} et ceux de \textit{policy evaluation}. Les premiers visent à trouver le "meilleur" chemin parmis toutes les options possibles, c'est à dire celui maximisant une fonction objectif. Les seconds permetent de voir l'évolution de variables clés au fil du temps, mais ne vise pas à maximiser des variables. \\
    Malgrè de nombreuses limites, ils constituent un outil essentiel de la prise de décision climatique.
    }
}

\newglossaryentry{scenario}
{
    name=Scénario, 
    description={Un scénario est une jeu de paramètre que l'on donne à un modèle}
}

\newglossaryentry{imp}
{
    name=Illustrative mitigation pathways, 
    description={Trajectoires d'atténuation stéréotypée}
}

\newglossaryentry{attenuation}
{
    name=Atténuation, 
    description={Réduction de l'ampleur du changement climatique par la réduction des émissions de gaz à effet de serre}
}


\newglossaryentry{ethique}
{
    name=éthique, 
    description={}
}

\newglossaryentry{procedural ethics}
{
    name=éthique procédurale, 
    description={Respect des lignes conductrices et des usages dans un travail de recherche. Correspond à ce que l'on appelle communément de la \textit{bonne recherche} (good science). \cite{tuana_leading_2010} la définit ainsi : \textit{ethical aspects of the process of conducting scientific research, such as: falsification, fabrication, and plagiarism; care for subjects (human and non-human animal); responsible authorship issues; analysis of and care for data}.}
}

\newglossaryentry{intrinsic ethics}
{
    name= éthique intrinsèque, 
    description={Valeurs personnelles qui sont incorporées dans le travail de recherche, de manière consciente ou non. \cite{tuana_leading_2010} la définit ainsi : \textit{ethical issues and values that are embedded in or otherwise internal to the production of scientific research and analysis. These involve ethical issues arising from, for example: the choice of certain equations, constants, and variables; analysis of data; handling of error, and degree of confidence in projections.}}
}

\newglossaryentry{extrinsic ethics}
{
    name=éthique extrinsèque, 
    description={Dimension éthique des effets que produit la recherche sur la société. \cite{tuana_leading_2010} la définit ainsi : \textit{ethical issues that are external to the production of scientific research. These arise, for example, when considering the impact of scientific research on society; e.g., the effects of technological innovations on social ends such as health and well-being, whether pressing social and economic issues are likely to be addressed and if so, who benefits, and the role of science in policy-making.}}
}




\newacronym{ipcc}{IPCC}{Voir GIEC}

\newacronym{giec}{GIEC}{Groupe Intergouvernemental d'Experts sur le Climat. Il s'agit dun groupe d'experts, qui travaillent pour faire la synthèse des connaissances disponibles en matière de climat. \cite{cointe_ar6_2024}}







\frontmatter

%% Define the main parameters
\title{Fonctions de dommages dans les modèles intégrés}
\subtitle{Quelle forme, étendus et enjeux pour les fonctions de dommage ?}
\author{Gabriel Genelot}

\subject{\textit{Mémoire de recherche}} % Cover only
\affiliation{} % Cover only
\coverimage{figures/logos/clouds.jpg} % Aspect ratio of 2:3 (portrait) recommended
\definecolor{title}{HTML}{A9CA53} % Color for cover title

\makecover

\begin{titlepage}

\begin{center}

%% Print the title
{\makeatletter
\largetitlestyle\fontsize{45}{45}\selectfont\@title
\makeatother}

%% Print the subtitle
{\makeatletter
\ifdefvoid{\@subtitle}{}{\bigskip\titlestyle\fontsize{20}{20}\selectfont\@subtitle}
\makeatother}


\bigskip
\bigskip

%% Print the name of the author
{\makeatletter
\largetitlestyle\fontsize{25}{25}\selectfont\@author
\makeatother}

\bigskip

\href{mailto:gabriel.genelot@minesparis.psl.eu}{gabriel.genelot@minesparis.psl.eu}

\bigskip
\bigskip

%% Print table with names and student numbers
%\setlength\extrarowheight{2pt}
%\begin{tabular}{lc}
%    Student Name & Student Number \\\midrule
%    First Surname & 123456 \\
%\end{tabular}

\begin{centering}
    \includegraphics[width=3cm]{illustrations/ccby.png} % Ajustez la taille du QR code selon vos besoins
\end{centering}


\vfill

\textbf{Résumé} \\

\begin{center}
\justify


Nous explorons les enjeux éthiques liés à la modélisation des impacts du changement climatique. Dans un premier temps, nous recensons des fonctions de dommages issues de différents modèles, pour en comparer les caractéristiques et identifier des enjeux éthiques. Nous implémentons ensuite les fonctions de dommage de DICE, FUND et WITNESS au sein du modèle WILIAM, et proposons un coefficient qui représente l'équité spatiale. Nous abordons l'épistémologie des modèles intégrés au regard de ce nouveau coefficient, avant de de réaliser des entretiens semi-directifs avec des acteurs du régime climatique (chercheurs, décideurs). Nous montrons que les fonctions de dommage sont très sensibles aux implications éthiques sous-jacentes, et qu'il faut qu'elles soit explicitées et modifiables par les utilisateurs. 

\end{center}


\bigskip

\bigskip

%% Print some more information at the bottom
\begin{tabular}{lp{10cm}}
    Directrice : & Nadia Maïzi \\
    Tuteur : &  Thibaut Feix\\
    Cadre : & Mémoire de recherche de M2 \\
    Faculté : & Institut d'études du développement de la Sorbonne (IEDES), Université Paris 1 - Panthéon Sorbonne
\end{tabular}

\bigskip
\bigskip

%% Add a source and description for the cover and optional attribution for the template
\begin{tabular}{p{15mm}p{10cm}}
    %Couverture : & Photo by Tom Fisk from \href{https://www.pexels.com/photo/green-forest-2739664/}{Pexels}  \\
    % Feel free to remove the following attribution, it is not required - still appreciated :-)
    %Style: & EPFL Report Style, with modifications by Batuhan Faik Derinbay
\end{tabular}

\end{center}



%% Insert the EPFL logo at the bottom of the page
\begin{tikzpicture}[remember picture, overlay]
    \node[above=10mm] at (current page.south) {%
         \includegraphics[width=0.5\linewidth]{figures/logos/iedes_logo.jpg}
    };
\end{tikzpicture}

\end{titlepage}

\chapter*{Note d'intention}
\addcontentsline{toc}{chapter}{Note d'intention}

Ce mémoire est l'aboutissement de six mois de réflexion, de tâtonnements et de questionnements. En réalité, c'est plutôt une étape dans la continuité de nombreuses années d'études au cours desquelles j'ai eu du mal à trouver dans quelle case me ranger, et à définir par quelle fenêtre regarder. Le résultat est un travail varié : de nombreuses méthodes différentes, tout autant de questions foisonnantes, qui pourraient chacune faire l'objet d'un mémoire, voire d'une thèse, peut-être même d'une carrière universitaire. \\

Cette possibilité d'explorer et de rester ouvert à la sérendipité a été précieuse et centrale, autant dans ma formation que dans ce travail. Elle m'a offert l'opportunité de m'intéresser à beaucoup de sujets, et surtout d’observer le même sujet sous de nombreux angles. Cela a rendu l'ensemble parfois difficile à suivre, et certains diront que c'est un peu brouillon. C'est un constat que je reconnais : il y a dans ces pages de nombreux détours, des points qui sont ouverts sans jamais être refermés, et peut-être autant de manques, qui auraient pu être approfondis. C'est d'ailleurs souvent ce que l'on attend de la recherche et de ses hussards en blouse blanche : se spécialiser, pour maitriser toujours plus sa discipline, ses méthodes et ses questions. C'est d'ailleurs ce qui permet d'avoir une grande rigueur, nécessaire pour manipuler avec précision et parcimonie des outils théoriques très pointus. \\

Malgré cela, j'ai tenté de suivre tout au long de ce mémoire un fil conducteur. Il s'agit des dommages liés au changement climatique, ou plus précisément de leur modélisation. C'est un objet, un tout petit objet, même : une équation, une ligne de code, quelques lignes dans une documentation. On pourrait alors penser : quoi de plus simple de se spécialiser autour de ce petit objet, si petit qu'on peut aisément en dessiner les contours et en maitriser les enjeux dans le temps imparti pour un mémoire universitaire. C'est en partie vrai, mais surtout faux. Car cet objet est relié de toutes parts à tant d'autres, et pas des moindres : d'abord aux sciences du climat (rien que ça); puis à leur interaction avec les sociétés, et donc les sciences sociales et humaines (rien que ça), qu'il cherche à synthétiser en quelques termes (décidément). \\

J'ai donc choisi d'adopter ce fil conducteur. Ce travail est donc une sorte de grand \emph{patchwork} d'idées, de littératures, de méthodes, qui s'articulent autour du même objet, s'éclairent mutuellement et permettent, je l'espère, de répondre à ces questions essentielles avec un regard nouveau.  \\

À travers ce mémoire, j’ai plongé dans un monde fascinant et qui m'était tout à fait inconnu. J'ai appris énormément de choses, de concepts, et d’idées qui m'ont constamment nourri de nouvelles questions et réflexions. L’objectif de ce mémoire n’est pas tant d’offrir des réponses que d’ouvrir des portes, et d’accompagner mes amis et lecteurs — géographes, économistes, philosophes, sociologues, mathématiciens, ingénieurs et curieux — dans l'exploration de ces questions, en mettant à leur disposition les pistes que j’ai pu rassembler. Mon espoir est que cela suscite de nombreux échanges, et je suis convaincu que ce n’est qu’un début.
\chapter*{Remerciements}
\addcontentsline{toc}{chapter}{Remerciements}

% Encadrement

Je voudrais commencer ce manuscrit en remerciant celles et ceux qui m'ont accompagné dans ce voyage initiatique à la recherche. Tout d'abord, je suis très reconnaissant envers Nadia Maïzi, directrice du Centre de Mathématiques Appliquées, qui m'en a ouvert les portes avec toute sa confiance, son exigence et sa bienveillance. Ces rendes-vous réguliers ont été autant de jalons où j'ai été impressionné par la précision des réflexions, la richesse des échanges et la qualité des conseils. Cette présence discrète et rassurante m'a permis de grandir intellectuellement et méthodologiquement; elle est rare, et je suis heureux de l'avoir trouvée ici. Je voudrais bien sûr remercier Thibaut Feix, qui m'a encadré au quotidien et avec qui j'ai partagé un bureau, des doutes et des questions. Merci de m'avoir proposé ce sujet, qui s'est révélé passionnant, et merci croire en une science plus humaine et juste. \\


% Encadrement plus lointain, inspirations

Je voudrais également remercier toutes celles et ceux qui m'ont encadré de près ou de loin, répondu à des questions ou montré des voix. Je pense bien sûr à Béatrice Cointe, qui m'a aidé à formuler des questions, à Mathias Girel, grâce à qui j'ai osé m'approprier les outils essentiels mais impressionnants de l'épistémologie, Marc Fleurbeay, qui inspire mon chemin d'étudiant en économie, Alessandra Giannini, dont la confiance m'a permis de voir de plus près les rouages du régime climatique. Je veux aussi exprimer ma reconnaissance auprès de tous les autres que je ne peux pas citer, qui par une remarque, une question ou un échange, ont fait avancer ma réflexion. \\ 

% Labo

Mais cette aventure n'est pas qu'intellectuelle. Aussi, je voudrais remercier Claire, Lucas, Alice, Victor, Grégorio, Sophie, Marie, et tous les autres, pour m'avoir accueilli à Sophia-Antipolis, et surtout pour faire résonner les couloirs du CMA d'éclats de rire et rayonner de bonne humeur. \\

% Proches

Enfin, je voudrais remercier mes proches, celles et ceux qui ont partagé des heures de bibliothèque et m'ont permis, par la diversité et la richesse de leurs points de vue, de louvoyer entre les disciplines pour trouver ma voie. Une pensée toute spéciale va à mes grands-parents, qui m'ont toujours guidé par leur curiosité. 

%\input{frontmatter/summary}

\tableofcontents

%\listoffigures

%\listoftables

%\input{frontmatter/nomenclature}

%% ----------------------------------------------------------------------
%%    Mainmatter (Arabic page numbering)
%% ----------------------------------------------------------------------

\mainmatter



\PEEL{Les modèles intégrés sont sensibles aux choix éthiques des modélisateur.ices, ce qui a des conséquences importantes sur leur rôle dans la société.}{Taux d'actualisation, inégalités}{La prise en compte des inégalités (spatiales, sociales, temporelles) est un facteur majeur du niveau de dommage; la manière de les prendre en compte résulte de choix éthiques.}{On doit donc avoir un regard critique sur ces différents choix. }




\chapter*{Introduction}
\newrefsegment

\PEEL{Exposez l'idée principale ou l'argument que vous souhaitez développer dans cette partie.}{Fournissez des preuves, des données ou des citations qui soutiennent votre point.}{Expliquez en quoi les preuves que vous avez fournies sont pertinentes et comment elles appuient votre point.}{Faites le lien avec le sujet principal ou avec la section suivante de votre mémoire.}


%% Amorce

% réduction de l'incertitude => elle est désagréable donc on cherche toujours à la minimiser; 
% representer le monde le plus précisement possible => fantasme et ambition des sciences => La carte et le navigateur \cite{edenhofer_mapmakers_2014}. 



% Comment prendre les bonnes décisions face au changement climatique ? 

Borges nous confie une fable sur la cartographie. Il raconte l'histoire d'un cartographe, qui cherche à représenter l'empire avec une précision infinie. Il a abouti à une carte qui fait la taille de l'empire \autocite{palsky_borges_1999}.

\begin{authoredquote}[(Suarez Miranda, Viajes de Varones Prudentes, Livre IV, Chapitre XIV, Lérida, 1658]
    DE LA RIGUEUR DE LA SCIENCE \\
    
    En cet empire, l'Art de la Cartographie fut poussé à une telle Perfection que la Carte d'une seule Province occupait toute une ville et la Carte de l'Empire toute une Province. Avec le temps, ces Cartes Démesurées cessèrent de donner satisfaction, et les Collèges de Cartographes levèrent une Carte de l'Empire, qui avait le Format de l'Empire et qui coïncidait avec lui, point par point. Moins passionnées pour l'Étude de la Cartographie, les Générations Suivantes réfléchirent que cette Carte Dilatée était inutile et, non sans impiété, elles l'abandonnèrent à l'Inclémence du Soleil et des Hivers. Dans les Déserts de l'Ouest, subsistent des Ruines très abimées de la Carte. Des Animaux et des Mendiants les habitent. Dans tout le Pays, il n'y a plus d'autre trace des Disciplines Géographiques.
\end{authoredquote}

Le parallèle avec la modélisation est assez fécond. En effet, la cartographie, comme la modélisation, sont des représentations schématiques de la réalité. Les deux visent à se repérer. Cette métaphore est d'ailleurs reprise par \autocite{edenhofer_mapmakers_2014}, qui indique que \enquote{the report provides a “living map,” drawn in a social learning process between scientists (mapmakers) and policy-makers (navigators), to be used to traverse the largely unknown territory of climate policy}. Dans un contexte aussi incertain que le changement climatique et les politiques publiques qu'il implique, la modélisation éclaire les chemins possible, en cartographiant dans le temps les phénomènes. \\

Longtemps, l'emphase a été mise sur la production de connaissance scientifique, pour s'assurer de l'existence de celui-ci, puis pour en mesurer avec toujours plus de précision l'ampleur et la vitesse. Cet effort de connaissance s'est aussi déployé par des tentatives de vulgarisation et de sensibilisation aux questions climatiques. Si ces efforts ont permis d'établir de manière presque consensuelle qu'il y avait un grave danger et une nécessité d'action face aux changements climatique, les mesures à prendre sont beaucoup moins claires. Deux exemples en témoignent assez bien. D'abord, la difficulté qu'ont les différentes nations à s'accorder sur une conduite commune, et ce, malgré l'ampleur de la crise et des moyens déployés (COPs, diplomatie climatique, etc.). Ensuite, la part qu'ont pris les sujets climatiques dans les positionnements politiques, comme nouveau marqueur : il faudrait plus de \textit{"justice sociale"}, ou encore lutter contre une \textit{"écologie punitive"}. \\

Une des difficultés réside dans l'incertitude qui entoure le changement climatique : d'abord, historiquement, autour de son existence; puis autour de son ampleur; aujourd'hui, autour des canaux par lesquels ces impacts vont se réaliser et leurs interactions avec des structures sociales par essence très complexes. 

\begin{figure}[h]
    \centering
    \includegraphics[width=\linewidth]{figures/trajectoire_spm6.png}
    \legende{Les trajectoires possibles pour atteindre les objectifs de développement durable}{Chaque fléche représente symboliquement un chemin de développement possible. Plus les flèches sont rouges, et plus le développement est loin des Objectifs de développement Durable et vulnérable aux aléas climatiques. Cette figure, issue du rapport de synthèse du GIEC (SPM.6) \textcite{lee_ipcc_2023}, montre que les décisions prises aujourd'hui influe les trajectoires possible demain. Elle illustre comment les concepts de trajectoires et de scénarios sont centraux dans le choix du chemin de développement.}
    \label{fig:pathways}
\end{figure}

Si l'existence de celui-ci ainsi que l'ampleur des conséquences qu'il induit ne font plus de doute, les actions à mettre en œuvre et les choix à faire pour le limiter et s'en protéger sont moins consensuelles. Elles font l'objet de nombreux débats, aux niveaux nationaux mais aussi internationaux dans la diplomatie climatique. Un des défis de ces décisions est l'incertitude qui les entoure : on ne connait pas les conséquences de chaque décision, et pourtant, il faut choisir un chemin. Une discipline cherche à éclairer cette route sombre, à la manière des phares d'une voiture : la prospective. Un outil particulièrement utilisé pour réduire cette incertitude est la modélisation, et particulièrement la modélisation intégrée. \\

%% Définition des termes

Pour aborder ce sujet, nous allons mobiliser plusieurs concepts, que nous serons amenés à redéfinir au fil de nos réflexions. 

% Modélisation 

D'abord, celui de \gls{modelisation}. Il s'agit d'une simplification de la réalité, qui permet de mieux la comprendre. Nous nous intéresserons particulièrement aux \gls{iam}, \textit{représentations simplifiées de systèmes sociaux et physiques complexes, qui se concentrent sur les interactions entre l'économie, la société et l'environnement}. Plus simple que les modèles climatiques, ils représentent à la fois des composantes physiques, économiques et énergétiques. Ils permettent ainsi de considérer simultanément ces sous-systèmes et leurs interactions. Les impacts du changement climatique représentent les effets délétères du changement climatique sur les sociétés ou les écosystèmes. 

% Responsabilité / choix / éthique / incertitude



%% Rappel du sujet

Nous nous intéressons donc ici à la modélisation des impacts du changement climatique, c'est-à-dire à la manière dont ils sont représentés dans les modèles intégrés. La question de recherche principale est : faut-il représenter les impacts du changement climatique dans les modèles ? Nous cherchons à y répondre par les sous-questions suivantes : Avec quel niveau de complexité faut-il représenter les interactions entre les sociétés et le climat ? Comment représenter les impacts-non monétaires ? Quelle est la responsabilité des modélisateur sur l'interprétation qui est faite des modèles ?  \\

%% Plan 

\begin{figure}
    \centering
    \includegraphics[width=\textwidth]{illustrations/intro.png}
    \legende{Les grandes étapes de la vie d'une modèle}{La modélisation consiste à représenter de manière simplifiée des phénomènes. Dans notre cas, il s'agit de phénomènes liés au changement climatique. Il existent avant toute forme de représentation (A), puis font l'objet d'une simplification (B). Ils sont ensuite interprétés (C), avant de conduire à des actions, telles que des décisions politiques (D). Chacune de ces étapes correspond à un chapitre : d'abord, on commence par évoquer les différents phénomènes qui sont modélisés (\ref{chapter:introduction}), puis on cherche à savoir comment ceux-ci sont représentés (\ref{chapter:litrev}). On analyse ensuite l'importance du choix de représentation sur les interprétations possibles (\ref{chapter:modelisation}), avant de discuter des enjeux éthiques liés à l'interprétation des résultats des modèles (\ref{chapter:ethique}). Enfin, on s'intéresse à la manière dont cette chaîne de production de connaissance alimente le débat public (\ref{chapter:socio}).}
    \label{fig:enter-label}
\end{figure}



Le  chapitre \ref{chapter:introduction} est consacré à une présentation du contexte scientifique et politique dans lequel s'inscrivent les \gls{damage function}. Il présente d'abord les différents risques climatiques, c'est-à-dire les raisons pour lesquelles on s'intéresse au changement climatique. Il poursuit en présentant l'intrication entre les sciences dures et la politique dans les négociations climatiques. Enfin, il présente les outils qui permettent d'éclairer ces décisions, notamment les modèles intégrés et leurs fonctions de dommage. Il sert avant tout à contextualiser l'environnement dans lequel se situe la modélisation intégrée. \\

Le chapitre \ref{chapter:litrev} présente une revue de littérature sur les \gls{damage function}. Il vise à présenter un état des lieux, qui se veut le plus complet mais non exhaustif, des fonctions de dommages qui sont utilisées (ou non) dans les modèles. Il est suivi d'une analyse de cet état des lieux, en classant les fonctions de dommages selon leur utilisation finale, leur forme ou encore les paramètres qui sont pris en compte. \\

Le chapitre \ref{chapter:modelisation} est la partie la plus expérimentale du mémoire. On cherche à quantifier l'effet des choix de modélisation sur le résultat des modèles, c'est-à-dire à quel point ils sont sensibles à leurs hypothèses. Dans un premier temps, on décrit la méthodologie pour pouvoir comparer les différentes fonctions entre elles, puis on réalise de nombreuses simulations avec de légères variations. Ces résultats font ensuite l'objet d'une analyse économétrique, pour quantifier l'effet des variations sur le niveau de dommage final et sa distribution. Cette expérimentation se fait dans le modèle \Gls{WILIAM}, auquel on a ajouté des fonctions reproduisant le comportement des fonctions de dommage d'autres modèles. \\

Dans le chapitre \ref{chapter:ethique}, on identifie des enjeux éthiques liés à la modélisation intégrée, que l'on tente d'éclairer à travers une approche épistémologique. Des exemples, tels que le choix de la forme, de la fonction, des paramètres ou encore des phénomènes représentés, sont analysés grâce à des philosophes des sciences. \\


Le chapitre \ref{chapter:socio} s'interroge sur la perception de ces enjeux éthiques chez les personnes qui participent au débat public sur les questions climatiques. À travers des entretiens semi-directifs chez une dizaine d'acteurs (scientifiques, techniciens, politiques et société civile), on cherche à établir quelle compréhension des enjeux éthiques transparait dans leur transposition au débat public. 

\begin{tcolorbox}[title=Avertissement]
    L'ensemble du code source utilisé dans le cadre de ce mémoire est accessible en libre accès sur \href{https://github.com/ggenelot/damage-functions-modeling}{ce dépôt github}. Par ailleurs, un des fils conducteurs de ce mémoire est de mener une réflexion épistémologique au plus près du modèle, et notamment du code. Cela est notamment rendu possible par \href{https://damage-functions-modeling.readthedocs.io/en/latest/index.html}{ce blog}, qui permet d'intégrer des morceaux de code. Tout au long du mémoire, des passages feront référence à des morceaux de code et/ou à des analyses qui y sont présentées. Il suffit alors de cliquer sur le lien (voir figure \ref{fig:logo}) pour y avoir accès. Il est également possible de se rendre sur le site en scannant le QR code (voir figure \ref{fig:qrcode}). \\

    \begin{center}
        \begin{minipage}{0.3\textwidth} % Ajustez la largeur selon vos besoins
            \centering
            \includegraphics[width=3cm]{figures/logos/development.png}
            \captionof{figure}{Logo cliquable permettant d'accéder au site compagnon}
            \label{fig:logo}
        \end{minipage}%
        \hspace{0.05\textwidth} % Espace entre les deux images
        \begin{minipage}{0.3\textwidth} % Ajustez la largeur du minipage
            \centering
            \includegraphics[width=3cm]{illustrations/frame.png} % Ajustez la taille du QR code selon vos besoins
            \captionof{figure}{QR code vers le site compagnon}
            \label{fig:qrcode}
        \end{minipage}
    \end{center}



    
\end{tcolorbox}

\begin{tcolorbox}[title=License]
    Toute la production originale est distribué sous la licence CC0. Ceci ne comprend pas les images et figures issues d'autres sources, qui appartiennent à leurs auteurs respectifs. De la même manière, le code source de WILIAM appartient à ses développeurs, et la propriété intellectuelle de chaque fonction de dommage revient à ses créateurs. \\

     \begin{center}
        \begin{minipage}{0.3\textwidth} % Ajustez la largeur selon vos besoins
            \centering
            \includegraphics[width=6cm]{illustrations/ccby.png}
            \captionof{figure}{Distribué sous licence CC-BY}
            \label{fig:ccby2}
        \end{minipage}%
    \end{center}

\end{tcolorbox}


Les annexes proposent des documents complémentaires, notamment un index / glossaire, qui permet de mieux situés les termes les uns par rapport aux autres. 



\chapter{Impacts, risques et mesures}
\label{chapter:introduction}
\newrefsegment

\chapterabstract{Le changement climatique est à la fois extremement incertain, et nécessite des actions et des prises de décisions rapides et de grande envergure. Ce paradoxe a donné naissance à des institutions, comme le CCNUCC ou le GIEC, et à des outils, comme la modélisation intégrés. A chaque fois, l'objectif est de réduire l'incertitude et de favoriser le rapprochement entre l'action politique et la connaissance scientifique. Dans cette introduction, nous introduisons quelques uns des concepts cadres qui nous serons utiles tout au long du mémoire}

\newpage

%% Accroche
%Agir dans l'incertain : voici un des défis auquel nous soumet le changement climatique. L'action est nécessaire, tant l'ampleur de ces impacts est importante. Et l'incertitude est omniprésente : 

Les impacts du changement climatique sont de plus en plus présents dans le débat public : sécheresse, inondations, tempêtes font régulièrement les couvertures des journaux, et la plupart des partis politiques français ont pris position sur un programme climatique. C'est là une des spécificités du changement climatique : l'articulation très fine entre une réalité scientifique de plus en plus consensuelle d'une part, et de choix politiques incertains et normatifs d'autre part. 

%% Définition des termes

C'est justement la définition classique d'un risque : l'interaction entre un aléa et un enjeu. L'aléa, dans ce contexte, désigne un événement physique dont l'occurrence est possible : une canicule, un ouragan, la montée du niveau de la mer, etc. Un des effets du changement climatique est d'augmenter l'amplitude et la fréquence des aléas. L'enjeu est lié à ce que l'on a à perdre, c'est-à-dire ce qui a de la valeur et qui peut être affecté par la réalisation de l'aléa. 

%% Rappel du sujet
Nous nous intéressons au rôle des fonctions de dommage dans la formation de ces décisions politiques. 



%% Problématique
Nous tenterons ici de donner des éléments de contexte à la question suivante : 
Comment les fonctions de dommage des modèles intégrés permettent-elles de prendre en compte les risques climatiques ? Celle-ci s'accompagne d'autres questions : Quels sont les risques climatiques ? Comment sont pris en compte ces risques dans la gouvernance mondiale et nationale ? Et quels outils permettent d'éclairer ces prises de décision ? 



%% Annonce du plan 
Ce chapitre est construit en trois parties. Dans une première partie, nous nous intéresserons aux impacts du changement climatique, en les classant en trois types : les effets de tendance, qui sont linéaires ; les effets ponctuels et catastrophiques ; et les effets de seuil, ou tipping points. Nous aborderons dans une deuxième partie la manière dont les institutions internationales se sont organisées pour faire face à ces risques, et comment ces questions articulent des composantes scientifiques et politiques. Enfin, dans une troisième partie, nous détaillerons certains des outils qui ont été développés pour répondre à ces enjeux : les modèles intégrés, leurs fonctions de dommage et le coût social du carbone. 




\section{Le changement climatique : tendances et impacts}
\label{sect/1/1}

Dans cette première section, nous allons présenter (très) succinctement les effets du changement climatique sur les sociétés. Il s'agit principalement d'une mise en contexte de différents éléments tirés du Synthesis report de l'AR6 du GIEC \cite{lee_ipcc_2023}. \\

\begin{figure}
    \centering
    \includegraphics[width=\textwidth]{figures/spm1.png}
    \legende{Les impacts du changement climatique sont nombreux et sous des formes variées}{Les impacts du changement climatique vont continuer à s'intensifier. Ceux-ci sont regroupés en quatre catégories : eau et nourriture; santé et bien-être; villes, peuplement et infrastructures; biodiversité et écosystème (panneau a). Ces impacts sont issus de phénomènes naturels attribués à l'action humaine (panneau b). Les choix de trajectoires d'émissions impactent le niveau de réchauffement moyen, qui lui même impacte l'intensification des phénomènes. Figure issue du Synthesis Report du GIEC.}
    \label{fig:ipcc-impacts}
\end{figure}

Pour coller au mieux aux enjeux de la modélisation, nous classons ici les impacts du changement climatique en trois catégories : les effets tendanciels, qui désignent une tendance longue, dont la progression est stable et durable (augmenation du niveau de la mer, acidification des océans); les catastrophes, qui désignent des évenements soudains et de grande ampleur, souvent peu probable mais avec un impact fort (cyclones, inondations, sécheresses); enfin, on décrit les tipping points, ou points de bascule : ce sont des altérations du fonctionnement même du système, qui ne réagit plus de la même manière. 

\subsection{Les effets tendanciels}
\label{sect/1/1/1}

% Presenter les rapports du GIEC

Les effets les plus connus et souvent présentés en premier sont les effets que l'on pourrait qualifier de tendanciels. On désigne par ce terme des effets qui ont lieu de manière progressive, régulière et sur le long terme. Un exemple caractéristique est la montée du niveau de la mer. Celle-ci étant liée à la fonte des glaces et à la température des océans, elle a lieu de manière régulière au fur et à mesure que les océans se réchauffent. 



\subsection{Les catastrophes}
\label{sect/1/1/2}

A l'inverse des effets tendanciels, les catastrophes désignent des évenement soudains, ponctuels et peu probables. Ils ont la particularité d'être très mal représenté par une moyenne. Ces événements sont dans la majorité des cas non réalisés, et ont donc un niveau de dommage nul; en revanche, lorsqu'ils se réalisent, ils causent des dommages importants. Ainsi, la moyenne (ou espérance) du niveau de dommage ne capte que très mal ces phénomènes. En effet, cette moyenne serait assez faible par rapport au niveau maximal de dommage; par ailleurs, ce niveau de dommage catastrophique, lorsqu'il est atteint, peut être exacerbé par la saturation des moyens de réponse, telle que la destruction d'infrastructures critiques (routes, hopitaux) ou la surcharge de structure de réponse (services de secours, mécanismes de soutien aux populations). 

\subsection{Les tipping points}
\label{sect/1/1/3}

Enfin, les tipping points consistent en une altération profonde et durable du fonctionnement d'un système. Ayant changé de conditions, le système évolue dans une nouvelle zone, où les comportements peuvent avoir changé. Par exemple, si des conditions de sécheresse ont amené à une dégradation profonde de la flore, celle-ci ne peut plus jouer son rôle de régulateur hydrique. La disparition de ce régulateur provoque en retour des sécheresses plus importantes qu'initialement, ce qui exacerbe le phénomène. 

\begin{figure}
    \centering
    \includegraphics[width=\linewidth]{figures/tipping_point.png}
    \legende{Après un point de bascule, la dynamique du système change radicalement.}{Les tipping points, comme les événements catastrophiques, sont plus difficiles à modéliser que la tendance moyenne des dommages. }
    \label{fig:tipping-point}
\end{figure}

\begin{figure}
    \centering
    \includegraphics[width=\linewidth]{figures/earth_tipping_point.png}
    \legende{Les points de bascule dans le contexte climatique}{}
    \label{fig:earth-tipping-point}
\end{figure}

\begin{figure}
    \centering
    \includegraphics[width=1\linewidth]{figures/Tipping_points_2022_list.jpeg}
    \legende{Liste des points de bascule à l'échelle planétaire}{}
    \label{fig:enter-label}
\end{figure}

\paragraph{Les risk tipping points}

Au-delà de la définition classique du tipping point, l'Université des Nations Unies propose, dans son rapport sur les risques interconnectés, une nouvelle définition des risques interconnectés. Un \textit{risk tipping point}, ou point de bascule des risques, désigne \textit{l'instant où un système socioécologique ne peut plus absorber le risque et réaliser ses fonctions}. Après le passage de ce point de bascule, la possibilité d'un impact catastrophique augmente substantiellement. Six risques sont identifiés comme particulièrement représentatif des effets systémiques d'un driver sur tous les autres : l'accélération de l'extinction de la biodiversité, la réduction de l'eau de surface disponible, la fonte des glaciers, les débris spatiaux, la chaleur trop importante, et un futur qui n'est plus assurable. Parmi ces points de bascule, quatre sont  reliés à l'augmentation de la température atmosphérique ou océanique, et quatre à l'augmentation de la concentration en gaz à effet de serre dans l'atmosphère. \cite{united_nations_university_-_institute_for_environment_and_human_security_unu-ehs_interconnected_2023}

Ce concept est intéressant pour deux raisons : d'abord, il illustre la complexité des systèmes physiques et sociaux, et la complexité de leur interaction. Ensuite, il montre que la prise en compte d'un impact par le moyen d'un seul mécanisme risque de sous-estimer cet impact, car cela ne permet pas de prendre en compte les réactions en chaines et les interactions entre les différents impacts. 

\section{Prendre des décisions dans l'incertain : le rapprochement de la science et du pouvoir}
\label{sect/1/2}

Dans cette section, nous proposons un bref retour historique sur des grandes dates ayant marqué la politique climatique. Des premières identifications du changement climatique à l'avènement d'un \emph{régime climatique}, nous verrons comment les discussions autour des enjeux climatiques se sont structurés. On cherchera à présenter les grands repères, tout en montrant comment l'histoire des négociations climatiques et des modèles intégrés sont intimmement liées, et s'influencent réciproquement. Cette partie permettra donc d'introduire l'influence qu'a le contexte sur les modèles et vice-versa. 

%Ressource : gouverner le climat

\subsection{Historique des négociations climatiques}
\label{sect:1.2.1}



\begin{figure}
    \centering
    \includegraphics[width=\linewidth]{illustrations/frise.png}
    \legende{Historique des négociations climatiques}{A faire sur Inkscape + regarder chez PBL s'ils ont pas déjà des choses comme ça}
    \label{fig:frise}
\end{figure}

\subsection{Le cadre général : CCNUCC et COP}
\label{sect:1.2.2}
Faire un retour de l'histoire des COP, de la CCNUCC 

\subsection{La synthèse des connaissances actuelles : le GIEC}
\label{sect:1.2.3}

\subsection{Loss and damages : dommages, responsabilité et évaluation}
\label{sect:1.2.4}

\section{Les outils : de la modélisation intégrée}
\label{sect:1.3}

Un des outils phare développé pour comprendre le changement climatique sont les \gls{iam}.

\begin{figure}
    \centering
    \includegraphics[width=0.9\linewidth]{figures/spm2_5.png}
    \legende{Trajectoires présentées dans le rapport de synthèse du GIEC}{Les rapports du GIEC présentent des trajectoires. Celles-ci sont obtenues à l'aide de modèles.}
    \label{fig:ipcc-pathways}
\end{figure}


\subsection{Les modèles intégrés, ou comment cartographier les dynamiques du monde}
\label{sect:1.3.1}

Se baser beaucoup sur l'article de Cointe 2024 + Gouverner le climat \\

La modélisation intégrée a pris beaucoup de place au fur et à mesure que le giec a pris de l'importance

\subsection{Le coût social du carbone}
\label{sect:1.3.2}

Le coût social du carbone désigne la valorisation économique de toutes les conséquences de l'émission de carbone. Il part du principe que le coût d'utilisation des énergies carbonées (en particulier, le prix de vente de l'essence, du gaz, etc.) ne reflète pas l'ensemble des coûts qui sont causés par cette utilisation. Il y donc création d'une externalité, c'est à dire d'un impact (en l'occurence, les dommages environnementaux) qui n'est pas pris en compte lors de la transaction (en l'occurence, l'achat de carburant). Cette situation est donc suboptimale, et le surcoût est supporté par la communauté, et non par les acteurs, qui dès lors prennent une décision qui est globalement désaventageuse. Ce concept s'inscrit dans une conception économique néolibérale des échanges et dans l'analyse économique du droit.  \\

Il s'agit d'un concept utilisé principalement par les administrations des Etats-Unis, qui doivent prendre en compte cette mesure dans l'évaluation d'impact des différents projets qu'elles mettent en place. La définition donnée par l'Académie Nationale des Sciences des Etats Unis est la suivante : \emph{The social cost of carbon is « defined for a given year as the present discounted value of the future damage4 caused by a 1 metric ton increase in CO2 emissions to the atmosphere, in that year, or, equivalently, the benefits of reducing CO2 emissions by the same amount in that year. » }\footnote{Le coût social du carbone est défini pour une année donnée comme la valeur actualisée des dommages futurs causés par une tonne de CO2 émises dans l'atmosphère, ou, de manière équivalente, aux bénéfices apportés par la réduction des émissions de CO2 de la même quantité.} \\

Depuis plus de 30 ans, le coût social du carbone est calculé par un groupe de travail inter-agences fédérales qui évalue et donne une valeur chiffrée à une tonne de carbone. Des décrets présidentiels ont progressivement fixé les conditions dans lesquelles les agences doivent tenir compte de cette mesure. Désormais, toutes les agences fédérales doivent considérer l'impact en termes d’émissions de CO2 dans leurs évaluations d'impact, en tenant compte de cette valeur monétaire. \\

\begin{quote}[National Academy of Science, 2017]
     In deciding whether and how to regulate, agencies should assess all costs and benefits of available regulatory alternatives, including the alternative of not regulating. Costs and benefits shall be understood to include both quantifiable measures (to the fullest extent that these can be usefully estimated) and qualitative measures of costs and benefits that are difficult to quantify, but nevertheless essential to consider. Further, in choosing among alternative regulatory approaches, agencies should select those approaches that maximize net benefits. 
\end{quote}

Les valeurs du SCC sont calculées par le groupe de travail inter-agences sur le coût social du carbone. Pour ce faire, le groupe de travail utilise trois modèles (FUND, DICE et PAGE) et exécute des simulations en faisant varier aléatoirement les paramètres. La valeur qui est conservée est la moyenne de l'ensemble des simulations. 

\begin{figure}
    \centering
    \includegraphics[width=0.9\linewidth]{figures/scc.png}
    \legende{Distribution du Coût Social du Carbone selon les simulations et le taux d'actualisation (en \$/T $CO_2$)}{Le groupe de travail inter-agences pour le coût social du carbone réalise de nombreuses simulations à partir des modèles FUND, DICE et PAGE. Chaque simulation abouti à une estimation du SCC, dont les distributions sont représentées ici. Elles sont séparées par taux d'actualisation, c'est à dire la valeur que perdent les dommages chaque année. Plus cette valeur est importante, plus la préférence pour le présent est importante, et moins les dégâts futurs sont reflétés dans le SCC. On voit que ces valeurs varient considérablement selon le choix du taux d'actualisation. D'après \cite{national_academy_of_sciences_valuing_2017}}
    \label{fig:scc}
\end{figure}

\subsubsection{Historique du concept}

Le coût social du carbone 

\subsubsection{Intérêt}

\subsubsection{Critiques}




\begin{figure}
    \centering
    %\includegraphics{}
    \legende{Estimation des coûts sociaux du carbone par le groupe inter-agence}{Depuis 30 ans, les agences fédérales des États-Unis doivent inclure le coût social du carbone dans leurs études d'impact. Celui-ci est estimé à partir de trois modèles intégrés : RICE, FUND et PAGE. Sa valeur est très sensible du taux d'actualisation.}
    \label{fig:scc}
\end{figure}

\subsection{Les fonctions de dommage}
\label{sect:1.3.3}

\begin{figure}
    \centering
    \includegraphics[width=4cm]{figures/campus.jpg}
    %\begin{tikzpicture}[scale=1]

\def\figureheight{20} % Choisis la hauteur désirée
\def\nodedistance{\textwidth*1/3}
\def\verticaldistance{2cm}
\def\nodewidth{3cm}

% Trajectory
\draw[line width = 2pt, rounded corners=8pt] (0,0) -- node[midway, below]{Les premières intuitions} (\textwidth,0) -- (\textwidth,-\figureheight*1/3) -- node[midway, below]{L'essor de la climatologie}(0,-\figureheight*1/3) -- (0,-\figureheight*2/3) -- node[midway, below]{Naissance du \textit{régime climatique}}(\textwidth,-\figureheight*2/3);


% Phase 1 : les premières intuitions


\node (1822) at (0,0.5) {1822};
\node (fourrier) [rectangle, draw, above of = 1822, text width=\nodewidth, text centered, yshift=2cm]{
Fourrier théorise l'effet de serre \\
\includegraphics[width=\linewidth]{images/Fourier2.jpg}
};

\node (1859) [right of = 1822, node distance = \nodedistance] {1859};
\node (tyndall) [rectangle, draw, above of=1859, text width=\nodewidth, text centered]{John Tyndall};

\node (1896) [right of = 1859, node distance = \nodedistance] {1896};
\node (farrhenius) [rectangle, draw, above of=1896, text width=\nodewidth, text centered]{Svante Arrhenius};

\node (1938) [right of = 1896, node distance = \nodedistance] {1938};
\node (callendar) [rectangle, draw, above of=1938, text width=\nodewidth, text centered]{Guy Callendar};

% 1822 : Joseph Fourrier
% 1859 : John Tyndall
% 1896 : Svante Arrhenius
% 1938 : Guy Callendar


% Phase 2 : l'essor de la climatologie

% Phase 3 : l'essor du régime climatique


% Information boxes
\draw[fill=blue!20] (0.5,-1) rectangle (2.5,-2);
\node at (1.5,-1.5) {Création du GIEC};
\draw[fill=green!20] (7.5,-1) rectangle (9.5,-2);
\node at (8.5,-1.5) {Première COP};
% Add more information boxes as needed
\end{tikzpicture}
    \legende{Évolution jointe de la modélisation et des négociations climatiques.}{Les modèles climatiques ont permis en premier de concevoir et de détecter le changement climatique (A). Leur essor a accompagné le cadrage de la question climatique (B), et ils sont désormais un outil de prise de décision (C).}
    \label{fig:frise}
\end{figure}

\ref{sect/1/1}




\chapter{La relative diversité de la représentation des dommages}
\label{chapter:litrev}
\newrefsegment

\PEEL{Les manières de représenter les dommages du changement climatique varient selon leur niveau de désaggrégation, le type de modèle utilisé (simulation / optimisation), leur calibration et les phénomènes qu'elles prennent en compte. }{Nordhaus et Stern se disputent sur la valeur du taux d'actualisation (avec pourtant les mêmes modèles). Le SCC est calculé aux US alors que les dommages ne sont pas pris en compte dans les modèles de l'IIASA database. Il y a une forte utilisation des indicateurs économiques. }{Ces différents duels montrent la diversité de choix possibles offerts aux modélisateur.ices.}{Répondre à ces questions est un choix important, qui dépasse le pure cadre technique pour rejoindre la dimension éthique.}


\chapterabstract{Les fonctions de dommage permettent de modéliser les impacts du changement climatique sur d'autres parties du modèle. Elles peuvent varient par leur existence, forme, calibration ou par les paramètres ou secteurs qu'elles prennent en compte. Pourtant, un changement dans ces fonctions peut radicalement faire changer le fonctionnement d'un modèle, et ainsi les résultats et les conclusions qu'on en tire. Cette partie s'intéresse donc aux différentes fonctions de dommage qui existent dans la littérature.}

%%Accroche
Les modèles intégrés sont ainsi derrière de nombreuses publications qui informent le débat public : rapports du GIEC, Stern Review, revue du coût social du carbone. Ils ont donc un rôle de premier plan dans la prise de décisions sur les questions climatiques, sur les options qu'ils estiment possibles ainsi que sur les conséquences anticipées de telle ou telle action. Un élément clé de cette modélisation est la prise en compte des impacts climatiques. 


%%Définition des termes

%%Rappel du sujet
Les différentes formes de fonction de dommage dans les modèles intégrés

%%Problématique
Quelles sont les fonctions de dommages utilisées dans les modèles intégrés ? Et, plus précisement, quels modèles utilisent des fonctions de dommage ? Quels phénomènes sont représentés ? Quelles variables entrent en compte, et comment sont-elles paramétrisées ? 

%% Annonce du plan
Dans une première partie, nous détaillerons la méthodologie utilisée pour obtenir la base de donnée des fonctions de dommage. Dans une seconde partie, nous la décrirons avec différentes statistiques descriptives. Enfin, nous aborderons trois points critiques des fonctions de dommage : leur forme; leurs paramètres; et la calibration. 

\begin{methodbox}[Revue de la littérature]
Pour obtenir une base de données des différentes fonctions de dommage, nous avons cherché à fusionner plusieurs sources de données. D'abord, l'IAM Consortium publie sur son site internet les documentations de nombreux modèles intégrés, sous la forme d'un wiki. Ces fiches sont rédigées par les équipes des modèles - ce qui permet d'avoir une source primaire sur les informations concernant les modèles - mais sont souvent incomplètes. En revanches, des "cartes", qui détaillent les principales caractéristiques de chaque modèle, sont également disponibles. Ce sont principalement celles-ci qui sont utilisées dans la base de données. Une autre source de données est le fichier des scénarios utilisés par le GIEC. Celui-ci permet d'avoir des informations sur chacun des scénarios soumis pour l'AR6, et d'avoir accès à de nombreuses informations : modèle utilisé, vetted ou non, impacts climatiques pris en compte ou non. A ces différentes sources, on ajoute manuellement des modèles, basée sur une lecture aléatoire de la littérature. Ils comprennenent notablement les modèles utilisés par l'agence interagence du coût social du carbone, ceux cités par Souffron et Jacques, ceux utilisés par les SSP, ainsi que d'autres modèles intégrés trouvés par littérature interposée. 
Le monde de la modélisation est très vaste, les modèles souvent compliqués à comprendre et parfois peu transparents. Ainsi, il a toujours été préféré de se baser sur des sources explicites. Bien qu'un véritable effort pour chercher à avoir une vision sur le plus de modèles possibles, cette étude ne peut pas être considérée comme un recensement exhaustif des modèles intégrés ni de leurs fonctions de dommage. \\ \gls{latex}

Une fois cette première liste de modèles obtenue, un premier tri est effectué entre ceux qui intègrent une fonction de dommage et ceux qui n'en intégrent pas. On considére ici les fonctions de dommages explicitement définies telles quelles, bien que d'autres fonctions puissent in fine avoir un comportement similaire. Ainsi, pour certaines, on a une connaissance explicite : par exemple, les fiches de l'IAMC comportent une case sur les impacts modélisés. Pour les autres, on considère qu'elles n'ont pas de fonction de dommage si les termes "damage function" ou "damage" ne sont pas présents dans leur documentation ou les publications associées, et s'il n'est pas fait mention de fonctions de dommage dans d'autres sources. \\

Pour chaque modèle incluant des fonctions de dommage, on cherche dans sa documentation la description de ces fonctions de dommage. La plupart du temps, celle-ci comporte une équation et les variables associées. Un script Chat-GPT est alors utilisé sur la partie du document qui est décrit la fonction de dommage. Celui-ci interprète et met en forme (sous la forme d'un tableau CSV) toutes les variables présentes dans cette fonction. Elles sont alors contrôlées visuellement. Cette étape est repétée pour chaque fonction de dommage. Une fois les variables de chaque fonction de dommage d'un modèle identifiées, ce fichier est téléversé dans la base de données, à l'aide du logiciel Airtable, dans la table 'Variable'. \\

Une fois les variables insérées dans leur table, les fonctions de dommage sont incluses dans la table 'Damage functions'. Chaque fonction est assortie à un nom, soit celui-donné dans la publication, soit choisi selon le contexte. Sont également ajoutés le nom du modèle, le numéro de l'équation, l'annotation zotero et le DOI de la publication, afin de pouvoir retrouver rapidement la fonction de dommage. Sont alors ajoutés d'une part les variables qui viennent en entrées de l'équation, et la variable qui sur laquelle l'équation agit. \\

Enfin, une autre table est ajoutée : celle des risques identifiés par le GIEC. Ceux-ci sont issus du rapport de synthèse de l'AR6, et la classification est faite par l'auteur. Lorsqu'une fonction de dommage décrit un des risques identifiés par le GIEC, elle se voit liée à celui-ci. 



\end{methodbox}

\begin{figure}
    \centering
    %\includegraphics{figures/campus.jpg}
    \legende{Processus de sélection des modèles.}{Les modèles sont sélectionnées selon plusieurs critères : présence dans une des bases de données (GIEC, SCC, SSP, IAMC) ou dans une review. Ils sont ensuite comparés selon leurs caractéristiques.}
    \label{fig:méthodo-litrev}
\end{figure}

\section{Tour d'horizon des modèles et de leurs fonctions de dommage}

Les modèles intégrés sont très variés. Ils ont des histoires différentes (issus de l'énergie, de l'économie ou du climat), des perspectives différentes (en particulier entre les États-Unis ou l'Union européenne), des questions différentes, des choix de modélisation différents (simulation ou optimisation). 

Revue de la littérature se classe en 3 catégories : la construction des IAMs, la comparaison des résultats des IAMs, les articles iam par iam

\subsection{Une littérature composée d'article par modèle ou de méta-analyses}

La littérature existante s'articule autour de trois axes. D'abord, il y a une abondante littérature autour des modèles. En effet, chaque modèle fait l'objet de nombreuses publications pour en présenter la structure, puis pour en donner les résultats. Il y a par ailleurs de nombreuses ressources appartenant à la littérature grise : des blogs, du code source, des rapports aux commanditaires, des modèles, etc. Ensuite, il y a des méta-analyses qui comparent les résultats des modèles; enfin, tout un pan de la littérature s'intéresse à la structure même des modèles, à travers des comparaisons entre modèles. 

\subsubsection{Documentation et résultats de chaque modèle}

\paragraph{Les modèles sont accompagnés d'analyses}

La principale source d'information que l'on a sur les modèles est les publications scientifiques qui en présentent les résultats.
Il s'agit de publications qui démontrent quelque chose ou répondent à une question (voir, par exemple, \cite{int_panis_externe_2000, dafermos_how_2021, baumstark_remind21_2021, dafermos_stock-flow-fund_2017, cherp_global_2016, burke_global_2015}). Le modèle et/ou la fonction de dommage y sont peu décrits (au plus une équation sans paramétrisation), souvent dans la section méthodologie. Ces articles sont intéressants car ils permettent de comprendre la finalité des modèles, les conditions pour lesquelles ils ont été conçus. En revanche, ils ne permettent pas à eux seuls de reproduire l'analyse ou de la modifier. 


\paragraph{Le niveau de documentation et d'ouverture varie considérablement}

Une autre source de documentation est la littérature grise qui accompagne les publications scientifiques. On en trouve des formes très variées, telles que des rapports (voir \cite{medeas_guiding_2019, asbjorn_aaheim_grace_2018, european_commission_ginfors-e_2022, european_commission_gem-e3_2013, nordhaus_dice_2013}), des \emph{working papers} (voir par exemple \cite{dafermos_stock-flow-fund_2017, giraud_coping_2016, calvin_gcam_2019, bosetti_witch_2006, ghersi_imaclim-p_2014}), des blogs (voir par exemple \cite{tol}

\paragraph{}

\subsubsection{Sur les résultats des IAMs}





Meta analyse des estimations du niveau du changement climatique
\cite{howard_few_2017} => 

\cite{gillingham_modeling_2018} gillingham => peut être à mettre dans la partie \ref{chapter:modelisation} ?

\cite{keppo_exploring_2021} => grosse revue de littérature sur les IAMs

\cite{harmsen_integrated_2021} => évaluation des IAMs à travers une méthodologie précise




\subsubsection{Sur la structure des IAMs }

\paragraph{Il existe des fonctions de dommage, mais sous des modèles semblables entre eux et orthodoxes.}

Dans une importante revue de littérature sur l'évaluation des dommages climatiques par les modèles intégrés, \cite{diaz_quantifying_2017} passe en revue les caractéristiques et contraintes liées aux fonctions de dommages. Elle inspecte en particulier DICE, FUND et PAGE, qui sont trois modèles conçus spécifiquement pour évaluer monétairement les impacts. Elle formule de nombreuses critiques : l'extrapolation à de hautes températures ou à d'autres régions, le choix des impacts représentés, l'absence d'interaction inter-régionales ou temporelles, l'absence de représentation de l'adaptation, des données scientifiques datées, la représentation lacunaire de l'incertitude, et d'autres encore. 

Cet article a servi de réference à la suite de ce travail. Néanmoins, les modèles représentés sont similaires, avec des objectifs similaires et produits par des communautés scientifiques proches. Or, la plupart des critiques faites par Diaz sont liées à des méthodes de modélisation ou  à des hypothèses générales. Pour comparer ces modèles et défier ces hypothèses, il faut sortir de cette communauté.  \\


Nous nous intéressons donc à d'autres revues de littératures, plus large, mais permettant d'avoir un plus large panel de modèles. 

\paragraph{Chercher des modèles plus divers permet de répondre à certaines des critiques}


Par exemple, \cite{souffron_successful_2024} cherchent à explorer la diversité des modèles, et montrent qu'un plus large panel de modèles permet de mieux guider les politiques publiques. Ils montrent que les modèles orthodoxes sont particulièrement limités. Ils présentent ensuite des modèles alternatifs, mettant en avant les atouts et contraintes de chacun. Enfin, ils émettent des recommandations de caractéristiques que pourraient avoir des modèles idéaux, notamment le fait d'avoir des fonctions de dommage. 






\cite{souffron_successful_2024} => présente plein de modèles et les commente. Pas vraiment une revue de littérature sur les damage functions mais plus général, permet de bien se repérer dans la littérature + grosse emphase sur les modèles hétérodoxes

\cite{review of information on models}  Une revue des différents modèles 

\paragraph{D'autres revues comparent des hypothèses plus varéies dans les modèles}

\cite{krey_looking_2019} => comparaison des hypohtèses technologiques des modèles

Sur les types de modèles : 
\cite{mercure_modelling_2019} => présente les principales différences entre les modèles de simulation et d'optimisation, et montre que c'est aussi une différence paradigmatique. 

\paragraph{La place particulière de l'IAMC}


Il y a une multitude de ressources liées à la modélisation des dommages dans les modèles intégrés. Il n'existe pourtant pas de plateforme unique permettant de recenser toutes les formes de fonctions de dommage, leurs effets, atouts et contraintes. 

\section{Méthode : construction de la base de données}

\begin{table}[]
    \centering
    \begin{tabular}{c|c}
         &  \\
         & 
    \end{tabular}
    \legende{Les différents modèles recensés et leurs fonctions de dommage}{Description rapide}
    \label{tab:my_label}
\end{table}

\subsection{Premiers indicateurs quantitatifs}

\begin{figure}
    \centering
    %\includegraphics[width=0.5\linewidth]{}
    \legende{Visualisation graphique des différents modèles}{Description rapide}
    \label{fig:enter-label}
\end{figure}

\subsection{Les principaux modèles}

On présente ici rapidement des modèles, leurs caractéristiques générales et la forme de leur fonction de dommage. Il s'agit surtout de montrer la diversité des approches de modélisation, et non d'avoir un recensement systématique. Les lecteurs plus curieux trouveront plus de caractéristiques en suivant le lien. \blog{https://damage-functions-modeling.readthedocs.io/en/latest/1_introduction/functions.html}

\subsubsection{DICE}

DICE veut dire XXX et constitue une des premières tentatives de modéliser les relations entre l'économie et le climat. Il comporte une fonction de dommage, extremement simplifiée. 

\begin{equation}
\begin{array}{ll}
    & \displaystyle  \Delta = \psi_{1}T_{AT}(t) + \psi_{2}[T_{AT}(t)]^{2} \\
    & = [0.0]T_{AT}(t) + [0.003467][T_{AT}(t)]^{2}
\end{array}
\label{eq:df_dice2023}
\end{equation}

\subsubsection{FUND} 

A l'inverse, FUND présente un niveau de désagrégation assez avancé. Il y a de nombreuses fonctions de dommage, qui sont utilisées les unes dans les autres. Leur forme générale est d'obtenir un niveau de dommage à partir d'une elasticité entre un phénomène et un autre. 

Par exemple, l'impact du changement climatique sur l'agriculture est construit par la somme de trois canaux : d'une part, l'impact du niveau de changement climatique; d'autre part, la vitesse de ce changement (représenté en \ref{eq:fund_A2}); enfin, l'augmentation de la fertilisation des plantes. 

\begin{equation}
    A_{t,r}^{r}=\alpha_{r}\left(\frac{\Delta T_{t}}{0.04}\right)^{\beta}+\left(1-\frac{1}{\rho}\right)A_{t-1,r}^{r}
    \label{eq:fund_A2}
\end{equation}

D'autres impacts sont aussi représentés, tels que l'impact du niveau des eaux sur le littoral, les écosystèmes, les domages liés aux cyclones tropicaux et extra-tropicaux, etc. , et ce par différents biais : des dommages directs (monétaires), et la monétarisation de perte de vie ou de perte de temps de vie. 

\begin{equation}
    D_{t,r}^{\nu}=D_{1990,r}^{\nu}Q_{r}^{\nu}\left(T_{t}-T_{1990}\right)^{\beta}\left(\frac{y_{t,r}}{y_{1990,r}}\right)^{\gamma}
    \label{eq:fund_HV}
\end{equation}

\begin{equation}
    V S L_{t,r}=\alpha\left(\frac{y_{t,r}}{y_{0}}\right)^{\gamma}
    \label{eq:VSL}
\end{equation}

\subsubsection{Autres formes quadratiques}

Les formes quadratiques issues de DICE/RICE ont été grandement critiquées (voir section \ref{ss:forme}), et d'autres formes quadratiques sont apparues. C'est le cas notamment dans DEFINE (\ref{eq:DEFINE}), dont on peut trouver une forme très similaire dans le Giraud Stock-Flow consistent model. Il s'agit d'équations qui, comme DICE, sont extrememnt simplifiées et aggrégée, mais la forme de la relation est différente. 

\begin{equation}
    \Delta T = 1 - \frac{1}{1 + \eta_1 TAT + \eta_2 TAT^2 + \eta_3 TAT}
    \label{eq:DEFINE}
\end{equation}

\subsubsection{Autres formes non quadratiques}



\section{Trois questions centrales}


\subsection{La forme : comment représenter un phénomène qui n'existe pas encore ?}
\label{ss:forme}

La première difficulté quant à la représentation des impacts du changement climatique est le choix de la forme de la fonction de dommage. En effet, les phénomènes à l'origine des dommages climatiques sont à la fois incertains et récents. D'abord, les phénomènes physiques, dont on a du mal à avoir une représentation fiable et précise (voir notamment la section \ref{fig:tipping-point} sur les tipping points). Ensuite, et peut être plus encore, l'interaction entre ces évenements et les sociétés humaines est extremement difficile à comprendre, et donc à modéliser. Exercice de simplification, la modélisation nécessite pourtant d'extraire des lois générales de phénomènes particuliers, pour pouvoir établir des relations entre eux. \\

Ainsi, il est déjà très difficile de choisir une forme fonctionnelle qui soit adaptée. On peut contrer cet argument en avançant que la pratique de la modélisation doit assumer cette simplification. De ce point de vue, un modèle ne vise pas à reproduire la réalité ou des mécanismes réels, mais seulement à fournir un cadre explicatif suffisant pour rendre intelligible des choses qui ne l'étaient pas avant. \\

Le caractère récent des impacts du changement climatique, et encore plus de la prise en compte de ces effets en tant qu'effets du changement climatique, complique encore un peu la tâche. En effet, les séries temporelles étant courtes, il est difficile d'en extraire des formes fonctionnelles qui correspondraient effectivement à la relation entre les différentes variables. \\

Enfin, ces sources de données ne couvrent qu'un intervalle d'anomalie de température assez faible. En effet, on estime le réchauffement climatique actuel à environ 1.2 \textdegree C par rapport à la période pré-industrielle. Cependant, les estimations vont plus haut : l'accord de Paris pour le climat vise à avoir une réchauffement limité à 2 \textdegree C , et le plus proche possible de 1.5 \textdegree C; des estimations vont plus loin encore. Ainsi, des fonctions calibrées sur l'intervalle $[+0; +1.2]$ pourraient ne pas du tout capter les phénomènes observés dans le futur, sur un intervalle plus grand. Là réside un des grands défis de la modélisation des dommages : il s'agit de modéliser des phénomènes qui n'existent pas encore, et qui surviendront dans un contexte probablement très différent sur de nombreux aspects de celui dans lequel a lieu la modélisation. 

\begin{figure}[ht]
\centering
\includegraphics[width=\textwidth]{results/shape.png}
\legende{Niveau de dommage en fonction de l'augmentation de température}{Différentes formes fonctionnelles peuvent avoir un pouvoir explicatif important sur l'intervalle $[+0; +1.2]$, sur lequel on dispose de données empiriques, tout en divergeant grandement pour des valeurs plus haute de changement de température. Une fonction de dommage correctement calibrée pour des phénomènes observés n'est pas forcément calibrée pour les phénomènes futurs.}
\end{figure}



\begin{figure}
    \centering
    %\includegraphics{}
    \legende{Analyse en composantes principales des modèles.}{Les modèles sont classés selon beaucoup de composantes, pour identifier des similitudes ou des patterns.}
    \label{fig:ACP}
\end{figure}

\subsection{Les paramètres : quel niveau de complexité faut-il, et que prendre en compte ?}

\begin{figure}
    \centering
    \includegraphics[width=0.9\linewidth]{figures/sankey.png}
    \legende{Sankey diagram des différentes fonctions}{Les modèles ne prennent pas en compte tous les mêmes données en entrées, et elles ne permettent pas d'expliquer les mêmes phénomènes.}
\end{figure}

\subsection{La calibration : un \textit{"tiens"} vaut-il deux \textit{"tu l'auras"} ?}

\section{Trois querelles pour répondre à ces questions}

\subsection{Le taux d'actualisation : la querelle Nordhaus / Stern}

\cite{guigourez_10_2023} => description de la querelle

\begin{figure}
    \centering
    \includegraphics[width=\linewidth]{figures/actualisation.PNG}
    \legende{Valeur actualisée selon le temps et le taux d'actualisation}{Graphique qui représente la valeur d'une unité monétaire dans le temps, actualisée au présent, selon le taux d'actualisation. En jaune, celui choisi par Nordhaus, et en bleu celui choisi par Stern. Clé de lecture : avec un taux d'actualisation de 3\%, 100\$ de dommage en 2080 ne valent que 18\$ en 2020, alors que 100\$ de 2040 valent 56\$. }
    \label{fig:discount-rate}
\end{figure}

\subsection{Pourquoi représenter les dommages ? Les SCC vs le reste du monde}

Comme nous avons pu le voir, les critiques quant à la quantification et la monétarisation des dommages sont déjà nombreuses. Se pose donc la question suivante : faut-il représenter les dommages, et si oui, pourquoi et comment ? Pour éclairer ce dilemne, nous allons nous pencher sur deux une distinction particulière, entre les modèles qui représentent les dommages et ceux qui ne le font pas. \\

D'une part, certains modèles représentent des dommages de manière explicite. Ce sont essentiellement des modèles utilisés pour mesurer le coût social du carbone. Les arguments qui sont avancés par leurs défenseurs sont solides. Nous vivons dans un monde où la valorisation économique est centrale. Conformément à la théorie économique, et notamment à celle consacrée aux biens communs et aux externalités, si on ne peut pas compter la valeur de quelque chose, et surtout qu'elle ne s'impose pas à nous lors des transactions, alors cette valeur est invisible. De ce point de vue, et malgré les nombreuses limites (souvent assumées d'ailleurs) de la modélisation de ces phénomènes, il est nécessaire de représenter les dommages dans le modèle, même de manière imparfaite. En effet, selon ce point de vue, ne pas les représenter revient à les négliger, les omettre, ou en d'autre terme, à leur attribuer arbitrairement une valeur nulle. 
C'est dans cette perspective que se place de nombreux modèles, qui cherchent à proposer de nouvelles manières de représenter les dommages. Nombreux sont ceux qui, en réponse à la simplicité de la représentation de Nordhaus, ont proposé des fonctions de dommage toujours plus complexe, prenant en compte plus de mécanismes. 
\\

D'autre part, on peut argumenter que la représentation de ces dommages donne une fausse sensation de certitude. D'une part, elle done l'impression que le modèle est plus réaliste, en ce sens que ce qu'il dit \textit{serait} plus proche de la réalité et moins sujet aux travers de la simplification car justement plus complexe. Pourtant, on peut argumenter que la représentation des dommages, y compris quand elle est sophistiquée, fait appel à de nombreuses hypothèses, notamment concernant la permanence des phénomènes qui ont lieu. On a alors un modèle qui non seulement n'est pas forcément plus fiable ou plus proche de la réalité, mais qui est en plus nettement moins lisible, compréhensible - et dont les limites ne peuvent que très difficilement être interprétées et critiquées. 

\subsection{Compter ce qui n'a pas de prix : la difficile monétarisation}



\begin{figure}
    \centering
    \includegraphics[width=\linewidth]{results/carte_progressive.png}
    \legende{Double progression du niveau de dommage}{Le niveau de dommage suit une double progression. D'une part, il augmente avec le temps (qui est lui même très lié au niveau de réchauffement); d'autre part, il augmente avec le modèle : certains modèles donnent des niveaux de dommage plus importants que d'autres.}
    \label{fig:carte-progressive}
\end{figure}



\chapter{Rendre visible : quantifier les choix éthiques}
\label{chapter:modelisation}
\newrefsegment
\PEEL{Les différents choix de modélisation abordés plus haut ont une variation sur le niveau de dommage importantes, presque aussi / plus importante que d'autres facteurs souvent pris en compte (incertitude, scénario climatique).}{On utilise une analyse économétrique pour essayer de quantifier ce phénomène. }{Même si les résultats ne sont pas super forts, on voit qu'il y a une variation importante qui est issue de la variation de ces paramètres éthiques.}{Le modèle ne donne pas de vérité absolue, il sort ce qu'on y a fait entrer => les modèles sont profondément politiques, et reflètent en ce sens une certaine conception du monde.}




\chapterabstract{Ce chapitre propose une approche plus économétrique des effets de la modélisation des fonctions de dommage. On utilise le modèle WILIAM, qui pourrait devenir un modèle de référence de la commission européenne, auquel on change les fonctions de dommage. On le fait tourner avec de nombreux scénarios, pour obtenir des résultats. On réalise ensuite une étude économétrique de ces résultats, pour savoir qu'elle paramètre a le plus d'effet sur les résultats du modèle.}

\newpage

%\cite{errickson_equity_2021} => il peut y avoir une plus grande variation du SCC selon l'équité que selon l'incertitude climatique

Nous avons vu dans le chapitre précédent que les modèles varient dans leurs choix de représentations des dommages. De plus, ces différents choix semblent tous valables à leur manière : les phénomènes représentés sont incertains, et cette variabilité représente surtout l'incertitude qui entoure ces phénomènes. Pourtant, ce sont bien les mêmes phénomènes; il ne devrait donc pas y avoir de différence entre les modèles. Ces différences sont donc apportées exclusivement par les choix de modélisation; or, elles peuvent être importantes, et donner des conclusions radicalement différentes, donnant lieu à des interprétations opposées. La querelle entre Nordhaus et Stern, évoquée plus haut, en est un exemple éloquent. 

Il est donc intéressant de voir dans quelle mesure ces variations influencent le comportement du modèle. En effet, si elles sont peu importantes comparées à d'autres facteurs, elles ne compromettent pas le message du modèle. En revanche, si elle joue un rôle central dans le fonctionnement de celui-ci, alors il convient de leur accorder une place majeure, tant dans l'analyse des résultats que dans la conception. 

\begin{figure}
    \centering
    \includegraphics[width=\linewidth]{results/carte_progressive.png}
    \legende{Double progression du niveau de dommage}{Le niveau de dommage suit une double progression. Chaque ligne correspond à une décennie; chaque colonne correspond à une fonction de dommage. D'une part, il augmente avec le temps (qui est lui même très lié au niveau de réchauffement); d'autre part, il augmente avec le modèle : certains modèles donnent des niveaux de dommage plus importants que d'autres. On voit ici a quel point les variables méthodologique (en l'occurence, le choix de la fonction de modèle) impactent le niveaux de dommage, d'une manière presque aussi importante que l'augmentation de la température.}
    \label{fig:carte-progressive}
\end{figure}

On cherche donc à mesurer quantitativement les effets de ces variables sur les résultats du modèle pour répondre à la question suivante : les variations induites par les choix éthiques sont-elles du même ordre de grandeur que les variations issues de variables physiques ou de choix de modélisation ? 



On définit ici la notion de paramètre éthique : elle désigne les paramètres qui n'ont pas vocation à représenter un phénomène physique ou social observable, mais plutôt à représenter une valeur morale. Le taux d'actualisation, par exemple, entre tout à fait dans cette définition. En effet, il représente l'importance accordée à l'équité intergénérationnelle. bien qu'il y ait eu des tentatives d'observer empiriquement cette valeur, notamment en s'intéressant au taux de préférence pure pour le présent dans les prêts de long terme, il semble difficile de penser que cette valeur soit valable à plus long terme ou dans d'autres contextes. Plutôt, il apparait que ce paramètre reflète un système normatif, et relève de ce fait plutôt du choix moral que du choix technique. Se pose alors la question suivante : quelle part de la variation du niveau total de dommage est-elle expliquée par ces paramètres éthiques.  On s'intéressera ici à une autre dimension éthique, qui est celle de l'équité spatiale. Pour ce faire, on construit un pondérateur géographique, qui pondère le niveau de dommage par région selon le niveau de revenu.

Dans cette partie, on développe un protocole expérimental qui vise à mesurer le niveau de dommage produit par différents modèles, différentes fonctions de dommages et des variations du pondérateur géographique. On régresse ensuite les données obtenues pour mesurer quantitativement la part de la variation qui est expliquée par ces différents paramètres. 

\section{Cadre conceptuel}

Le cadre conceptuel général de cette expérience est celui de la modélisation intégrée. Il s'inscrit en cela dans la lignée des réflexions développées dans le chapitre \ref{chapter:introduction}. Il semble néanmoins nécessaire de préciser quelques points. \\

\subsection{Choix du modèle}

D'abord, le choix du modèle. Pour réaliser cette expérience, nous avons choisi d'utiliser le modèle WILIAM, pour plusieurs raisons. 

\paragraph{Modèle Open-source} WILIAM est un modèle Open-Source, ce qui signifie que l'ensemble du code source est disponible en ligne, avec une licence permettant une utilisation, diffusion et modification du programme sans limites. C'est nécessaire puisqu'on a ainsi accès à tout le code source, ce qui permet de travailler sur une base solide; on peut également le modifier librement, ce qui est au cœur de l'expérience; enfin, on peut le diffuser librement aussi, ce qui est fait à travers le dépôt GitHub et le blog. Par ailleurs, bien que ce ne soit pas en soi une garantie de qualité, l'utilisation de l'open source permet d'améliorer la robustesse et la transparence des modèles, puisque chacun peut voir les hypothèses, proposer des modifications, et corriger des erreurs. 

\paragraph{Application aux politiques publiques} WILIAM est issu du projet LOCOMOTION, financé par la Commission Européenne. Il répond à une demande de celle-ci de s'équiper d'un nouveau modèle. Ainsi, WILIAM est un modèle qui vise à répondre à des questions de politique publique. 

\paragraph{Modèle complexe} WILIAM est un modèle particulièrement complexe : il compte près de 4 000 variables. Cela le rend particulièrement vulnérable aux problématiques de transparence et de complexité excessive que nous développerons plus tard. Néanmoins, cela nous permet d'avoir un modèle le plus complet possible. Cela permet d'intégrer les fonctions de dommage facilement, car les variables sont déjà existantes. 

\paragraph{Modèle de simulation} WILIAM est un modèle de simulation. Cette caractéristique est importante car elle permet de voir les valeurs de chaque paramètre à tout moment, mais aussi de représenter des paramètres non monétaires. De plus, ça le rend moins sensible aux critiques de normativité inhérente aux modèles d'optimisation. 

% \section{Présentation des données}


\section{Méthode}

\subsection{Données}


\begin{methodbox}[Modélisation des fonctions de dommage dans WILIAM]

Pour modéliser les fonctions de dommage dans WILIAM, on commence par créer un nouveau repo sur GitHub, dans lequel on télécharge le code source de WILIAM. On s'assure que tout est fonctionnel, et que le modèle tourne bien. La visualisation du modèle se fait grâce au logiciel VENSIM. \\

On rassemble les sources de données disponibles des modèles : code source, documentation et / ou publications qui y font référence. Ces sources n'offrent pas le même niveau d'information. Par exemple, les modèles dont on ne trouve pas le code source mais uniquement des publications sont plus difficiles à modéliser car il manque souvent des informations cruciales, telles que les valeurs obtenues pour les paramètres. \\

Une fois la définition de la fonction identifiée, on les modélise à l'aide de l'interface graphique du logiciel VENSIM. On sauvegarde régulièrement en réalisant des \textit{commit} sur GitHub pour garder une trace de la progression.
    
\end{methodbox}

\begin{methodbox}[Conversion des paramètres selon les bonnes zones géographiques]

    Le modèle WILIAM comporte 35 régions. En revanche, les autres modèles n'ont pas nécessairement le même découpage géographique. Deux cas se distinguent : soit le modèle est agrégé au niveau mondial; on peut alors utiliser les fonctions de dommage région par région, sans besoin de les changer. Soit le modèle possède des régions, et il faut alors convertir les régions du modèle A en régions de WILIAM pour avoir la bonne paramétrisation. Ce deuxième cas de figure est particulièrement vrai dans le cas du modèle FUND, qui a des régions très différentes de celles de WILIAM, et qui repose beaucoup sur de la paramétrisation. Il faut donc convertir les paramètres des régions de FUND en régions de WILIAM. 

    Par ailleurs, il est à noter que si la documentation de FUND est très complète et accessible en ligne, les tables de paramètre ne sont pas disponibles dans un format facilement lisible tel que CSV. La méthode retenue a été de lire la page de la documentation (en Markdown), et de la découper table par table, puis valeur par valeur à l'aide d'un script python, qui permet ensuite d'enregistrer ces données dans un format lisible par VENSIM. Ces opérations sont longues et peuvent être source de nombreuses erreurs. Ainsi, le non-partage des données dans un format standard est un vrai frein à la reproductibilité. \\
    
    On utilise alors un script Python qui désagrège les zones de FUND en pays, et qui associe à chaque pays sa zone dans WILIAM. 
    
\end{methodbox}

La plupart des fonctions de dommage dépendent du niveau de revenu par habitant, de plusieurs manières. Le premier cas de figure est qu'elles donnent un niveau de dommage qui est exprimé en proportion du PIB. On a alors un niveau de dommage absolu qui est plus grand dans les régions plus riches, puisque le PIB est lui-même plus grand. Ce point est essentiel, en particulier dans les modèles d'optimisation. En effet, ces modèles cherchant à trouver la solution la moins couteuse, ils vont être beaucoup plus sensibles aux dommages dans les pays ayant un PIB plus important que dans les pays ayant un niveau de revenu plus faible. \emph{De facto}, ils vont chercher à minimiser les dommages dans les pays les plus riches, quitte à augmenter les dommages dans les pays les plus pauvres. \\

La deuxième manière est que la fonction de dommage intègre directement le niveau de revenu dans le calcul du niveau de dommage. Selon le signe de l'équation, cette méthode peut renforcer les effets dans les pays les plus pauvres, ou au contraire les augmenter. 

On peut finalement citer une technique permettant de monétiser des impacts non monétaires, tels que des morts ou des années d'invalidité. La technique utilisée est assez classique en économie : il s'agit de calculer la valeur d'une vie statistique, ou \emph{value of a statistical life} (VSL). Cette méthode part de l'hypothèse que la valeur de la vie dans un lieu donnée correspond à un certain multiple du niveau de revenu par habitant dans ce lieu. On peut prendre en exemple la VSL du modèle FUND, qui prend la forme suivante : 

\begin{equation}
    VSL_{t, r} = \alpha (\frac{y_{t,r}}{y_0})^\epsilon
\end{equation}
où $t$ désigne l'année, $r$ désigne la région, $\alpha$ et $\epsilon$ des paramètres permettant de calibrer la fonction, $y_{t,r}$ le revenu par habitant dans la région $t$ à l'instant $r$, et $y_0$ un revenu de référence.  \\

Ce choix de modélisation pose de nombreuses questions. 
D'abord, d'un point de vue pragmatique. Les pays les plus pauvres sont susceptibles d'être les pays les plus impactés par le changement climatique, d'une part par une exposition plus grande à des phénomènes extrêmes, et d'autre part par une situation de pauvreté préexistante qui fragilise la résilience. 
Ensuite, pour des raisons morales. Le principe de responsabilité commune mais différenciée sur le changement climatique, qui préside le régime climatique depuis sa formation, instaure une distinction claire entre les pays des Nords, qui ont bénéficié de l'émission de carbone pendant des décennies pour leur développement, et les pays des Suds, dont le poids historique est relativement limité. On pourrait penser qu'il est souhaitable de pénaliser grandement les impacts du changement climatique dans les pays des Suds, du fait de cette responsabilité moindre. 
Enfin,  on peut aussi envisager une perspective de justice sociale pour traiter de ce sujet. On peut par exemple imaginer que les dommages sur les plus pauvres, touchant des populations déjà fragiles, sont particulièrement impactants. Il pourrait être dès lors notre responsabilité de limiter ces impacts, à la manière du Maximin de Rawls. 

Notre propos ici n'est pas de défendre l'une ou l'autre de ces positions. Chacune défend un objectif et un système normatif différent. Elles ont toutes leurs arguments et leurs travers, et il n'est pas question d'affirmer que l'une d'elle est la seule manière possible de concevoir le problème. Il s'agit même plutot de l'inverse : montrer qu'une multitude de choix de modélisation existent, et que leurs implications sont radicalement différentes. 

En revanche, nous argumentons qu'il y a nécessairement, dans les pratiques de modélisation, le choix de l'une ou l'autre de ces conceptions.  Nous cherchons dès lors à comprendre les effets de ce choix sur le niveau de sortie du modèle, en appliquant un pondérateur géographique (voir encadré méthodologique) en sortie des fonctions de dommage. Le rôle de ce coefficient est précisément de  rendre visible l'existence de ce choix, et de permettre aux utilisateurs de varier les paramètres. 

\begin{methodbox}[Définition du pondérateur géographique]
Nous observons la forme de la valeur d'une vie statistique, développée dans FUND. Cette expression est une bonne référence, car : 

\begin{itemize}
    \item il s'agit d'une pratique courante
    \item elle représente un phénomène dont on pourrait penser qu'il ait la même valeur partout (en l'occurrence, la mort)
    \item 
\end{itemize}
\blog{https://damage-functions-modeling.readthedocs.io/en/latest/3_modelling/ethical_coef.html}

Cette expression permet de comparer un même phénomène (la mort), dans des contextes dans lesquels le revenu est différent. Elle est donc adaptée pour représenter un même phénomène (un niveau de dommage, au sens physique) dans un système de comptabilité différent (en l'occurrence monétaire). 

Nous l'adaptons ainsi : 

\begin{itemize}
    \item les variations du coefficient $\alpha$ permettent de faire varier le sens et l'intensité de la relation entre niveau de revenu et niveau de dommage comptabilisé
    \item les variations de $y_0$ permettent de faire varier la limite entre les pays considérés comme riches et les pays considérés comme pauvres
    \item les variations de $\epsilon$ permettent de faire varier la convexité de la relation, c'est-à-dire à quel point les valeurs pour les différents pays vont être différentes. 
\end{itemize}

On peut retrouver une représentation graphique de ces coefficients dans la figure \ref{fig:ponderateur_selon}.
\vspace{2cm}

\centering
%\includegraphics[width=0.75\linewidth]{figures/coef.png}

\end{methodbox}
\begin{methodbox}[Exécution des différents runs]
L'exécution des runs est automatisée à l'aide d'un script Python et de PySD. \blog{https://damage-functions-modeling.readthedocs.io/en/latest/3_modelling/run.html} Les résultats sont ensuite stockés dans des objets NetCDF4. 

    
\end{methodbox}

\subsection{Modèle économétrique}

\subsubsection{Au niveau régional}


On construit un modèle économétrique dont la forme générale est la suivante : 


\begin{align}
\label{eq:modele}
    \text{log(dommage)} = & \ \beta_0  + \underbrace{\beta_1 \cdot \text{temperature}}_\text{Variables physiques}  + \underbrace{\beta_2 \cdot \text{equation}}_\text{Variables méthodologiques} \\
    & + \underbrace{ \beta_3 \cdot \text{pondérateur géographique} + }_\text{variables éthiques}  + \epsilon
    \end{align}


Les résultats sont plus probants lorsque l'on passe le niveau de dommage au log. Ceci n'a rien d'étonnant. En effet, le modèle a été construit ainsi. En repartant de la manière dont le niveau de dommage est calculé et implémenté dans le dommage, on peut obtenir des équivalences entre ces différentes équations : 

\begin{align*}
& D_{t,r} = f_{dommages} \cdot GDP \cdot \text{pondérateur géographique}\\
\iff & D_{t,r} = f(\text{temperature}, \text{ autres facteurs})_{t,r} \cdot GDP_{t,r} \cdot \text{pondérateur géographique} \\
\iff  & D_{t,r} = \text{temperature}_{t,r}^{\beta_1} \cdot GDP_{t,r}^{\beta_2} \cdot \text{coef}^{\beta_3} \\
\iff & log(D_{t,r}) = log(\text{temperature}_{t,r}^{\beta_1} \cdot GDP_{t,r}^{\beta_2} \cdot \text{coef}^{\beta_3}) \\
\iff & log(D_{t,r}) = \beta_1 \cdot log(\text{temperature}_{t,r}) + \beta_2 \cdot log(GDP_{t,r}) + \beta_3 \cdot log(\text{coef})
\end{align*}
$D_{t,r}$ désigne le niveau de dommage; $f_{dommages}$ désigne la fonction de dommage, qui donne un premier niveau de dommage en fraction du PIB, généralement selon la température.  \\

On obtient ainsi quasiment la forme du modèle économétrique. Ceci peut interroger : en effet, que démontrons-nous, si ce n'est la validité empirique de ces équivalences ? En effet, le pouvoir de démonstration est faible. De toute façon, la causalité ne nous intéresse pas ici :  puisque l'on a \emph{construit} le modèle, on sait de façon certaine la causalité; c'est même une définition. Par ailleurs, on a un contrôle total sur les facteurs qui influencent nos données. En effet, dans les travaux empiriques, les données sont issues de collectes. Elles font l'objet de nombreux biais, et surtout sont influencées par de nombreux facteurs autres que ceux mesurés par le modèle. Ici, on construit ces données, et on peut donc savoir précisément quel facteur influence nos résultats. \\

L'objectif ici n'est pourtant pas de montrer une causalité, mais bien de mesurer l'ampleur de cette variation, et de pouvoir comparer les impacts des différents facteurs. Les régressions permettent de synthétiser ces variations en des coefficients uniques, \emph{ceteris paribus}.

L'objectif est de quantifier l'impact relatif des variables physiques, méthodologiques et éthiques. On réalise ensuite cinq régressions linéaires, qui sont des variations du modèle présenté plus haut. 


On a d'abord une régression simple du niveau de dommage selon la température : 

\begin{align}
\text{log(dommage)} = & \ \beta_0  + \beta_1 \cdot \text{temperature}  + \epsilon
\end{align}

Cette première étape nous permet de comparer les résultats de nos modèles plus complexes avec un modèle plus simple, pour s'assurer que prendre en compte les autres facteurs apporte effectivement un pouvoir explicatif complémentaire.  \\

On ajoute ensuite le pondérateur géographique, sans tenir compte à ce stade de la forme de l'équation ni contrôler par la région. 

\begin{align}
\text{log(dommage)} = & \ \beta_0  + \beta_1 \cdot \text{temperature}  + \beta_2 \cdot \text{pondérateur géographique} + \epsilon
\end{align}

On cherche maintenant à observer l'effet de la forme de la fonction de dommage : 

\begin{align}
\text{log(dommage)} = & \ \beta_0  + \beta_1 \cdot \text{temperature}  + \beta_3 \cdot \text{equation} + 
 \beta_2 \cdot \text{pondérateur géographique}  + \epsilon
\end{align}

On essaie également de contrôler les différences régionales en tenant compte de celles-ci grâce à une dummy. 

\begin{align}
\text{log(dommage)} = & \ \beta_0  + \beta_1 \cdot \text{temperature}  + \beta_2 \cdot \text{pondérateur géographique} + \beta_3 \cdot \text{région} + \epsilon
\end{align}

Finalement, le modèle le plus complet prend en compte à la fois la température (variable physique), l'équation (variable méthodologique) et le pondérateur géographique (variable éthique), tout en contrôlant par les régions. 

\begin{align}
\text{log(dommage)} = & \ \beta_0  + \beta_1 \cdot \text{temperature}  + \beta_2 \cdot \text{pondérateur géographique} + \beta_3 \cdot \text{région} + \beta_4 \cdot \text{équation} + \epsilon
\end{align}

\subsubsection{Au niveau mondial}

On poursuit l'analyse en agrégeant les données au niveau mondial. Cette démarche est plus proche de ce qu'il se passe dans les modèles d'optimisation tels que ceux utilisés pour le \gls{cout social du carbone}. Cette agrégation est une somme de tous les dommages pour chaque $ (\text{exposant}, \text{constante}) $, c'est-à-dire $ (\text{année}, \text{run}) $. On évalue ainsi le modèle suivant : 

\begin{align}
\text{log(dommage)} = & \ \beta_0  + \beta_1 \cdot \text{temperature}  + \beta_2 \cdot \text{log(coefficient)} + \beta_3 \cdot \text{log(constante)} + \beta_4 \cdot \text{équation} + \epsilon
\end{align}


\subsubsection{Évaluation de l'estimateur} \textcite{wooldridge_introductory_2016} propose cinq critères pour un bon estimateur. D'abord, on doit écrire notre modèle économétrique sous la forme d'une combinaison linéaire des variables explicatives. C'est bien le cas ici (voir équation \ref{eq:modele}). Ensuite, l'échantillon doit être aléatoire. Dans notre cas, nous avons fait varier de manière aléatoire les paramètre du modèle; on observe néanmoins des schémas dans la distribution des valeurs des niveaux de dommage; il est donc possible que cet échantillon ne soit pas tout à fait aléatoire. Cela s'expliquerait par exemple par des différences systématiques liées à la région (on observe d'ailleurs que lorsque l'on contrôle par la région, cet effet disparait).  Il faut ensuite avoir une variation des échantillons, ce qui est notre cas. Puis, il faut que l'espérance conditionnelle des termes d'erreur soit nulle. On observe bien cela ici. En revanche, les tests d'homoscédasticité montrent qu'il y a de l'hétéroscédasticité.  Ces différentes analyses sont développées sur le site \blog{https://damage-functions-modeling.readthedocs.io/en/latest/3_modelling/regprep.html}.

\begin{figure}
    \centering
    \includegraphics[width=\linewidth]{results/slr_4_2.png}
    \legende{Espérence conditionnelle des termes d'erreurs}{Cette figure représente la distribution des espérences des termes d'erreurs selon plusieurs  des conditions sur la variable explicative $x$. Cela signifie qu'il n'y a pas trop d'erreur systématique selon la valeur  de la variable explicative. On remarque néanmoins que la forme n'est pas symmétrique, avec des erreurs fortement négatives. }
    \label{fig:slr_4}
\end{figure}

L'échantillon ne respect pas les cinq critères de Wooldridge. De ce point de vue, on peut penser que ce n'est pas un très bon estimateur, et qu'il faut chercher à améliorer notre modèle. Cependant, nous ne cherchons pas ici à avoir un pouvoir explicatif, mais bien à avoir une mesure quantitative d'un phénomène dont on sait qu'il existe, puisque nous l'avons défini ainsi. 


\begin{methodbox}
On utilise ici le modèle WILIAM. Il s'agit d'un modèle open source, donc il est assez facile de le répliquer, modifier et distribuer. Le code source est disponible sur GitHub. On réalise un fork du code source c'est-à-dire une branche de code. Notre branche de code devient un projet autonome sur lequel on peut travailler de manière semi-indépendante du projet. En effet, les modifications du projet WILIAM ou celles réalisées ici ne s'affecteront pas, à moins que l'on push notre code vers le code principal, ou que l'on pull les nouvelles modifications du code principal vers notre branche. Ces deux opérations peuvent se faire sous supervision, pour s'assurer que l'on n'interagit pas avec différentes parties de notre code, ce qui risquerait de fausser nos résultats. \blog{https://damage-functions-modeling.readthedocs.io/en/latest/3_notebooks/spatial_temporal_equity.html} \\

Dans cette branche, on modifie le code pour qu'il représente les fonctions de dommage. On modélise ainsi plusieurs formes de fonctions de dommage, issues de la revue de littérature évoquée plus haut. On fait tourner le modèle avec ces différentes fonctions de dommage, et en faisant varier aléatoirement les différents paramètres : taux d'actualisation, les différents paramètres qui entrent en compte dans le modèle. \\

On récupère les résultats et on procéde à des analyses économétriques sur les résultats pour voir quels sont les paramètres qui ont le plus influencé les résultats. 

\end{methodbox}

\begin{figure}
    \centering
    \includegraphics[width=\textwidth]{illustrations/rect1.png}
    \legende{Méthodologie de la comparaison des modèles. (SVG)}{On utilise le modèle open-source WILIAM. Celui ci est composé de différents modules, dont un module représentant les dommages (représenté ici en rouge). On identifié les fonctions de dommage d'autres modèles (DICE, FUND et DEFINE). On les isole, et on les implémente dans le modèle WILIAM. On a alors un nouveau modèle, qui contient simultanément plusieurs manières de représenter les dommages. On connait la distribution de certains paramètres, et on cherche à voir l'effet de cette variation sur les résultats du modèle. On réalise des scénarios, qui sont des combinaisons de tirage aléatoire des variables. Le modèle est executé avec chaque scénario, de nombreuses fois. Tous ces \textit{runs} forment une grande base de donnée, sur laquelle on va réaliser une étude économétrique.}
    \label{fig:methodo-simu}
\end{figure}

\subsection{Limites et contraintes}

Plusieurs limitations doivent être soulignées. 

\paragraph{Transposition dans un autre modèle} D'abord, on utilise les fonctions de dommage de différents modèles dans le modèle WILIAM. Cela permet d'avoir les mêmes conditions pour toutes les fonctions de dommage. Ceci étant dit, les fonctions ne sont plus dans leur milieu d'origine. Ceci pose deux problèmes. D'abord, les fonctions peuvent avoir été calibrées pour fonctionner correctement dans certaines conditions de fonctionnement de température, et pas au-delà; et ces conditions peuvent ne pas être atteintes dans WILIAM. Cela est notamment vrai pour toutes les variables qui nécessitent une calibration, par exemple celles où la population entre en jeux. Une population différente peut aboutir à une fonction de dommage qui est différente. Ensuite, il y a un grand risque d'erreur de transcription. En effet, les variables peuvent avoir les mêmes noms mais ne pas désigner la même chose précisement ou ne pas être calculée de la même manière; ou encore, la documentation d'origine peut ne pas être claire sur les caractéristiques précises d'une variable (notamment, le niveau de désagrégation). On peut néanmoins considérer que ce n'est pas une limitation très forte. En effet, nous nous intéressons ici à la \emph{relation} entre les hypothèses et le niveau de dommage, et pas au niveau de dommage \emph{en soi}. Ainsi, une mauvaise calibration risque de donner des valeurs trop faibles ou trop élevées, mais n'a pas d'effet sur la relation en soi. Par ailleurs, ces fonctions sont justement présentées comme des relations liant les phénomènes climatiques à un niveau de dommage. Si changer de modèle affecte grandement la validité de la relation, alors on peut interroger la validité de la relation dans son ensemble. 

\paragraph{Colinéarité} Le problème majeur de cette technique est que les données analysées ont été produites dans la perspective de cette analyse. Ainsi, il est possible que le résultat de l'analyse  reflète que les hypothèses et les variations qui ont été volontairement ajoutées au modèle. En particulier, le niveau de dommage dépend beaucoup de la valeur du coefficient d'éthique spatial, qui est lui-même construit pour l'expérience. 



\section{Résultats}

\subsection{Par région}

\begin{table}
    \centering
     \resizebox{\textwidth}{!}{\begin{table}[!htbp] \centering
\begin{tabular}{@{\extracolsep{5pt}}lc}
\\[-1.8ex]\hline
\hline \\[-1.8ex]
& \multicolumn{1}{c}{\textit{Dependent variable: total_damage}} \
\cr \cline{2-2}
\\[-1.8ex] & (1) \\
\hline \\[-1.8ex]
 coef & 98983.902$^{***}$ \\
& (7328.425) \\
 const & -14886131.099$^{***}$ \\
& (466565.386) \\
 equation_dice_tot_eq_dice_total_impact & -1341822.952$^{***}$ \\
& (244920.818) \\
 equation_witness_tot_eq_witness_total_impact & 608260.919$^{**}$ \\
& (244920.818) \\
 total_radiative_forcing & 5345723.126$^{***}$ \\
& (131309.687) \\
\hline \\[-1.8ex]
 Observations & 2520 \\
 $R^2$ & 0.443 \\
 Adjusted $R^2$ & 0.443 \\
 Residual Std. Error & 5019383.135 (df=2515) \\
 F Statistic & 500.959$^{***}$ (df=4; 2515) \\
\hline
\hline \\[-1.8ex]
\textit{Note:} & \multicolumn{1}{r}{$^{*}$p$<$0.1; $^{**}$p$<$0.05; $^{***}$p$<$0.01} \\
\end{tabular}
\end{table}}
     \legende{Régressions par région}{Résultats des régressions du logarithme du niveau total de dommages, zone par zone, en fonction du changement de température, du pondérateur géographique, et de la forme de la fonction de dommage. Chaque colonne correspond à une analyse, chaque ligne à une variable. La variable expliquée est le log du niveau de dommage de chaque pays. Clé de lecture : le fait d'avoir une fonction de dommage de la forme DICE diminue le niveau de log\_dommage de 19.6\% par rapport à la fonction de DEFINE, toute autre chose étant égale par ailleurs.}
     \label{tab:reg_region}

\end{table}

Les résultats du premier jeu de régression linéaires sont présentées dans la table \ref{tab:reg_region}. Cinq variables y sont présentées : le niveau de changement de température par rapport à la période préindustrielle (en degré Celsius); la valeur du pondérateur géographique, passée au log; la présence ou non de dummy régionales, qui permettent de contrôler par zone; et enfin des dummy qui indique si la fonction de dommage est de la forme DICE ou WITNESS. Ces deux dernières valeurs sont à comparer \enquote{hello} à la valeur 'par défaut', qui est celle de DEFINE. \\






Tous les coefficients sont significatifs au seuil de 0.01. Ceci reflète le fait que les variables sont construites, et donc expliquées grandement par les variables explicatives prises en compte. 

\blog{https://damage-functions-modeling.readthedocs.io/en/latest/3_modelling/regressions.html}

La première régression est selon le changement de température uniquement. Elle sert d'élément de comparaison, pour s'assurer qu'ajouter les autres variables explicatives à l'analyse permet en effet d'augmenter le pouvoir explicatif du modèle. Le $R^2$ est en effet faible, à 0.315, ce qui explique que le niveau de changement de température explique 31\% des variations du niveau de dommage, passé au logarithme. Le coefficient est de 2.8, ce qui veut dire qu'une augmentation de température de 1 \textdegree de la température résulte en une augmentation du log de dommage de 2.8, toute autre chose étant égale par ailleurs. C'est très important, puisque pour revenir au niveau de dommage pur il faut passer par l'exponentielle : $e^{2.8}=16.44$. Une augmentation de 1 \textdegree de la température est liée à une augmentation de 1544\% du niveau de dommage absolu. \\



En ajoutant le pondérateur géographique et la forme des fonctions de dommage dans le modèle, on augmente le pouvoir explicatif pour obtenir un $R^2$ de 0.49. C'est plus, mais moins que ce que l'on attendrait. \\

À l'inverse, si on a la température, le pondérateur géographique et les dummy régionales, on obtient un $R^2$ de 0.95, ce qui est beaucoup plus important. On obtient une valeur similaire pour la relation log-dommage / température. Cependant, l'écart-type des valeurs est plus petit, signifiant que ce coefficient est plus fiable. On a une relation d'environ 1 pour la relation log-damage / log-coef. C'est tout à fait cohérent avec la manière dont a été construite la variable $damage$. Cela veut dire qu'une augmentation de 1\% de la valeur de ce coefficient résulte en une augmentation de 1\% du niveau absolu de dommage. \\


Enfin, en prenant en compte toutes les variables explicatives, on obtient un $R^2$ de 0.96. Cela veut dire que 96\% des variations de log\_damage sont expliquées par les variables explicatives prise en compte dans le modèle. On peut ainsi dire qu'il n'y a pas ou peu d'autres facteurs qui expliquerait des variations du niveau de dommage. Par ailleurs, le fait qu'ajouter des dummy régionales augmente tant le pouvoir explicatif du modèle laisse penser que parmi les facteurs influençant le niveau de dommage, certains, même non captés explicitement par notre modèle, sont étroitement liés à la région. Ils peuvent être de différents ordres (par exemple, le PIB). Les coefficients pour l'augmentation de température et pour le pondérateur géographique sont similaires aux autres modèles. On obtient également des valeurs pour les formes des fonctions de dommage. Ainsi, avoir une forme de DICE diminue le niveau de dommage absolu de 17.8\% ($1 - e^{-0.196}= 1 - 0.822 = 0.178$) par rapport à une fonction de dommage de la forme DEFINE. De la même manière, une fonction de dommage de la forme WITNESS correspond à une augmentation du niveau absolu de dommage de 42\% ($ e^{0.35} = 1.42$). \\

\subsection{En agrégeant au niveau mondial}

\begin{table}
    \centering
    \resizebox{\textwidth}{!}{\begin{table}[!htbp] \centering
\begin{tabular}{@{\extracolsep{5pt}}lcccc}
\\[-1.8ex]\hline
\hline \\[-1.8ex]
& \multicolumn{4}{c}{\textit{Dependent variable: log\_damage}} \
\cr \cline{2-5}
\\[-1.8ex] & \multicolumn{1}{c}{Temp} & \multicolumn{1}{c}{Simple} & \multicolumn{1}{c}{Equation} & \multicolumn{1}{c}{Logged}  \\
\\[-1.8ex] & (1) & (2) & (3) & (4) \\
\hline \\[-1.8ex]
 Temperature Change & 4.487$^{***}$ & 4.627$^{***}$ & 4.627$^{***}$ & 4.756$^{***}$ \\
& (0.166) & (0.108) & (0.097) & (0.102) \\[2em]
 Exponent & & 0.895$^{***}$ & 0.895$^{***}$ & \\
& & (0.110) & (0.098) & \\[2em]
 Constant & & -0.000$^{***}$ & -0.000$^{***}$ & \\
& & (0.000) & (0.000) & \\[2em]
 log\_exponent & & & & 0.409$^{***}$ \\
& & & & (0.055) \\[2em]
 log\_constant & & & & -1.064$^{***}$ \\
& & & & (0.111) \\[2em]
 DICE form damage function & & & -0.237$^{**}$ & -0.237$^{*}$ \\
& & & (0.116) & (0.123) \\[2em]
 WITNESS form damage function & & & 0.308$^{***}$ & 0.308$^{**}$ \\
& & & (0.116) & (0.123) \\[2em]
\hline \\[-1.8ex]
 Observations & 81 & 81 & 81 & 81 \\
 $R^2$ & 0.903 & 0.962 & 0.971 & 0.967 \\
 Adjusted $R^2$ & 0.901 & 0.960 & 0.969 & 0.965 \\
 Residual Std. Error & 0.757 (df=79) & 0.479 (df=77) & 0.426 (df=75) & 0.451 (df=75) \\
 F Statistic & 731.509$^{***}$ (df=1; 79) & 648.547$^{***}$ (df=3; 77) & 495.566$^{***}$ (df=5; 75) & 441.062$^{***}$ (df=5; 75) \\
\hline
\hline \\[-1.8ex]
\textit{Note:} & \multicolumn{4}{r}{$^{*}$p$<$0.1; $^{**}$p$<$0.05; $^{***}$p$<$0.01} \\
\end{tabular}
\end{table}}
    \legende{Résultats des régressions aggrégées par pays}{La variable expliqué est le logarithme du niveau total de dommage, par année. Clé de lecture : augmenter de 1\% la valeur de la constante de normalisation correspond à une diminution de 68\% du niveau de dommage. }
    \label{tab:reg_glob}
\end{table}

La table \ref{tab:reg_glob} présente les résultats d'un deuxième jeu de régressions. Ici, nous avons regroupé les résultats par année et par run de simulation. En d'autres termes, ce sont les mêmes résultats que la table \ref{tab:reg_region}, mais les valeurs sont la somme de tous les dommages mondiaux au niveau global, et plus par région. Cette approche permet de décomposer le pondérateur géographique en deux : l'exposant et la constante de normalisation. Par ailleurs, cette échelle permet d'être plus proche des conditions dans lesquelles sont calculés des valeurs agrégées des impacts du changement climatique, telles que le Coût Social du Carbone. Les résultats des trois modèles ont des $R^2$ importants ($>0.8$), ce qui reflète le fait que les différences régionales disparaissent dans l'agrégation au niveau mondial. \\

On obtient des résultats similaires entre les trois modèles, avec des $R^2$ qui augmentent légèrement à mesure qu'on ajoute des variables explicatives. Nous nous concentrons ici sur les résultats du modèle 3. La relation entre l'augmentation de température et le niveau de dommage est plus importante encore que dans les régressions précédentes, puisqu'elle est de 4.11. On obtient des valeurs très similaires pour l'effet des fonctions de dommage. Une variation de l'exposant du pondérateur géographique de 1\% explique une variation de 18,8\% du niveau total de dommage. De la même manière, une augmentation de la constante de normalisation de 1\% correspond à une diminution de 68\% du niveau de dommage total. 


\section{Discussion}







\begin{figure}
    \centering
    \includegraphics[width=\textwidth]{results/cartogramme_3.png}
    \legende{Carte en anamorphose selon le niveau de dommage}{L'espace est déformé pour que la surface de chaque zone soit proportionnelle à la valeur absolue des dommages subis. La carte du centre représente l'ensemble de l'échantillon. La carte du haut représente uniquement les runs où il y a un coefficient > 1 et une constante > 20000 (scénario où l'on exacerbe les inégalités). La carte du bas représente uniquement les runs où il y a un coefficient < 1 et une constante < 15000 (scénario ou on pénalise fortement les dommages sur les plus faibles revenus). Ces cartes ont une valeur avant tout heuristique : c'est pour cette raison qu'il n'y a pas d'échelle. L'objectif est avant tout de visualiser l'effet qu'ont les distributions de nos paramètres sur la distribution des dommages comptabilisés.  La disparition de l'UE27 n'est pas volontaire et ne doit pas être interprétée. }
    \label{fig:anamorphose}
\end{figure}

%\begin{figure}
   % \centering
    %\includegraphics[width=0.5\linewidth]{}
    %\legende{Les dommages selon la manière de comptabiliser le revenu par habitants}{}
    %\label{fig:carte1}
%\end{figure}


%\begin{figure}
   % \centering
    %\includegraphics{}
    %\legende{Coûts des dommages selon la fonction de dommage incluse dans le modèle WILIAM.}{Graphiques représentant la valeur totale des dommages dans le modèle WILIAM selon la fonction de dommage. Les couleurs représentant le modèle d'origine de la fonction de dommage; la ligne pleine la médiane, la zone grisée les valeurs interquartiles.}
    %\label{fig:simu}
%\end{figure}



\subsection{Interprétation des résultats}

%Que signifient vos résultats par rapport à vos hypothèses ? Quelles conclusions pouvez-vous en tirer ? 

Les résultats par région (voir table \ref{tab:reg_region}) permettent de mettre en évidence le rôle que joue chacun des trois types de variables explicatives détaillées plus haut : physique, méthodologique et éthique. 
Sans surprise, les variables climatiques jouent un rôle majeur dans la détermination du niveau de dommage. Ici, elles sont incarnées par l'augmentation de la température moyenne par rapport au niveau préindustriel. Ce résultat était attendu, et est cohérent avec les hypothèses. En effet, l'augmentation de la température est un des canaux principaux des effets du changement climatique. Par ailleurs, la plupart des fonctions de dommage sont des fonctions de la température. 
Les variables méthodologiques, ici représentées par le choix de la fonction de dommage, jouent également un rôle important. Ainsi, une fonction type DICE (voir \ref{eq:df_dice2023}) produira, toute autre chose étant égale par ailleurs, des dommages 17\% plus faible que DEFINE (voir \ref{eq:DEFINE}), tandis qu'une équation type de WITNESS (voir \ref{eq:WITNESS}), qui prend en compte un niveau plus élevé de dommage suite au passage d'un tipping point, produira des dommages 42\% plus élevés. 
En guise de variable éthique, nous avons choisi de travailler avec le pondérateur géographique. Les résultats ici ne prêtent pas spécifiquement à interprétation, puisqu'ils sont principalement le reflet de la manière dont $log\_damage$ a été construite. \\

Les résultats au niveau global (voir \ref{tab:reg_glob}) permettent d'interpréter plus précisement le rôle de chacun des composants de ce pondérateur géographique. Une variation de l'exposant du pondérateur géographique de 1\% explique une variation de 18,8\% du niveau total de dommage. C'est important, puisque celui-ci est défini selon une loi uniforme normale de moyenne 0 et d'écart-type 1.

De la même manière, l'effet de la constante de normalisation du niveau de revenu de 1\% correspond à une diminution de 68\% du niveau de dommage
total. Cette constante est calibrée selon une loi uniforme sur l'intervalle 10000-50000. La variation de cette constante peut s'interpréter comme une variation de ce qui est considéré comme un niveau de revenu moyen. En effet, le pondérateur géographique va favoriser les pays dont le revenu par habitant est inférieur à cette valeur en augmentant la manière dont les dommages sont comptabilisés, tout en atténuant le poids des dommages des pays plus riches. Ainsi, choisir une valeur plus grande de cette constante de normalisation tend à gommer les inégalités entre les pays. 

\begin{figure}[htbp]
    \centering
    \begin{minipage}{0.5\textwidth}
        \centering
        \includegraphics[width=\linewidth]{results/exponent.png} % Remplacez par le chemin de votre image
        \subcaption{Valeur du pondérateur selon l'exposant}
        \label{fig:coef}
    \end{minipage}%
    \hfill
    \begin{minipage}{0.5\textwidth}
        \centering
        \includegraphics[width=\linewidth]{results/norm_constant.png} % Remplacez par le chemin de votre image
        \subcaption{Valeur du pondérateur selon la constante}
        \label{fig:cons}
    \end{minipage}%
    \legende{Valeur du pondérateur selon la valeur de l'exposant et de la constante de normalisation}{L'axe des ordonnées représente la valeur du pondérateur géographique. Ces deux figures permettent de se représenter l'effet des variations des deux paramètres (exposant à gauche, constante de normalisation à droite) sur la valeur du pondérateur géographique. A gauche, on fixe une constante de normalisation à 25000, et faison varier l'exposant suivant une loi normale (la valeur de l'exposant est représentée par la couleur : plus c'est violet, plus la valeur est élevée). A droite, on fixe l'exposant à 1, et on fait varier la constante de normalisation en suivant une loi uniforme sur $[10000, 50000]$. En pratique, les deux sont affectés aléatoirement à chaque \emph{run}. Clé de lecture : chaque ligne correspond à un fonction permettant d'avoir le pondérateur géographique. Elle correspond à un tirage particulier du couple $(\text{exposant}, \text{constante})$. Afin d'obtenir le pondérateur géographique d'une région, il faut choisir une courbe, lire le revenu par habitant en abscisse, puis lire la valeur du pondérateur géographique sur l'axe des ordonnées. Pour avoir celle d'une autre région dans les même conditions, il faut renouveller l'opération en utilisant la même courbe. }
    \label{fig:ponderateur_selon}
\end{figure}


\subsection{Comparaison avec la littérature}

Ces résultats sont en cohérence avec des acquis de la littérature. 

D'abord, on retrouve une relation très importante entre le changement de température et le niveau de dommage. Malgré l'introduction du pondérateur géographique dans le modèle, cette relation reste très significative et d'une amplitude très importante. \\

Nous retrouvons également que le choix du modèle influence grandement le résultat. C'est cohérent avec la littérature, qui montre qu'une partie importante des variations des résultats des modèles provient du choix du modèle lui-même. On peut par exemple citer \textcite{gillingham_modeling_2015, gillingham_modeling_2018}, qui montrent cette sensibilité. 

Enfin, nous montrons que les variables éthiques jouent un grand rôle dans le niveau de dommage comptabilisé.

La valeur du taux d'actualisation est une discussion classique de ce sujet. Nous complétons ici la question de l'équité temporelle par l'équité spatiale. 

On retrouve des résultats allant dans le sens de \textcite{dennig_inequality_2015}, qui montrent qu'une prise en compte des effets différenciés des impacts du changement climatique selon le niveau de revenu permet d'aboutir à un coût social du carbone très différent. Nous retrouvons ainsi des résultats similaires pour les variations spatiales et interrégionales. 


\subsection{Implications pratiques}

Ces résultats montrent le rôle crucial qu'ont les hypothèses éthiques dans la détermination du niveau de dommage. Elles appellent à la plus grande prudence s'agissant de déterminer un niveau \emph{optimal} de dommages. Il serait en effet très sensible à la distribution de ceux-ci et à la pondération des dommages selon les régions. Il semble donc que l'utilisation de telles fonctions de dommages nécessite d'une part une discussion permettant d'établir de manière juste (avec toute la difficulté qu'un tel terme suppose) une comptabilité des dommages; d'autre part, une transparence importante de ces hypothèses. Loin d'être un enjeu technique réservé aux seuls spécialistes de la modélisation, ces questions sont des questions éminemment politiques. Nous avons montré ici qu'elles jouent un rôle déterminant dans l'estimation du niveau de dommage. 


\subsection{Perspectives futures}
% Quelles recherches supplémentaires pourraient être nécessaires pour approfondir vos découvertes ou explorer des questions connexes ?

De nombreuses pistes pour de la recherche ultérieure peuvent être citées. \\

D'abord, solidifier les outils. Comme nous l'avons évoqué ou l'évoquerons à de multiples reprises, les modèles sont des outils difficiles à prendre en main, avec des fonctionnements et des documentations protéiformes. Il est donc évident que de nombreuses erreurs de transcription, paramétrisation ou compréhension se soient glissées dans le code. Une première étape serait donc de chercher à augmenter la fiabilité du code source utilisé pour l'expérience. Par exemple, l'implémentation des fonctions de dommage de FUND n'ont pas pu aboutir, car elles donnaient des résultats trop imprécis. On pourrait par exemple contribuer au package PySD, pour permettre d'automatiser la création de documentation à partir du modèle. Un tel outil permettrait d'écrire automatiquement les fonctions de WILIAM selon leurs inputs et leurs outputs, et d'écrire directement de la documentation lisible par des humains. Par exemple, elle permettrait d'écrire des équations \LaTeX à partir du code source, ce qui permettrait de s'assurer que le code source et la documentation sont en adéquation. On pourrait pour cela s'inspirer du fonctionnement de ReadTheDocs. \\

Nous avons cité des études s'intéressant à l'équité générationnelle, à l'équité sociale, et nous nous sommes penchés sur l'équité spatiale. Il pourrait être intéressant d'implémenter simultanément ces différentes approches, pour évaluer leur influence respective sur les résultats du modèle. \\

De plus, nous n'avons implémenté que trois fonctions de dommage. Nous pourrions continuer à implémenter d'autres fonctions, et notamment des fonctions ne prenant pas en compte l'aspect monétaire, mais par exemple prenant en compte l'aspect énergétique ou matériel des dommages. \\

Enfin, il serait intéressant de développer un outil permettant à un utilisateur non expérimenté de varier ses paramètres, afin d'apporter ces questions dans un débat qui dépasse la communauté de la modélisation. 
\chapter{Éthique de la modélisation des dommages}
\label{chapter:ethique}
\newrefsegment

\PEEL{Les modélisateur.ices ont une responsabilité vis à vis de ce que leurs modèles produisent. }{Fournissez des preuves, des données ou des citations qui soutiennent votre point.}{Expliquez en quoi les preuves que vous avez fournies sont pertinentes et comment elles appuient votre point.}{Faites le lien avec le sujet principal ou avec la section suivante de votre mémoire.}


\chapterabstract{Les choix de modélisation des fonctions de dommage sont lourds de conséquences sur le message que véhiculent les résultats de ces modèles. Ces résultats alimentent le débat public sur les questions climatiques. Vu les enjeux sociaux et sociétaux qui sont à l'oeuvre, la pratique de la modélisation implique des questions éthiques importantes. Quelles sont-elles, et comment sont-elles prises en compte ? Cette partie est plus conceptuelles et épistémiques, et cherche à identifer des points d'attention de la modélisation.}


Les modèles intégrés sont dans une position ambivalente : outils techniques complexes et abstraits, ils sont pourtant utilisés par des utilisateurs variés pour produire des décisions très concrètes. Cette tension est particulièrement claire dans le cas des fonctions de dommage. Comme nous avons pu le voir dans la partie précédente, les résultats qu'elles produisent sont très sensibles aux différentes hypothèses normatives sur lesquelles elles sont construites. Ainsi, la modélisation apparait empreinte de questions éthiques essentielles. 

Dans cette partie, nous reviendrons sur plusieurs points de tension, que nous éclairerons à l'aide de textes épistémologiques. 

Notre question de recherche principale sera la suivante : \textit{comment les systèmes normatifs des chercheurs transparaissent-ils dans les modèles ?}

Nous suivrons trois objectifs. D'abord, présenter des outils épistémologiques qui permettent d'éclairer ces questions. Ensuite, de présenter des points particuliers des modèles qui nous semblent porteurs d'enjeux éthiques, d'une manière qui se veut le plus proche possible du fonctionnement concret du modèle. Enfin, de croiser les deux approches pour proposer une lecture pratique de l'éthique de la modélisation des dommages. 

Dans un premier temps, nous présenterons le cadre conceptuel de Tuana, qui permet de décomposer en trois sphères les enjeux éthiques liés à la recherche. Nous discuterons ensuite de l'idée selon laquelle la modélisation des dommages impose une grille de lecture particulière, socialement, épistémologiquement et historiquement située, et cadre ainsi le débat sur l'action climatique. Enfin, nous aborderons la question de la responsabilité des chercheurs quand au produits de leur recherche, notamment à travers le concept de doute normativement inapproprié. 








\begin{displayquote}
    Dépolitiser le réel c'est le repolitiser au profit de l'oppresseur
\end{displayquote}


\begin{table}
    \centering
    \begin{tabular}{|c|c|c|c|} \hline 
         Pratique&  Enjeu&  Concept associé& Commentaire\\ \hline 
         Choisir les modèles pour faire tourner les scénarios&  Framing / possibility space&  & \\ \hline 
         Attribuer une valeur à un paramètre (taux d'actualisation)&  Equité intergénérationnelle&  & \\ \hline 
         Fabrique du doute&  Responsabilité dans l'interprétation des résultats&  Doute normativement inappropriés& \\ \hline
    \end{tabular}
    \legende{Enjeux éthiques dans les modèles et cadre d'analyse associés}{}
    \label{tab:ethique}
\end{table}


\cite{schienke_intrinsic_2011} => sur l'éthique intrinsèque dans les IAMs
\cite{weitzman_modeling_2009} => sur les événements non-linéaires



\section{Les trois niveaux de l'éthique et la responsabilité du modélisateur}

Cette section pose plusieurs questions quant à la place de la technique et de la modélisation dans la cité. 

Elle s'appuie notamment sur \cite{jonas_principe_2008} et \cite{vast machine}, ainsi que sur la classification des enjeux éthiques de \cite{tuana_leading_2010}.


Comme nous l'avons vu précédemment, le processus de modélisation est avant tout un processus de sélection de ce qui est représenté et ce qui ne l'est pas, et de décision de la manière de le représenter. Comme nous le verrons dans le chapitre \ref{chapter:socio}, l'imperfection et la partialité des modèles est souvent assumée, voire revendiquée par les équipes de modélisation. Elle s'accompagne souvent d'une tentative de dé-responsabilisation de l'équipe de modélisation. Celle-ci prend généralement la forme suivante : 

\begin{displayquote}
    Les résultats des modèles sont valables uniquement sous les conditions (nombreuses et irréalistes) dans lesquelles il a été conçu. On ne peut donc pas extrapoler les résultats, ou faire dire au modèle autre chose que ce qu'il veut dire. 
\end{displayquote}
Pourtant, les résultats des modèles sont utilisés de manière large, notamment pour prendre des décisions de politique publique, y compris par des personnes n'ayant pas de compétence spécifique en modélisation, ni de connaissance particulière des modèles utilisés pour produire les connaissances qu'elles utilisent. Il y a donc un paradoxe : d'une part, l'idéal d'une compréhension fine des hypothèses de la modélisation, qui permet d'être conscient des choix de modélisation et de leurs implications. Cet idéal protège la responsabilité du modélisateur : en effet, les hypothèses ayant été clairement formulées, l'utilisateur final devient responsable de sa propre interprétation. Celle-ci se fait en accord avec les hypothèses énoncées, que l'utilisateur final fait sienne. D'autre part, il est difficile de comprendre un modèle pour plusieurs raisons : technique (les logiciels ne sont pas/plus accessibles), légales (le code source n'est pas accessible librement), capacitaires (les codes sources sont difficiles à interpréter, surtout si on n'est pas familier du langage utilisé), théoriques (les modèles intégrés font appel à des concepts issus de nombreuses disciplines). Cette difficulté à comprendre les modèles font que les résultats sont \textit{de facto} souvent interprétés dans l'état dans lequel ils sont délivrés, sans la contextualisation permise par le code. Dès lors, le choix des hypothèses est invisibilisé, et la responsabilité du modélisateur peut s'étendre jusqu'à l'interprétation du modèle, voire jusqu'aux conséquences de cette interprétation. \\

Pour éclairer ce paradoxe, nous allons nous intéresser aux dimensions éthiques de la modélisation. Pour cela, nous allons nous appuyer sur les travaux de Tuana, qui a cherché à développer un cadre conceptuel de l'éthique dans la recherche, et qu'elle a adapté spécifiquement au cas de la modélisation intégrée. \\

Elle développe trois dimensions éthiques de la recherche scientifique : l'éthique procédurale, l'éthique intrinsèque et l'éthique extrinsèque. 

\begin{figure}
    \centering

    \includegraphics[width=0.6\linewidth]{venn_ethics.png}
    \legende{Les trois dimensions de l'éthique dans la recherche.}{\cite{tuana_climate_2019} décrit trois dimensions éthiques dans la recherche : l'éthique procédurale consiste en le respect des normes établies et reconnues dans la communauté. C'est en général ce à quoi on fait référence quand on parle de \textit{bonne science}. L'éthique extrinsèque désigne les questions éthiques qui sont liées aux utilisations des productions scientifiques. Enfin, l'éthique intrinsèque désigne les questions qui sont incluses dans le mode de production de la connaissance (en l'occurence, le modèle). }
    \label{fig:diag-venn}
\end{figure}


\subsection{L'éthique procédurale}

L'\Gls{procedural ethics} désigne ce qui est couramment regroupé sous le terme de \textit{bonne science}, ou \textit{good science}. Il s'agit de produire de la connaissance en respectant les attentes de la communauté scientifique en termes d'honneteté, de sincérité et de rigueur. Tuana \cite{tuana_leading_2010}  la définit comme ceci : 

\begin{displayquote}
Ethical aspects of the process of conducting scientific research, such as: falsification, fabrication, and plagiarism; care for subjects (human and non-human animal); responsible authorship issues; analysis of and care for data.
\end{displayquote}
La plupart des travaux de modélisation s'inscrivent pleinement dans cette démarche, et sont en phase avec les attentes et bonnes pratiques de la communauté : 

\begin{itemize}
    \item tranparence : les codes sources sont ouverts et accessible, il y a une documentation plus ou moins complète
    \item honneteté : les hypothèses sont clairement énoncées
    \item sincérité des résultats : les résultats sont reproductibles facilement
\end{itemize}

Pourtant, de nombreuses critiques viennent compromettre ce constat. Par exemple, le GIEC remarque qu'il faudrait des explications plus détaillées sur le fonctionnement interne des modèles. 


\begin{displayquote}
    Coucou c'est moi
\end{displayquote}
\begin{displayquote}
Transparency is needed to reveal conditionality of results on specific choices in terms of assumptions (e.g., discount rates) and model architecture. More detailed explanations of underlying model dynamics would be critical to increase the understanding of what drives results (Bistline et al. 2020; Butnar et al. 2020; Robertson 2020). » ([“Mitigation Pathways Compatible with Long-term Goals”, 2023, p. 304]
\end{displayquote}
Un bon exemple est la documentation de FUND. Celle-ci est disponible en ligne, sur le site \href{http://www.fund-model.org/}{fund-model.org/}. Elle est détaillée; toutes les équations et leurs variables y sont décrites; les paramètres en entrée sont également disponibles. De plus, le code source complet, dans un langage open source (julia) est disponible sur github. Il y a donc un véritable effort de transparence et d'ouverture. 

% Pourtant, et comme le remarquent les auteurs de ce modèle, \emph{les modèles sont souvent inutiles dans des mains inexpérimentées}.
Malgré ces efforts importants, il reste difficile de 1/ observer le modèle et en interpréter la conception 2/ le faire tourner pour le réimplémenter.  Par exemple, les paramètres changent de nom, ou ne désignent plus les mêmes choses sans que cela soit indiqué.

Le paramètre $T$, par exemple, désigne successivement \emph{ global mean temperature (in degree centigrade)} et \emph{global mean temperature above pre-industrial (in degree Celsius) at time t}. Il est plus cohérent avec les equations de considérer qu'il s'agit de la même variable; pourtant, les noms portent à confusion : dans un cas il s'agit d'une valeur mesuré, dans l'autre de l'écart de cette valeur avec la période pré-industrielle. 

Les tables de paramètres sont disponibles aussi, ce qui est beaucoup. Néanmoins, elles sont disponibles sous deux formes peu exploitables. La première est sous une forme de distribution de ces paramètres. Celle-ci a l'avantage de bien comprendre comment ces paramètres sont produits. Néanmoins, sans graine aléatoire, on ne peut pas reproduire à l'identique le fonctionnement du modèle, et ainsi s'assurer que les effets observés ne sont pas simplement liés à un tirage différent. De plus, les distributions sont données en fragments de code Julia, ce qui limite les exports dans d'autres formats. L'autre forme est en texte sur le site. Cette forme présente l'avantage d'être très facilement lisible par des humains. Néanmoins, il est très difficile d'extraire ces données dans un format compatible avec les outils de modélisation, tels que Python ou VENSIM. Dans notre cas, il a donc fallu réfléchir à un programme permettant de convertir le Markdown en format de données CSV, ce qui est à la fois compliqué et source d'erreurs potentielles. \\

Cet exemple nous offre quelques enseignements. D'abord, la tranpsarence est très difficile. Les modèles sont complexes, avec souvent de nombreuses variables et équations, d'autant plus de paramètres et de sources. Ouvrir de manière convenable un modèle est une action couteuse, qui recquière un investissement en temps et en ressource important, alors même que peu de gens sont demandeurs de l'accès aux modèles. Ensuite, qu'elle n'est pas suffisante : avoir accès aux paramètres ne suffit pas à en comprendre les effets sur le modèle. 

Ainsi, le respect des normes scientifiques en termes d'ouverture, de rigueur et de transparence est souvent mis en avant. Dans le cadre de la partie expérimentale de notre travail, nous avons pu constater que des efforts importants sont consentis par les équipes pour mettre à la disposition du plus grand nombre le plus de ressources possibles.  Il y a donc là une première tension, entre une transparence qui n'est jamais suffisante, mais qui est pourtant demandée, tout en étant en réalité peu utilisées (nous en reparlerons dans le chapitre \ref{chapter:socio}. 

%\begin{tcolorbox}
  %  Mais remarques de Keen sur Nordhaus
%\end{tcolorbox}

\subsection{L'éthique intrinsèque}

L'\Gls{intrinsic ethics} va plus loin dans l'exigence. Il s'agit de la dimension éthique qui est directement intégrée dans le processus de recherche : il ne s'agit plus de la forme de la recherche, mais du fond de la recherche. 

\begin{displayquote}
Ethical issues and values that are embedded in or otherwise internal to the production of scientific research and analysis. These involve ethical issues arising from, for example: the choice of certain equations, constants, and variables; analysis of data; handling of error, and degree of confidence in projections.
\end{displayquote}
Comme le souligne Tuana, la bonne prise en compte de ces questionnements nécessite d'avoir un accompagnement épistémologique au sein des équipes de recherche, c'est-à-dire de prendre en compte ces enjeux comme partie intégrante du développement du projet : \emph{the domain of intrinsic ethics will not be fully successful unless it includes the expertise of philosophers of science. It is only when this domain becomes a focus of our field, that the range of relevant issues and their ethical and epistemic significance will be fully appreciated.} 

\begin{tcolorbox}
    A développer : 
    \begin{itemize}
        \item choix du taux d'actualisation
        \item 
    \end{itemize}
\end{tcolorbox}

=> exemple 1 : choix du taux d'actualisation (faire une référence à la querelle Nordhaus / Stern)
=> explication 1 : le taux d'actualisation fait intervenir des conceptions éthiques importantes

\begin{displayquote}
    Modelling over many decades, regions and possible outcomes demands that we make distributional and [ethical]([[Ethique]]) judgements systematically and explicitly. Attaching little weight to the future, simply because it is in the future (‘[pure time discounting]([[discount rate]])’), would produce low estimates of cost – but if you care little for the future you will not wish to take action on climate change. » ([“Economic modelling of climate change impacts”, 2007, p. 143](zotero://select/library/items/L2NNDHEN))
\end{displayquote}

Un exemple d'éthique intrinsèque est le découpage par région. Les modèles se focalisent sur certaines régions plus que d'autres, sont calibrés sur certaines régions (où il y a par ailleurs des données plus nombreuses et de meilleure qualité), et sont donc bien plus pertinents pour décrire les phénomènes dans certaines régions plus que dans d'autres. 

\begin{figure}[htbp]
    \centering
    \begin{minipage}{0.45\textwidth}
        \centering
        \includegraphics[width=\linewidth]{figures/FUND_regions.png} % Remplacez par le chemin de votre image
        \subcaption{FUND}
        \label{fig:carte1}
    \end{minipage}%
    \hfill
    \begin{minipage}{0.45\textwidth}
        \centering
        \includegraphics[width=\linewidth]{figures/WILIAM_regions.png} % Remplacez par le chemin de votre image
        \subcaption{WILIAM}
        \label{fig:carte2}
    \end{minipage}%
    \legende{Granularité spatiale différente entre les modèles}{}
    \label{fig:trois_cartes}
\end{figure} 

=> exemple 2 : la granularité spatiale
=> révèle une cadrage / centrage du modélisateur 

\subsection{L'éthique extrinsèque}

Enfin, l'\Gls{extrinsic ethics} désigne les questionnements autour des effets de la production scientifique sur la société. C'est une sphère beaucoup plus large, puisqu'elle sort du domaine du laboratoire et de la communauté scientifique, pour mesurer les effets sur les sociétés. 

\begin{displayquote}
Ethical issues that are external to the production of scientific research. These arise, for example, when considering the impact of scientific research on society; e.g., the effects of technological innovations on social ends such as health and well-being, whether pressing social and economic issues are likely to be addressed and if so, who benefits, and the role of science in policy-making. This domain of ethics also includes ethical concerns arising from the impact of society upon science, for example the impact of funding on research trajectories or the ways in which wide-spread societal biases can impact research trajectories, as they arguably did with eugenics research. In the latter case, there are often links between the domains of extrinsic and intrinsic ethics.
\end{displayquote}
Ce questionnement s'inscrit dans une réflexion plus large du rapport entre la production scientifique et son application dans la société. 

\begin{tcolorbox}
\begin{itemize}
    \item 
    \item [[cadrage]], forcément normatif « IAM analysis could focus on only a subset of relevant futures and thus push society in certain directions without sufficient scrutiny » (\href{zotero://select/library/items/2SDDNUUF}{“Annex III: Scenarios and Modelling Methods”, 2023, p. 1862}) (\href{zotero://open-pdf/library/items/CHVFSLLH?page=22&annotation=4MBM5B9Q}{pdf})

\end{itemize}

\end{tcolorbox}


\begin{displayquote}
    « Hence, the modeling community prefers to discuss controversial modeling issues such as flux adjustments among themselves and to ‘reserve criticism for internal dealings within their own peer community’ (Ref 145, p. 30) because acknowledging uncertainty is seen to make the credibility of climate science in the policy process vulnerable to the attacks of climate skeptics. » ([Beck et Krueger, 2016, p. 638
\end{displayquote}

\begin{displayquote}
    Douglas97,99 argues that extrinsic ethics (and other non-epistemic consequences of scientific choices) create a specific responsibility for researchers who work on policy-relevant issues with an uncertain evidence base. » ([Beck et Krueger, 2016, p. 633]
\end{displayquote}
\begin{displayquote}
     By contrast, the extrinsic ethical consequence of the decision may be the implementation of suboptimal policies with possibly significant societal consequences.9 » ([Beck et Krueger, 2016, p. 633]
\end{displayquote}
 Ces considérations sur l'éthique extrinsèque prendront plus de sens dans la section suivante, on l'on développera l'idée que la modélisation participe activement au cadrage du débat public, et donc aux choix de futurs possibles. 

\section{Construire / dessiner les futurs possibles}

Cette section montre que le savoir produit par la science est situé dans le temps et dans l'espace. Ainsi, il n'est plus positif, mais bien normatif, en ceci qu'il décrit un univers des possibles. 

Un exemple de à quel point le framing peut impacter la connaissance est la classification des pays. Dans le SPM de l'AR5, les parties n'ont pas pu s'accorder sur un type de classification des pays à adopter ; finalement toutes les figures et textes associés ont été rejetées par les gouvernements. On pourrait ici penser que présenter une information ou une autre n'a pas d'importance, tant que celle-ci a été produite dans les normes scientifiques en vigueur. Pourtant, cet exemple montre que les gouvernements considèrent que le choix d'une classification porte en soi un message trop important ; reconnaissant alors l'absence de neutralité du contenu scientifique \cite{edenhofer_mapmakers_2014}. 

Les modèles intégrés sont utilisés comme base scientifique pour les négociations climatiques. Ils permettent notamment de décrire l'espace des possibles, c'est-à-dire l'ensemble des chemins qui peuvent être pris par les sociétés. Cette relation entre le modèle et la prise de décision est décrite par une image très parlante dans \cite{edenhofer_mapmakers_2014}, où les modélisateurs sont décrits comme des cartographes, et les décideurs comme des navigateurs. Dès lors, le rôle des modélisateurs-cartographe est de décrire l'espace possible, l'ensemble des zones qui sont navigables ; et, si possible, les conditions de navigation que l'on peut rencontrer dans ces zones.  En regard de ces nouvelles connaissances, les décideurs-navigateurs doivent décider du cap à suivre aujourd'hui selon la zone de navigation voulue pour demain. Cette distinction très nette entre décideurs et modélisateurs n'est pas sans limitations. L'une d'entre elle est le cadrage, c'est à dire la manière dont le débat public est façonné par le cadre qu'on lui donne.  Plusieurs choses influencent ce cadrage. Nous verrons d'abord que le prisme technique et énergétique aggrégé de la plupart des modèles représente ce genre de contraintes, sans questionner les modèles sous-jacents. Nous verrons ensuite que le choix des variables, et notamment la monétarisation des dommages fait que de nombreuses dynamiques ne sont soit pas prises en compte, soit prises en compte d'une manière que l'on peut questionner. Enfin, nous, aborderons l'idée que la connaissance est socialement construite, en se basant notamment sur les écrits de Helène Ongino, pour questionner le caractère universel des résultats des modèles. 

\subsection{Un monde homogène, technique et neutre ?}

Les relations entre modélisateurs et décideurs s'inscrivent dans ce que \cite{aykut_gouverner_nodate} nomment le modèle linéaire de l'expertise. Il s'agit de l'idée selon laquelle \emph{a connaissance précède l’application et le consensus scientifique précède l’action politique}. Le fonctionnement du régime climatique suit en effet ce modèle : les chercheurs publient, sont évalués par le GIEC, qui est interprété par le SBSTA, avant de devenir des politiques climatiques basées sur un consensus scientifique. L'exemple du coût social du carbone va aussi en ce sens : les modèles sont développés, puis évalués par le groupe inter-agence pour le coût social du carbone, avant d'être utilisé dans l'évaluation des politiques publiques par les agences fédérales. 

Il y a dès lors une \emph{séparation radicale entre science et politique : à la science, les faits, les connaissances ; à la politique, les décisions, les valeurs, les croyances}. 






=> il n'y a pas de modèle qui décompose les dommages par groupe sociaux

Helene Longino va plus loin dans sa critique de la science neutre en s'inscrivant dans une perspective d'épistémologie néo-marxiste. Elle met en avant trois points principaux : les conséquences du progrès technique, qui découle de la science, les travers du réductionnisme, et la possibilité d'une science plus émancipatrice. Cette approche remet les enjeux de rapport de force au cœur de l'épistémologie, et permet d'avoir une lecture politique de la production scientifique. \\

D'abord, elle clame que les dérives du progrès technique sont les inévitables conséquences de la production scientifique. 

\begin{displayquote}
    One is that the dystopic applications of modern science—the domination of political life by thermonuclear weapons, new particle beam weapons, and other monsters of annihilation; the control of human potentiality through genetic engineering; the proliferation of toxic wastes from science-based technologies; the displacement of human labor by automation—are not a misuse of socially neutral science but the inevitable result of bourgeois science.
\end{displayquote}

« there are concerns that IAMs are describing transformative change on the level of energy and land use, but are largely silent about the underlying socio-cultural transitions that could imply restructuring of society and institutions » (\href{zotero://select/library/items/2SDDNUUF}{“Annex III: Scenarios and Modelling Methods”, 2023, p. 1862}) (\href{zotero://open-pdf/library/items/CHVFSLLH?page=22&annotation=JY4VBIZY}{pdf})



=> extrait sur le réductionnisme.

\begin{figure}
    \centering
    \includegraphics[width=0.75\linewidth]{reductionisme.png}
    \legende{Modèle d'un canard dans une perspective réductioniste}{Le réductionisme tend à décrire les phénomènes par des lois physiques.}
    \label{fig:reductionnisme}
\end{figure}

\begin{displayquote}
    
\end{displayquote}

L'autrice continue sa critique en formulant un rejet du réductionnisme, c'est-à-dire à la croyance selon laquelle une notion peut être réduite en d'autres notions plus fondamentales. Elle appelle à une conception plus systémique des des relations causales. 

\begin{displayquote}
    The spirit, therefore, of these analyses might be better served by seeing them as urging a reconception of objects of inquiry in particular fields — specifically as urging their colleagues to abandon questions presupposing unidirectional or linear causal relations and to understand objects as constituted partly of the parts of which they are wholes and partly of the wholes of which they are parts. If this shift could be accomplished on internalist grounds, there would be less struggle over its acceptance.
\end{displayquote}

Ce point touche tout à fait les modèles, et particulièrement les fonctions de dommage. Les modèles sont une forme de réductionnisme, puisque l'on représente des phénomènes complexes à travers des fonctions plus simples. Il s'agit donc d'une critique récurrente, mais dont il convient de faire attention. \\


Une autre difficulté rencontrée par les modèles est la représentation des altérités et des différences. Les modèles tendent à décrire un monde homogène et neutralisé. En effet, les régions sont similaires; elles ont certe des paramétrisation différentes, mais les mécanismes décrits sont les mêmes. On gomme par là les différences culturelles et institutionnelles entre les différentes régions du monde. \\

\begin{tcolorbox}
    Partie sur Aykut et Dahan
\end{tcolorbox}

Ces critiques rejoignent celles formulées par Aykut et Dahan. 

\begin{displayquote}
    Au cours des années 1990, les pays en développement ne sont convaincus ni de la gravité du risque climatique ni du fait que ce problème les concerne. Ils contestent la prééminence de son traitement « physique » qui privilégie trop, selon eux, le global par rapport au local. Ils critiquent le point de vue de la modélisation numérique globale ou, du moins, le transfert de sa méthodologie au niveau politique ; transfert qui, disent-ils :  
    \begin{itemize}
        \item effacerait le passé (or, le Nord s’est industrialisé, équipé et il a pollué, pas le Sud)
	    \item naturalise rait le présent (en particulier, la référence à l’année 1990 dans le protocole de Kyoto est jugée inacceptable, le présent n’étant pas un acquis, mais devant être interrogé),
	   \item et globaliserait le futur (le CO 2 se globalise sans doute, pas les humains).
    \end{itemize}
\end{displayquote}

Si cette critique concerne principalement les modèles physiques tels que les modèles de circulation générale, ils peuvent être apportés également aux modèles intégrés. En effet, dans les modèles d'optimisation, où le coût total est minimisé, l'origine des émissions ou le lieu des dommages n'importe pas. 


\subsection{Le cadrage : dans la lumière ou dans l'ombre du projecteur}
% \subsection{La modélisation, une connaissance construite}

Les fonctions de dommage ne peuvent représenter qu'un nombre limité de phénomènes. Par cela, elles cadrent ce que les modèles considèrent comme dommages; si les modèles sont repris dans le débat public ou dans le régime climatique, elles cadrent même la perception possible des impacts du changement climatique. \\

\cite{aykut_gouverner_nodate} donnent cette définition du cadrage (ou \emph{framing} en anglais) : c'est un concept qui  \emph{renvoie à la la formulation d’un problème socio-technique dans les discours savants et publics, aux liens établis avec d’autres questions et aux mesures envisagées pour traiter le problème (solutions technologiques ou politiques, approches globales ou locales, etc.), souvent sous-jacentes à son évaluation}. Ils donnent différents éléments permettant de caractériser le cadrage du régime climatique. Le premier est sa globalité, c'est à dire l'idée selon laquelle le changement climatique est un problème global, avec des sources globales et des réponses globales. Cette idée prend sa source dans les modèles climatiques, qui, en représentant le climat comme une réalité globale et sans frontières, tendent à effacer les différences locales de responsabilité ou de vulnérabilité. Le cadrage du débat climatique par les sciences du climats se fait par trois canaux principaux : 



\begin{displayquote}
\begin{itemize}
    \item la **concentration sur les modèles globaux de l’atmosphère** comme outil incontournable des projections climatiques, y compris pour les prévisions régionales obtenues par descente en échelle (downscaling) des modèles globaux, a contribué à globaliser les problèmes, à désigner l’arène mondiale/globale comme l’échelle unique de traitement du risque climatique ;
	\item le **réductionnisme physico-chimique des sciences du climat** tend à mettre en avant les caractéristiques universelles des GES et à les séparer de leur signification sociale locale. Les molécules de méthane des rizières ou celles de gaz carbonique des voitures jouent un rôle identique dans la mise en équation de l’effet de serre. C’est ce que Demeritt (2001) a qualifié de déterminisme environnemental tacite ;
	\item la **focalisation dans les modélisations sur les « évolutions probables »** a longtemps contribué – contrairement à ce que les critiques récurrentes de « l’alarmisme » des rapports scientifiques suggèrent – à une marginalisation des scénarios du pire et de l’éventualité de changements brusques, ou tipping points. Shackley et Wynne (1996) constatent, dans une étude sur le traitement des incertitudes dans l’expertise climatique globale, une « mise à l’écart des extrêmes » (tuning out of extremes).
\end{itemize}
\end{displayquote}

Ce cadrage est accentué par les modèles intégrés, en ne permettant le débat que sur les concepts et à partir des hypothèses modélisées. Comme le souligne le GIEC, les modèles intégrés limitent les horizons possibles aux horizons imaginés par les modélisateurs : 

\begin{displayquote}
    IAM analysis could focus on only a subset of relevant futures and thus push society in certain directions without sufficient scrutiny. \cite{intergovernmental_panel_on_climate_change_ipcc_annex_2023}
\end{displayquote}

\begin{displayquote}
    - « The vast majority of IAM pathways do not consider climate impacts (Schultes et al. 2021). Equity hinges upon ethical and normative choices. As most IAM pathways follow th » ([“Mitigation Pathways Compatible with Long-term Goals”, 2023, p. 304](zotero://select/library/items/38PUG9F2)
\end{displayquote}

\cite{cointe_ar6_2024} complète cette remarque en ajoutant que le \emph{corridor} des possibles devient de plus en plus étroit, et limite certaines alternatives, comme des scénarios sans croissance, post-croissance ou de décroissance. 

Plus spécifiquement, les fonctions de dommage peuvent contribuer à cadrer le débat par de nombreux moyens. 

La monétarisation (voir section \ref{monetarisation}) implique une dichotomie claire entre ce qui est monétaire et ce qui ne l'est pas. Elle ne permet de modéliser que les phénomènes monétaires, et reflète ainsi une importance démesurée des indicateurs économiques classique par rapport à d'autres composantes, tels que le bien-être, la santé ou la biodiversité. Prendre en compte ses différents éléments implique dès lors de les monétariser, ce qui n'est pas sans question. 
De plus, ces modèles vont favoriser des solutions qui s'inscrivent dans le système financier, sans pouvoir proposer d'alternatives. Ainsi, les solutions proposées seront systématiquement des taxes, des coûts du carbone ou d'autres types d'instruments financiers.


=> lien direct avec l'approche avec la science : 

\begin{displayquote}
    Funtowicz and Ravetz question whether policy-relevant issues can be addressed using conventional scientific techniques ‘when facts are uncertain, values in dispute, stakes high, and decisions urgent.’ (Ref 37, p. 744) In this situation, they claim, a post-normal science that explicitly acknowledges the interweaving of knowledge and ethics is required. » ([Beck et Krueger, 2016, p. 630]
\end{displayquote}


\subsection{Trouver un niveau optimal de changement climatique, une approche fondamentalement normative}

Une approche à la question du rôle des modèles est celle de la distinction entre les modèles normatifs et positifs. Selon cette catégorisation, très présente en économie, il y aurait d'une part les \emph{sciences économiques}, qui visent à décrire le monde tel qu'il fonctionne, de la manière la plus neutre et précise possible (approche positive). D'autre part, il l'\emph{économie politique} serait plus normative, et viserait à décrire ce que serait un bon fonctionnement de l'économie (approche normativre). 

Ainsi, pour Nordhaus \cite{nordhaus_dice_2013}, \emph{one of the issues that pervades the use of IAMs is whether they should be interpreted as normative or positive}. Selon lui, les modèles de simulation (dont ceux de physique du climat, tels que les modèles de circulation générale) sont positifs, alors que les modèles d'optimisation peuvent être positifs ou normatifs. Par exemple, dans le cas où ils représentent des marchés fonctionnels, ils seraient une représentation précise du fonctionnement de ceux-ci, et donc positifs, tandis qu'ils seraient une approximation correcte du fonctionnement d'autres mécanismes. \\

On peut faire plusieurs critiques sur cette remarque. D'abord, considérer qu'un modèle d'optimisation représente correctement un marché puisque les \emph{mécanismes de marché sont des outils de maximisation ou de minimisation} revient à réduire l'importance de mécanismes non monétaires au sein du fonctionnement des marchés, ce qui est de moins en moins consensuel. 

Ensuite, comme nous l'avons développé plus haut, il est difficile de penser qu'un modèle, même de simulation ou même de sciences du climat, soit neutre. 

Enfin, et si l'on considère que les remarques précédentes sont des critiques inhérentes à tout exercice de modélisation, il convient de remarquer que les modèles d'optimisation sont intrinsèquement normatifs. \\

D'une part, parce qu'en affectant aux dommages une valeur, ils leur confèrent un caractère normatif. Par exemple, affecter à une vie humaine une valeur mobilise un système normatif et des réflexions morales importantes; par ailleurs, ces modèles affectent à différents événements des valeurs différentes. Il y a donc un classement, ou un ordonnancement, qui reflète une hiérarchisation sur une échelle normative. Un argument souvent développé est que le prix dévoile précisément le niveau de préférence, et donc la valeur empirique attribuée par les agents à tel ou tel bien ou service; cependant, ici, les valeurs ne sont pas issues de choix d'agents qui font valoir leurs préférences, mais bien des choix de modélisation. En cherchant à classer, à trouver une solution \emph{optimale}, ces modèles mettent en avant un système normatif particulier. 

D'autre part, parce qu'en contraignant les solutions et les mesures à des valeurs monétaires, ils en font le seul cadre d'analyse possible des effets du changement climatique. En effet, ces modèles valorisent la place d'outils monétaires, qui sont ainsi les seuls à pouvoir être implémentés : taxe carbone, mécanismes de marché, etc. Ainsi, comme le dit \cite{mercure_modelling_2019}, \emph{the finding that an optimal resource allocation is not achieved due to frictions and market failures, ultimately reflects a normative philosophy of science}. 

\begin{displayquote}
    This discussion implies that we can interpret optimization models as a device for estimating the equilibrium of a market economy. As such, it does not necessarily have a normative interpretation. Rather, the maximization is an algorithm for finding the outcome of efficient competitive markets.
\end{displayquote}

\begin{tcolorbox}
    parler du fait que l'optimisation c'est forcément très normatif. 

    => cf nordhaus (voir partie interprétation dans logseq)
\end{tcolorbox}


\subsection{Prendre en compte le non-monétaire}


« The difficulty in fully representing the extent of climate damages in monetary terms may be the most important and challenging limitation of IAMs and it is mostly directed to costbenefit IAMs. However, all categories of IAMs present important limitations (Annex III.I.9). » (\href{zotero://select/library/items/2SDDNUUF}{“Annex III: Scenarios and Modelling Methods”, 2023, p. 1844}) (\href{zotero://open-pdf/library/items/CHVFSLLH?page=4&annotation=YT933ZM4}{pdf})
 

\begin{itemize}
    \item « there are concerns that IAMs are missing important dynamics » (\href{zotero://select/library/items/2SDDNUUF}{“Annex III: Scenarios and Modelling Methods”, 2023, p. 1861}) (\href{zotero://open-pdf/library/items/CHVFSLLH?page=21&annotation=WDMBNU3A}{pdf})
=> et donc ne sont pas vraiment précis, ou passent à coté de certaines choses

\end{itemize}

\begin{displayquote}
    First, including direct impacts on the environment and human health (sometimes called ‘non-market’ impacts) increases our estimate of the total cost of climate change on this path from 5\% to 11\% of global per-capita consumption. There are difficult analytical and ethical issues of measurement here. The methods used in this model are fairly conservative in the value they assign to these impacts.  ([Stern, 2007, p. 10]
\end{displayquote}

Enfin, l'approche néomarxiste fait la promotion d'une science plus émancipatoire, et moins élitiste. Longino évoque notamment les travaux de Hilary Rose et Steven Rose, qui décrivent une science plus émancipatoire, qui aurait \emph{dépassé le clivage entre l'objet et le sujet, entre le rationnel et l'emotionnel, et qui ne serait plus dominée par une rationalité instrumentale. Elle serait caractérisée par des relations sociales démocratiques, c'est-à-dire l'abandon de l'élitisme, et ses théories incorporeraient une vue dialectique de la nature}.



\section{Les doutes normativement inappropriés, ou la limite entre mauvaise science et fabrique de l'inaction}

Cette section repose sur les approches des \textit{ignorance studies}, notamment \cite{melo-martin_fight_2018} et \cite{gross_routledge_2015}, avec comme ressources complémentaires \cite{noauthor_carnet_2024} et \cite{proctor_agnotology_2008}. Elle est tirée de réflexions tirées du cours de Mathias Girel à l'ENS \cite{girel_vertus_2023}. 



+ Stern / Nordhaus argument (c'est sûr qu'ils se sont balancé des trucs à la gueule) 

+ steve keen \cite{keen_appallingly_2021}

\subsection{Rester dans sa retenue ou s'engager ?}

\subsection{L'impossible neutralité des modèles}

Helène Longino définit deux types de valeurs. D'une part, les \emph{constitutive values}, qui désignent les valeurs méthodologiques admises par la comunauté, c'est-à-dire les bonnes pratiques et méthodes. D'autre part, les \emph{contextual values}, qui sont des valeurs personnelles, sociales ou contextuelles des individus. Ces valeurs sont plus normatives, et reflètent le cadre social et culturel dans lequel s'inscrit la démarche scientifique. \\

Se pose dès lors la question du rapport entre ces deux niveaux de valeurs. Dans quelle mesure les valeurs contextuelles influencent les valeurs constitutives, c'est-à-dire dans quelle mesure les normes sociales d'un contexte donné vont influencer la production scientifique ? Et dans quelle mesure les valeurs constitutives influencent les valeurs contextuelles, c'est-à-dire dans quelle mesure la production scientifique influence les normes sociales ? Cette question est celle de l'autonomie de la pratique scientifique du contexte personnel, social et culturel, c'est-à-dire précisément la question de la neutralité de la science. Elle répond de manière très claire à cette question : non seulement les deux interagissent fortement, mais cette interaction est au cœur de la pratique scientifique : 

\begin{displayquote}
    I will argue not only thatscientific practices and content on the one hand and social needs and values on the other are in dynamic interaction but that the logical and cognitive structuresofscientific inquiry.require such interaction. \textit{p. 5}
\end{displayquote}

L'autonomie de la production scientifique vis-à-vis des valeurs sociales est donc un mythe. Cependant, et contrairement à ce qui est généralement admis, cela ne remet pas en cause l'intégrité de la recherche. 

\begin{displayquote}
    Autonomy and integrity are separable attributes, and I shall consider them in sequence.
\end{displayquote}

La question de la neutralité de la science est régulièrement abordée en épistémologie. Une autrice a particulièrement abordé ce sujet, en montrant que la science est avant tout une pratique sociale, qui s'inscrit dans des dynamiques et un contexte particulier. Il s'agit d'Hélène Longino, dans \emph{Science as social knowledgde} \cite{longino_science_1990}. Elle développe plusieurs points qui vont être intéressants dans la perspectives des modèles. \\

D'abord, elle développe l'idée que la science n'est pas pure ou dénuée de valeur; au contraire, c'est une base solide pour construire des valeurs ensuite. Il ne s'agit pas de lutter pour une science sans biais ou sans valeur; mais plutot d'inclure ces valeurs au coeur du projet scientifique. 

\begin{displayquote}
    Instead of remaining passive with respect to the data and what the data suggest, we can, therefore, acknowledge our ability to affect the course of knowledge and fashion or favor research programs that are consistent with the values and commitments we express in the rest of our lives. From this perspective the idea of a value-free science is not just empty but pernicious. \textit{page 191}
\end{displayquote}

Elle va ensuite plus loin, en indiquant qu'il y a un choix fort à réaliser, entre s'accorder avec les systèmes normatifs traditionnels, ou s'accorder avec son propre système de valeurs. Le système normatif, qui est forcément inclu dans le processus scientifique, résulte dès lors d'un choix conscient (y compris s'il implique de rester dans le système de valeur traditionnel ou dominant). 

\begin{displayquote}
    In particular we can choose between being accountable to the traditional establishment or to our political comrades \textit{p 191}
\end{displayquote}

Le contexte de la modélisation est particulièrement intéressant de ce point de vue. En effet, comme nous l'avons montré plus haut, les hypothèses sont omniprésentes, et réflètent une conception du monde. Il y a donc un choix réel. Contrairement à ce que certains modélisateurs affirment, il ne s'agit pas d'une réalité objective ou neutre, mais bien d'une sélection hautement normaitve. \\




\cite{helgeson_attention_2022} => pote de Tuana qui parle des valeurs

\subsection{La responsabilité}

\begin{displayquote}
    « But the belief that nonepistemic motivations necessarily result in NID is mistaken. As many have argued, all science, or nearly all science, is in fact unavoidably motivated by some nonepistemic aims » ([Melo-Martín et Intemann, 2018, p. 35]
\end{displayquote}
=> renvoit à la notion d'éthique intrinsèque



 % Create file to add
\chapter{Interpréter le modèle dans le monde réel}
\label{chapter:socio}
\newrefsegment

\PEEL{Il est nécessaire de diversifier les représentations des dommages, mais aussi de prendre en compte les dommages non monétaires. Les decideureuses n'ont pas, n'auront pas et n'ont pas vocation à avoir de connaissances ou de compétences techniques.}{Fournissez des preuves, des données ou des citations qui soutiennent votre point.}{Expliquez en quoi les preuves que vous avez fournies sont pertinentes et comment elles appuient votre point.}{Il sont donc tributaires des modélisateur.ices pour l'interprétation des modèles.}

\begin{displayquote}
    It is the developer’s firm belief that most researchers should be locked away in an ivory tower. Models are often quite useless in unexperienced hands, and sometimes misleading. No one is smart enough to master in a short period what took someone else years to develop. Not-understood models are irrelevant, half-understood models treacherous, and mis-understood models dangerous\footnote{Le modélisateur a la conviction profonde que la plupart des chercheurs devraient être enfermés dans des tours d'ivoires. Les modèles sont souvent inutiles dans des mains inexperimentées, et parfois trompeurs. Personne n'est assez malin pour maitriser dans une courte période ce qu'une autre a développé en plusieurs années. Pas compris, les modèles ne sont pas pertinents; à moitié compris, il sont traitres; mal compris, ils sont dangereux.}. \\
    
    Richard Tol et David Anthoff, documentation du modèle FUND
\end{displayquote}
    
\chapterabstract{
 Cette partie du mémoire s'intéresse à la manière dont les résultats issus de la modélisation intégrée sont interprétés dans le débat public et permettent de prendre des décisions. Plus précisement, on s'interroge sur les utilisations des modèles par divers acteurs de la politique climatique : techniciens, décideurs, journalistes, scientifiques, activistes.  Pour cela, on réalise une série d'entretiens semis-directifs. Ils ont pour but de répondre aux questions suivantes : comment est interprétée l'incertitutde inhérente aux modèles pour la prise de décision ? 
}

Comme nous l'avons vu dans les chapitres précédents, les hypothèses jouent un rôle fondamental et déterminant dans le fonctionnement du modèle et les résultats qu'il propose. Ainsi, utiliser le modèle dans un cadre adapté requiert d'avoir une compréhension fine de ces hypothèses et de ces choix de modélisation. Cette compréhension permet de mieux articuler les différents enjeux éthiques qui sont à l'œuvre dans le processus. 

Une des limitations principales à cette compréhension est la présence de communauté distinctes, mues par des objectifs différents, et ayant des caractéristiques différentes. On définit ici la notion d'utilisateur final du modèle, qui désigne les personnes amenées à utiliser, sous une forme ou sous une autre, un modèle, ses résultats, ou les interprétations de ces résultats. Cette catégorie est extrêmement large, et regroupe entre autres les administrations publiques, les décideurs politiques, les journalistes, les membres du milieu associatif. Si ces profils sont très divers, ils se rejoignent sur deux points. D'abord, il s'agit d'individu ayant des actions liées au changement climatique, prenant et / ou influençant des décisions politiques. Ensuite, ce ne sont pas des experts de la modélisation, en ce sens qu'ils ne sont pas à l'origine de modèles. On a donc deux communautés distinctes : les modèles sont conçus et maitrisé au sein de la communauté des modélisateurs, mais leurs résultats et interprétations sont utilisés et converti en actions au par les utilisateurs finaux des modèles. 

Les membres de cette seconde partie sont ceux qui vont animer le modèle, et conduire ses conclusions dans le monde réel. D'une certaine manière, il s'agit d'une des étapes les plus importantes, ou en tout cas les plus politiques, puisqu'il s'agit d'ancrer les hypothèses normatives dans le monde réel, ce qui en fait dès lors des positions politiques. On sort alors du pur cadre théorique et conceptuel pour rejoindre celui de l'action publique. Ce passage est sensible, puisque les biais contenus dans la partie théorique deviennent autant d'inégalités ou d'injustices : de limitations scientifiques, ils deviennent de véritables choix de sociétés.

Ce raisonnement nous amène à la question suivante : dans quelle mesure les utilisateurs finaux de modèles sont en mesure de comprendre les modèles et leurs implications ? 

Nous allons répondre à cette question de manière qualitative (voir encadré méthodologique), à travers des entretiens semi-directifs avec des acteurs de la modélisation ou des utilisateurs finaux des modèles. Nous commencerons par une contextualisation théorique, puis nous détaillerons les résultats des entretiens à travers ces trois axes : d'abord, les modèles intégrés sont trop complexes pour être manipulés par des publics non avertis; ensuite, il existe d'autres méthodes de modélisation, telle que la modélisation participative, qui permettent de co-concevoir les modèles; enfin, les décideurs politiques ont un besoin fort de données quantitatives. 




\begin{methodbox}[Entretiens semi-directifs]

On réalise des entretiens semi-directifs avec des acteurs du débat public sur les questions climatiques. \\

Cette catégorie est volontairement large. On classe ensuite les enquêtés en quatre catégories d'acteurs : les scientifiques, les techniciens, les politiques et la société civile. Les \textbf{scientifiques} désignent les acteurs dont la parole est reconnue comme porteuse d'un message scientifique. Il s'agit généralement de modélisateurs, et parfois d'acteurs gravitant autour des milieux universitaires : comité d'éthique, communication scientifique non-vulgarisée. Les \textbf{techniciens} sont les acteurs d'administrations publiques n'ayant pas de mandats électifs. Il s'agit le plus souvent de spécialistes de sujets spécifiques, qui transcrivent les connaissances climatiques en plan d'action ou en texte réglementaires. Les \textbf{politiques} sont tous les acteurs qui sont dotés d'un mandat électif. Leur spécificité est d'être amené à prendre des décisions, à trancher dans des contextes où les conséquences des différentes actions sont soit empreinte d'incertitude soit de choix moraux. Enfin, la \textbf{société civile} désigne les acteurs non-institutionnels qui s'emparent de sujets climatiques. Il s'agit en particulier de journalistes, mais aussi d'activistes ou de personnes engagées dans le monde associatif. \\

Les entretiens sont du format semi-directif. Une grille d'entretien est soumise aux enquêtés (voir annexe \ref{ch:grille}). Cependant, les réponses sont longues et libres, et peuvent donner lieu à des questions inédites. Symétriquement, toutes les questions ne sont pas traitées dans tous les entretiens. Ces entretiens sont ensuite retranscrits, et des passages sont sélectionnés. \\

Notre échantillon se compose de 7 enquêtés : 3 personnes appartiennent à la communauté des scientifiques, 2 à la société civile, 2 techniciens. Plusieurs personnes appartenant à la catégorie \textit{politique} ont décommandé leur entretien en raison de la situation politique en France. Les entretiens ont duré en moyenne une heure, dans une grande flexibilité vis-à-vis de la grille de question. \\

Pour des raisons de temps, il n'a pas été possible de réaliser plus d'entretiens. Il faut donc prendre avec précaution ces conclusions, qui ne reflètent qu'une petite partie des perceptions qu'ont les acteurs de la modélisation des dommages. 

\end{methodbox}


%Beaucoup de remarques issues de \cite{intergovernmental_panel_on_climate_change_ipcc_mitigation_2023}



\section{Introduction, méthodologie et questions}

Le rôle des modèles dans le débat public est un sujet largement abordé dans la littérature de la sociologie de la modélisation. Dans les années 1970, les différentes versions du modèle World popularisent l'utilisation de systèmes dynamiques pour représenter le monde dans une approche systémique. Ils représentent le fonctionnement du monde autour de cinq grandes familles de variables, interdépendantes, et aboutissent à la conclusion qu'une croissance infinie n'est pas soutenable \cite{forrester_world_1971}.  Ces modèles, et en particulier World3, donnent naissance au rapport dit Meadows, ou encore rapport au Club de Rome, intitulé \emph{limits to growth} \cite{meadows_limits_1972}. Ce rapport a eu un fort retentissement politique et médiatique, et a contribué à la popularisation de l'idée que la croissance économique pure n'est ni souhaitable, ni soutenable \cite{edwards_global_1996}. 
Nous explorerons ici les liens qui se tissent entre la fabrique des modèles et leur utilisation, d'abord dans une perspective générale, puis dans le cas plus spécifique des fonctions de dommage. Nous questionnons ainsi le modèle linéaire de la science, développé par \cite{aykut_gouverner_nodate} \ref{model-lineaire}. Notre question de recherche est la suivante : comment les enjeux éthiques liés à la modélisation des dommages sont-ils perçus et interprétés par les utilisateurs finaux des modèles ? 

\subsection{Le rôle des modèles et des scenarios dans la prise de décisions}

Depuis les années 1990, les scénarios socio-économiques d'émissions jouent un rôle central dans les rapports du GIEC. Ils sont ainsi au cœur de ce qui est aussi qualifié de \emph{policy-relevant knowledgde}. Ils sont désormais produits de manière quasiment exclusive par les modèles intégrés \cite{cointe_organising_2019}, avec la volonté explicite de servir à la production de politique publique \cite{Weyant}. \\



%\cite{cointe_organising_2019} => le rôle des IAMs dans la prise de décision

Dès leur conception, les modèles intégrés sont orientés vers de la prise de décision. Ils sont plus simples que les modèles physiques tels que les modèles couplés, et ont pour vocation de fournir une intuition des effets des politiques climatiques. En ce sens, ils constituent un outil heuristique destiné à être utilisé soit directement par les administrations, soit au plus proche des administrations. Ils produisent des évaluations, et permettent des comparaisons, et non des prédictions \cite{edwards_global_1996}. 

\begin{displayquote}
“whether or not they are ever used directly by policymakers, these models are contributing to what I believe to be a fundamental shift in the structure of scientific work toward trans-disciplinary collaboration and communication. This means that ESMs and IAMs in fact contribute substantially to the basis of global change politics, in the important sense that they serve as one of the organizing principles of a large, growing, epistemologically coherent community.” (Edwards, 1996, p. 152)
\end{displayquote}


\cite{wynne_institutional_1984} montre que l'évaluation des modèles, y compris par les pairs, est difficile. Il développe l'exemple de l'IIASA Energy Study. Celle-ci a fait l'objet d'une attention particulière quant à la cohérence interne. Cette rigueur affichée a été perçue à l'extérieur, y compris par les pairs, comme signe d'une grande précision et objectivité du modèle.  Bien que cet exemple soit ancien, il montre comment la perception des résultats influence le regard critique qui est porté sur le modèle. 

\subsection{L'implication des choix de modélisation sur la décision}

Nous avons développé plus haut le concept de cadrage, ou de framing. Un exemple intéressant de comment ce cadrage se retranscrit dans le discours public est celui de l'absence de scénario de décroissance. \cite{cointe_understanding_2023} montrent que dans les 1,071 scenarios utilisés pour produire le 5ème rapport du GIEC (AR5) dans lesquels un taux de croissance annuel était calculable, tous avaient un taux de croissance supérieur à zéro. 
Il existe pourtant des manières de modéliser la décroissance. \cite{briens_decroissance_2015} en a fait l'objet de sa thèse. A travers différentes techniques, il propose d'explorer, dans une perspective de prospective, des scénarios de décroissance. 

Comme décrit plus haut, le rôle du GIEC est précisément de faire la synthèse des connaissances disponibles dans un format qui soit utile à la prise de décision. Exclure les scénarios de décroissance revient dès lors à les retirer de la connaissance valide, mais aussi des options de politique publique possibles. Ainsi, la variété des scénarios produits et pris en compte a des répercussions directes sur la nature des options de politique publiques envisagées. 

\subsection{La diffusion de connaissance auprès des utilisateurs finaux}

Pour les décideurs, choisir un modèle est une activité difficile. \cite{boulanger_models_2005} identifient cinq critères qui permettent d'évaluer la pertinence d'un modèle pour une politique publique : l'approche interdisciplinaire, le traitement de l'incertitude, la perspective à long-terme, la perspective global-local, et la participation. Nous reviendrons sur la notion de participation un peu plus loin. Cette approche illustre comment la difficulté d'accès pour des décideurs aux modèles intégrés. Ils doivent en effet choisir entre des options nombreuses, qui sont semblables sur de nombreux points, mais différentes sur d'autres. Ces différences, comme nous l'avons vu, peuvent parfois sembler très faible, mais peuvent avoir des conséquences importantes sur le comportement du modèle. \\


Pour répondre à cette difficulté, des auteurs proposent de créer des services scientifiques spécifiques, dont le rôle est précisement d'avoir une compétence technique au service de questions politiques. C'est le cas de \cite{auer_climate_2021}, qui proposent de créer un service des scénarios du changement climatique. Un tel service aurait pour mission de \emph{permettre à une communauté large de décideurs, acteurs économiques, financiers et régionaux d'avoir accès et d'utiliser des scénarios de changement climatique de pointe et de manière intéressante}. Il ferait le lien entre la communauté scientifique et les besoins des utilisateurs, en éclairant les questions de ces derniers au regard des possibilités et limites des modèles. \\

Un décalage qui crée de la friction entre les communautés scientifiques et le grand public est la gestion de l'incertitude. \cite{shackley_representing_1996} montre que cette tension vient d'une part de l'omniprésence de l'incertitude dans la production scientifique, et d'autre part de la demande de la part du public pour une science fiable et certaine (ou du moins perçu comme telle). En effet, l'incertitude est inhérente à la recherche, et particulièrement dans le cadre de la recherche autour du changement climatique. Elle fait partie des sujets régulièrement discutés au sein de la communauté. En revanche, exposée au grand jour, elle met en question la crédibilité de la production scientifique. Il y a dès lors un dilemme lors de la diffusion de cette incertitude : masquer cette incertitude pour avancer des résultats, qui par ailleurs peuvent être reconnus et consensuels malgré l'incertitude résiduelle; ou afficher cette incertitude, en accord avec les principes d'intégrité et de transparence, au risque que les résultats soient perçus pour moins fiables que ce qu'ils ne sont, et d'une perte de confiance ? 
Cette tension inhérente à la science est particulièrement intéressante dans le cas des fonctions de dommage. En effet, celles-ci sont très sensibles au traitement des incertitudes. Les exposer pleinement décrédibilise les modèles intégrés, bien que ce soient des outils d'aide à la décision essentiel et partiellement valable malgré ces incertitudes. Les masquer revient à imposer une vision et une interprétation. 



\subsection{La formation d'une communauté des modélisateur}

La complexité des modèle est un frein au sein même de la communauté. En effet, le nombre important de variables et d'équations, et l'immense littérature relative aux phénomènes que cherchent à modéliser les modèles intégrés font que chaque modèle doit être approprié pour lui même. Ainsi, les modélisateurs ne sont souvent pas familier d'autres modèles intégrés, pourtant proches les uns des autres. Pour faire face à une communauté fractionnée, deux initiatives ont vu le jour. \\

D'abord, l'\emph{Integrated Assesment Modeling Consortium (IAMC)}. Ce consortium a été créé en 2007 pour \emph{coordonner la production de RCP pour le GIEC} \cite{cointe_organising_2019}. A travers des colloques, le consortium vise à faciliter les échanges de point de vue, de documentation, et à coordonner les différents projets de modélisation intégrée. Nous avons d'ailleurs, dans la partie \ref{chapter:litrev}, beaucoup utilisé les données issues du wiki de l'IAMC \cite{noauthor_models_nodate}. Il est cependant important de noter que ces informations étaient souvent très lacunaires (des informations très vagues ou inexistantes) voir fausse (des équipes dont les modèles étaient indiquées comme ayant des fonctions de dommage ont été contactées, et nous ont bien confirmé l'absence de fonction de dommage dans ces modèles). \\

Ensuite, des \emph{Model Intercomparison Projects (MIPs)}. \\




%\cite{keppo_exploring_2021} => sur les limitations des IAMs liées à la communication à ce sujet

%\cite{davidson_climate_nodate} => sur l'intérêt de penser au pire scénario climatique pour limiter les dégâts de la potentielle crise


%\cite{dekker_consensus_2022} => sur les bases de données IIASA
  
%D'autres outils : la modélisation participative : 
%\cite{etienne_modelisation_2010}






\section{Résultats}

\subsection{La difficile mais nécessaire transparence}

Face à l'importance des dimensions éthiques et normatives des modèles intégrés, une attention de plus en plus soutenue a été accordée à la transparence.  Si c'est désormais un objectif affiché, c'est tout de même un véritable défi, comme nous le confie un modélisateur : 

\begin{displayquote}
    What is transparency ? Tranparency is not easy. One equation is easy to understand, 50 equations is difficult. 
\end{displayquote}
Cette remarque met en évidence une des difficultés principale du lien modélisateur - utilisateur, dont le renforcement est pourtant un des objectifs de la transparence. \cite{keppo_exploring_2021} soulignent que les hypothèses centrales des modèles sont souvent mise en avant, mais que chaque module du modèle est influencé  aussi par des hypothèses faites ailleurs dans le modèle. \\

De la même manière, le fait d'ouvrir le code et le modèle est une étape bienvenue pour augmenter la transparence; elle n'est pourtant pas suffisante. \cite{keppo_exploring_2021} avancent que peu de gens sont capables de comprendre et de faire tourner le programme, même lorsque le code est ouvert. Par ailleurs, \emph{implications of specific assumptions only become clear when one understands the model well}, ce qui maintient une grande barrière à la compréhension, quand bien même le modèle est ouvert. 

Ainsi, si l'ouverture du modèle fait sauter certaines barrières techniques, elle ne permet pas pour autant d'en diminuer tous les limites. 

\begin{displayquote}
    include lo
\end{displayquote}
Cependant, même si l'audience n'est pas très large, la transparence reste un atout. 

\begin{displayquote}
Our wiki is intented for two objective: internal, we use it; external, it's open for modellers. If any one wants to download the model and test it, they can. 
\end{displayquote}

\subsection{Seuls les modélisateur.ices comprennent leurs modèles (et encore)}

\textit{Une incompréhension de la part du grand public
=> exemple de journaliste ; SGPE; }

Les modèles sont difficiles à comprendre pour le grand public. La première barrière est souvent l'accès à la connaissance, dans le sens de faire la démarche d'accéder à la connaissance. 
Par exemple, lors d'un entretien avec une haute fonctionnaire d'une administration française, celle-ci reconnait ne pas se référer régulièrement aux rapports du GIEC, car ceux-ci sont trop volumineux. Par extension, elle ne cherche pas à avoir accès aux connaissances antérieures, et notamment aux modèles, principalement par manque de temps. On peut néanmoins remarque que c'est précisément pour synthétiser une connaissance variée et dispersée qu'est né le GIEC. \\

\textit{=> y compris chez les personnes qui ont déjà une certaine connaissance : activiste} \\


On constate le même raisonnement chez des personnes qui ont une pratique antérieure de la modélisation, de l'informatique ou des mathématiques par leur formation. Par exemple, une enquêtée active dans une association de plaidaoyer climatique déclare ne jamais avoir ouvert un modèle, alors même qu'elle en a les capacité technique et que le modèle est accessible en open-source. 

\begin{displayquote}
    Ajouter la quote de activiste
\end{displayquote}

Les limitations d'accès aux modèles sont lourdes, et dépassent des contraintes uniquement matérielles ou capacitaires. \\

\textit{Pas forcément d'envie non plus
=> journaliste ne souhaite pas franchement savoir; au contraire, c'est plutot pour lui le rôle du journaliste de connaitre ses sources} \\

Un enquêté journaliste apporte un autre point de vue sur la question. Il n'a lui non plus pas consulté directement les modèles, mais se base sur les rapports et autres analyses tirés de ces modèles, tels que les rapports du GIEC. Pour lui, cette limitation technique est inévitable, et le rôle du journalisme est précisément de choisir ses sources pour s'assurer qu'elles sont fiables. 

\begin{displayquote}
    Citation de journaliste
\end{displayquote}

Malgré les tentatives de transparence, il y aurait donc une part irréductible à la distance entre le modèle et ses utilisateurs finaux. 

\begin{displayquote}
    En fait, ce qui manque pour moi dans ces modèles, c'est des sciences sociales. C'est à dire qu'on a des modèles qui sont conçus par des gens qui sont très intelligents, qui sont des ingés super forts, qui sont capables de tout calculer et qui veulent en plus assez souvent à la fin qu'on est zéro au bout de la ligne et que le truc retombe d'une manière à peu près tranquille. Ça en fait du coup des modèles qui sont très, très théoriques dans la manière dont les choses se passent, dans l'économie et la société, la transition, elles ne se passent pas comme les modèles le dessinent pour plein de raisons.
\end{displayquote}
Par ailleurs, cette séparation entre la modélisation et l'interprétation est parfois perçue comme souhaitable. 
\begin{displayquote}
    C'est-à-dire le travail en amont, c'est le travail des modélisateurs. Ça, je pense que c'est à eux aussi d'être capables de dire, on rassemble les parties prenantes pour que nos présupposés, à nous, nos biais soient confrontés à d'autres biais et qu'ensuite on trouve le meilleur équilibrage de la manière la plus sincère possible. Évidemment, c'est difficile, c'est jamais parfait, mais qui brasse le plus de monde possible. Le travail des journalistes, il intervient quand même plutôt après. C'est-à-dire nous, derrière, on peut expliquer ce qu'il y a dans les modèles, confronter les résultats de ces modèles au débat public, aux décisions de politiques publiques des décideurs, etc. Mais ce n'est pas au même endroit, je pense.
\end{displayquote}
Distinction entre 
\begin{displayquote}
    C'est marrant parce que j'ai fait un truc devant les anciens des mines, il y a 15 jours ou 3 semaines. Et il y avait un retraité d'EDF qui était là,  qui vient me voir. Et il me dit, mais je ne comprends pas. Moi, j'ai tout à fait la solution face au changement climatique. La tonne de carbone évité est beaucoup moins chère dans les pays du Sud. Donc, en fait, il faut tout mettre, tout l'argent pour que c'est plus simple et c'est beaucoup moins cher. C'est dix fois moins cher. J'ai tout calculé moi-même.Donc là, il me montre son cahier, il y a tous ces calculs et tout ça. Donc, c'est à la fois un peu mignon, mais en même temps, pour moi, ça témoignait un peu de ce que je disais tout à l'heure, c'est à dire l'approche très, très techno, très ingé du problème. Comme si on pouvait dire, ah, ben en fait, on se fout, on fait baisser les émissions que dans les pays du Sud. Et nous, on ne le fait pas puisque ça coûte moins cher de le faire là bas. Enfin, c'est politiquement et socialement un peu impossible de tenir ce genre de discours.
\end{displayquote}


\subsection{La force de la modélisation participative}

Face à ce constat, une méthodologie de modélisation radicalement différente apparait. Plutot que des modèles importants, complexes, développé pendant des années par des équipes scientifiques de manière indépendante de l'usage finale des modèles ou de leur interprétation, la méthode de la modélisation participante prend en compte les acteurs, leurs visions et leurs valeurs dès la conception du modèle. \\

Le paradigme est tout à fait différent : plutot que de chercher à produire un modèle objectif et toujours plus réaliste, les acteurs de cette technique prennent à bras-le-corps les oppositions normative des acteurs. 

Répondre au problème par des modèles plus spécifiques
=> SGPE + biodiversité
=> enqueté modélisation participante

Une maniere de s'ouvrir à la communauté
=> Open source + tous les produits annexes chez WILIAM

\subsection{Les décideurs politiques ont besoin d'objectifs chiffrés}

A la fin, le modele doit sortir un chiffre, n'importe lequel
=> 


\section{Discussions}










\chapter{Conclusion}
\label{chapter:conclusion}
\newrefsegment


% Rappel de l'objectif

La problématique de ce mémoire est partie d'une première intuition, en voyant la fonction de dommage de DICE (équation \ref{eq:df_dice2023}) : celle que cette forme était absurdement simple et simpliste, et qu'on ne pouvait pas (ou même, ne devait pas) simplifier autant de paramètres, de variables, dans une formule semblale à une formule magique. Cette 




% Synthèse des chapitres

Nous avons d'abord cherché à voir les différentes formes de fonctions de dommage. \\

Nous avons ensuite cherché à mesurer l'effet d'hypothèses éthique, à travers l'exemple de l'équité spatiale. \\

Nous avons ensuite replacé ces choix dans un contexte épistémologique qui permet de les interpréter à travers un prisme éthique. \\

Nous avons finalement interrogé ce lien entre la science et les utilisateurs finaux du modèle. \\



% Reflexions ethiques 


% limites 

% implications pratiques 


% perspectives futures

% reflexions générales

%% Prevent urls running into margins in bibliography
\setcounter{biburlnumpenalty}{7000}
\setcounter{biburllcpenalty}{7000}
\setcounter{biburlucpenalty}{7000}



%% ----------------------------------------------------------------------
%%    Appendix (Letters for chapters)
%% ----------------------------------------------------------------------

\appendix

\renewcommand{\chaptername}{Annexe}
\newrefsegment


\chapter{Glossaire}
\begin{multicols}{2}
    \printglossary[title=]
\end{multicols}

%% Add bibliography

\chapter{Bibliographie}
\begin{multicols}{2}
\printbibliography[heading=subbibliography,segment=0,title={Préambule}]
\printbibliography[heading=subbibliography,segment=1,title={Introduction}]
\printbibliography[heading=subbibliography,segment=2,title={Impact, risques et mesures}]
\printbibliography[heading=subbibliography,segment=3,title={La relative diversité de la représentation des dommages}]
\printbibliography[heading=subbibliography,segment=4,title={Rendre visible : quantifier les choix éthiques}]
\printbibliography[heading=subbibliography,segment=5,title={Ethique de la modélisation intégrée}]
\printbibliography[heading=subbibliography,segment=6,title={Interpréter le modèle dans le monde réel}]
\end{multicols}

\chapter{Grille d'entretien}
\label{ch:grille}


\subsection*{Présentation de la recherche}
Je suis étudiant en géographie à l'École Normale Supérieure, à Paris. Je m'intéresse dans le cadre de mon mémoire aux modèles intégrés, et plus particulièrement aux fonctions de dommage, c'est-à-dire à la partie du modèle qui permet d'évaluer les impacts du climat.

Mes recherches portent sur les enjeux éthiques liés à la modélisation. Je réalise une série d'entretiens semi-directifs avec des acteurs de la modélisation et des utilisateurs de ces modèles. L'idée est de mieux comprendre comment les choix de modélisation sont perçus et interprétés par les acteurs qui utilisent des modèles.

J'aimerais vous poser des questions, conçues comme un guide pour cet entretien, qui peut néanmoins être très libre. L'objectif est de cerner votre perception et usage des modèles, ainsi que les questionnements que vous avez sur eux.

Si vous êtes d'accord, j'aimerais enregistrer l'entretien. Cet enregistrement est strictement confidentiel et sera détruit dès que les notes seront retranscrites.

\subsection*{Questions}
\subsubsection*{Contexte de l'enquêté}
\begin{itemize}
    \item \textbf{Activité actuelle}
    \begin{itemize}
        \item Quel est votre métier ?
        \item Depuis combien de temps le pratiquez-vous ?
        \item Travaillez-vous seul ou en équipe ?
        \item Si en équipe, vos collègues ont-ils des profils similaires au vôtre, ou au contraire différents ?
        \item Avez-vous l'occasion de discuter avec la communauté (par exemple, lors d'événements internationaux) ?
    \end{itemize}
    \item \textbf{Formation}
    \begin{itemize}
        \item Quelle est votre formation ?
        \item Avez-vous des compétences particulières en mathématiques, modélisation ou économie ?
    \end{itemize}
\end{itemize}

\subsubsection*{Connaissance générale des modèles intégrés}
\begin{itemize}
    \item \textbf{Sur le GIEC}
    \begin{itemize}
        \item Savez-vous ce qu'est le GIEC ?
        \item Quel rapport avez-vous avec le GIEC ?
        \item Avez-vous déjà contribué à des travaux pour le GIEC ?
        \item Avez-vous déjà lu un rapport du GIEC ? Si oui, quelle partie ?
        \item Quel type de rapport ? Pourquoi ?
    \end{itemize}
    \item \textbf{Sur les IAM}
    \begin{itemize}
        \item Savez-vous ce qu'est un modèle intégré ?
        \item Quels modèles intégrés connaissez-vous ?
        \item Savez-vous ce qu'est une fonction de dommage ?
        \item Selon vous, comment sont évalués les dommages climatiques ?
    \end{itemize}
    \item \textbf{Sur les résultats du GIEC}
    \begin{itemize}
        \item Avez-vous déjà vu ce graphique ou un graphique similaire ?
        \item Selon vous, comment a été produit ce graphique ?
    \end{itemize}
\end{itemize}

\subsubsection*{Pratique professionnelle des questions climatiques}
\begin{itemize}
    \item \textbf{Sur l'utilisation des modèles}
    \begin{itemize}
        \item Utilisez-vous des modèles intégrés ? 
        \begin{itemize}
            \item Utilisez-vous des modèles existants ou créez-vous vos propres modèles ?
            \item Utilisez-vous des modèles Open Source ?
            \item Utilisez-vous le modèle ou directement les résultats ?
        \end{itemize}
        \item Aimeriez-vous pouvoir le faire ?
        \item Le feriez-vous si vous aviez accès à des modèles plus simples ?
        \item Utilisez-vous des connaissances produites par des modèles intégrés ?
        \begin{itemize}
            \item Si oui, par quel moyen y avez-vous accès ?
            \item Si oui, quelles sont ces connaissances ?
        \end{itemize}
        \item Quelles sont les décisions pour lesquelles vous vous basez sur des modèles intégrés ?
    \end{itemize}
    \item \textbf{Sur la prise de décision}
    \begin{itemize}
        \item Avez-vous un mandat représentatif ?
        \item Êtes-vous amené à prendre des décisions ?
    \end{itemize}
\end{itemize}

\subsubsection*{Enjeux éthiques : équité et transparence}
\begin{itemize}
    \item \textbf{Technique}
    \begin{itemize}
        \item Comment gérez-vous l'incertitude dans les modèles que vous déployez ?
        \item Vous sentez-vous capable de comprendre les enjeux éthiques liés au modèle ?
        \item Êtes-vous capable d'adapter le modèle aux caractéristiques spécifiques de votre communauté ?
    \end{itemize}
    \item \textbf{Politique}
    \begin{itemize}
        \item Quelles décisions sont prises à l'aide d'outils de modélisation ?
        \item Utilisez-vous le modèle ou directement les résultats ?
        \item Quels outils de modélisation sont utilisés par vos équipes ?
        \item Que pensez-vous du concept d'équité intergénérationnelle ?
        \item Pensez-vous que c'est un concept dont il faudrait plus discuter ?
    \end{itemize}
    \item \textbf{Société civile}
    \begin{itemize}
        \item La modélisation est-elle assez transparente ?
        \item Faudrait-il qu'elle soit plus transparente ?
    \end{itemize}
    \item \textbf{Scientifique}
    \begin{itemize}
        \item Pensez-vous que les enjeux éthiques liés à la modélisation soient suffisamment pris en compte dans la communauté ?
        \item Jusqu'où vous pensez-vous légitime pour modéliser (par exemple, modéliser un événement climatique pour un économiste) ?
    \end{itemize}
\end{itemize}
\chapter{Fonctions de dommages recensées}

\input{.tex}

%\chapter{Liste des figures}
\listoffigures

%\chapter{Liste des tables}
\listoftables

%\listoffigures
%\listoftables

%\input{appendix/appendix-a}
%\input{appendix/appendix-b}
%\input{appendix/appendix-c} % Create file to add

\end{document}
