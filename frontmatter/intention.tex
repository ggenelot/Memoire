\chapter*{Note d'intention}
\addcontentsline{toc}{chapter}{Note d'intention}

Ce mémoire est l'aboutissement de 6 mois de réflexion, de tâtonnements et de questionnements. En réalité, c'est plutôt la continuité de nombreuses années d'études à ne pas savoir dans quelle case se ranger, ni à savoir par quelle fenêtre regarder. Il en ressort un travail varié : de nombreuses méthodes différentes, tout autant de questions larges, qui pourraient chacune faire l'objet d'un mémoire, voire d'une thèse, voire d'une carrière universitaire. \\

Cette possibilité de papillonner a été précieuse et centrale, autant dans ma formation que dans ce travail. Elle m'a permis de m'intéresser à beaucoup de sujets, et surtout à voir le même sujet avec beaucoup de regards. Elle rend le tout pas très lisible, et d'aucun diront que c'est un peu brouillon. C'est évidemment vrai : il y a dans ces pages de nombreux détours, des points qui sont ouverts sans jamais être refermés, et peut-être autant de manques, qui auraient pu être approfondis. C'est d'ailleurs souvent ce que l'on attend de la recherche et de ses hussards en blouse blanche : se spécialiser, pour maitriser toujours plus sa discipline, ses méthodes et ses questions. C'est d'ailleurs ce qui permet d'avoir une grande rigueur, nécessaire pour manipuler avec précision et parcimonie des outils théoriques très pointus. \\

J'ai tenté de suivre tout au long de ce mémoire une ligne directrice. Il s'agit des dommages liés au changement climatique, ou plus précisément de leur modélisation. Plus qu'une ligne directrice, c'est un objet. Un petit objet, même : une équation, une ligne de code, en tout cas quelque chose de très simple. On pourrait alors penser : quoi de plus simple de se spécialiser autour de ce petit objet, si petit qu'on peut aisément en dessiner les contours et en maitriser les enjeux dans le temps imparti pour un mémoire universitaire. C'est en partie vrai, mais surtout faux. Car cet objet est relié de toutes parts à tant d'autres, et pas des moindres : d'abord aux sciences du climat (rien que ça); puis à leur interaction avec les sociétés, et donc les sciences sociales et humaines (rien que ça), qu'il cherche à synthétiser en quelques termes (décidément). \\

J'ai donc essayé de suivre cette ligne directrice, toujours en ligne de mire, mais je n'ai pu résister à la tentation de faire des détours. Ce travail est donc une sorte de grand \emph{patchwork} d'idées, de littératures, de méthodes, qui s'articulent conjointement autour du même objet, avec, je l'espère, une forme de cohérence, et ce qu'il faut de rigueur.  \\

À travers ce mémoire, j'ai plongé dans un monde passionnant et qui m'était tout à fait inconnu. J'ai appris énormément de choses, de concepts, d'idées qui posaient autant de nouvelles questions et réflexions. L'objet de ce mémoire n'est pas tant d'offrir des réponses que d'ouvrir des portes, et d'accompagner mes amis et lecteurs, géographes, économistes, philosophes, sociologues, mathématiciens, ingénieurs et curieux dans toutes ces questions, en mettant à leur disposition les pistes que j'ai pu rassembler. J'espère que ce sera l'objet de nombreux échanges, et je suis convaincu que ce n'est que le début du voyage. 