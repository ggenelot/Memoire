\chapter*{Note d'intention}
\addcontentsline{toc}{chapter}{Note d'intention}

Ce mémoire est l'aboutissement de six mois de réflexion, de tâtonnements et de questionnements. En réalité, c'est plutôt une étape dans la continuité de nombreuses années d'études au cours desquelles j'ai eu du mal à trouver dans quelle case me ranger, et à définir par quelle fenêtre regarder. Le résultat est un travail varié : de nombreuses méthodes différentes, tout autant de questions foisonnantes, qui pourraient chacune faire l'objet d'un mémoire, voire d'une thèse, peut-être même d'une carrière universitaire. \\

Cette possibilité d'explorer et de rester ouvert à la sérendipité a été précieuse et centrale, autant dans ma formation que dans ce travail. Elle m'a offert l'opportunité de m'intéresser à beaucoup de sujets, et surtout d’observer le même sujet sous de nombreux angles. Cela a rendu l'ensemble parfois difficile à suivre, et certains diront que c'est un peu brouillon. C'est un constat que je reconnais : il y a dans ces pages de nombreux détours, des points qui sont ouverts sans jamais être refermés, et peut-être autant de manques, qui auraient pu être approfondis. C'est d'ailleurs souvent ce que l'on attend de la recherche et de ses hussards en blouse blanche : se spécialiser, pour maitriser toujours plus sa discipline, ses méthodes et ses questions. C'est d'ailleurs ce qui permet d'avoir une grande rigueur, nécessaire pour manipuler avec précision et parcimonie des outils théoriques très pointus. \\

Malgré cela, j'ai tenté de suivre tout au long de ce mémoire un fil conducteur. Il s'agit des dommages liés au changement climatique, ou plus précisément de leur modélisation. C'est un objet, un tout petit objet, même : une équation, une ligne de code, quelques lignes dans une documentation. On pourrait alors penser : quoi de plus simple de se spécialiser autour de ce petit objet, si petit qu'on peut aisément en dessiner les contours et en maitriser les enjeux dans le temps imparti pour un mémoire universitaire. C'est en partie vrai, mais surtout faux. Car cet objet est relié de toutes parts à tant d'autres, et pas des moindres : d'abord aux sciences du climat (rien que ça); puis à leur interaction avec les sociétés, et donc les sciences sociales et humaines (rien que ça), qu'il cherche à synthétiser en quelques termes (décidément). \\

J'ai donc choisi d'adopter ce fil conducteur. Ce travail est donc une sorte de grand \emph{patchwork} d'idées, de littératures, de méthodes, qui s'articulent autour du même objet, s'éclairent mutuellement et permettent, je l'espère, de répondre à ces questions essentielles avec un regard nouveau.  \\

À travers ce mémoire, j’ai plongé dans un monde fascinant et qui m'était tout à fait inconnu. J'ai appris énormément de choses, de concepts, et d’idées qui m'ont constamment nourri de nouvelles questions et réflexions. L’objectif de ce mémoire n’est pas tant d’offrir des réponses que d’ouvrir des portes, et d’accompagner mes amis et lecteurs — géographes, économistes, philosophes, sociologues, mathématiciens, ingénieurs et curieux — dans l'exploration de ces questions, en mettant à leur disposition les pistes que j’ai pu rassembler. Mon espoir est que cela suscite de nombreux échanges, et je suis convaincu que ce n’est qu’un début.