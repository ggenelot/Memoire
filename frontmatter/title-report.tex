\begin{titlepage}

\begin{center}

%% Print the title
{\makeatletter
\largetitlestyle\fontsize{45}{45}\selectfont\@title
\makeatother}

%% Print the subtitle
{\makeatletter
\ifdefvoid{\@subtitle}{}{\bigskip\titlestyle\fontsize{20}{20}\selectfont\@subtitle}
\makeatother}


\bigskip
\bigskip

%% Print the name of the author
{\makeatletter
\largetitlestyle\fontsize{25}{25}\selectfont\@author
\makeatother}

\bigskip

\href{mailto:gabriel.genelot@ens.psl.eu}{gabriel.genelot@ens.psl.eu}

\bigskip
\bigskip

%% Print table with names and student numbers
%\setlength\extrarowheight{2pt}
%\begin{tabular}{lc}
%    Student Name & Student Number \\\midrule
%    First Surname & 123456 \\
%\end{tabular}

\vfill

\textbf{Résumé} \\

\begin{center}
\justify


Ce compte-rendu est la restitution d'un travail d'exploration personnel réalisé dans le cadre du cours de Climatologie et Paléoclimatologie de la faculté de Sciences de Sorbonne Université. Il présente les résultats de plusieurs modélisations numériques qui simulent un système climatique dans des conditions pré-industrielles et un autre où la concentration en CO2 a été fortement augmentée. Deux modèles sont utilisés : le LMDz et le CMIP6. Il s'intéresse plus particulièrement aux précipitations annuelles moyennes et à la température annuelle moyenne en Amérique du Sud, et plus particulièrement dans la forêt amazonienne. On observe que les températures augmentent et que les précipitations diminuent quand la concentration en CO2 augmente. On observe également une modification des précipitations au niveau de la zone de convergence intertropicale, qui pourrait être signe d'un déplacement de celle-ci. On retrouve par ailleurs une dualité dans les réactions atmosphériques à une augmentation de la concentration en CO2. Celle-ci est évoquée dans la littérature, sans être résolue. 

\end{center}


\bigskip

\bigskip

%% Print some more information at the bottom
\begin{tabular}{ll}
    Enseignant : & Jean-Baptiste Madeleine \\
    Tuteur : &  Valentin Wiener\\
    Cadre : & Projet du cours de climatologie et paléoclimatologie \\
    Faculté : & Géosciences, Sorbonne Université Sciences
\end{tabular}

\bigskip
\bigskip

%% Add a source and description for the cover and optional attribution for the template
\begin{tabular}{p{15mm}p{10cm}}
    %Couverture : & Photo by Tom Fisk from \href{https://www.pexels.com/photo/green-forest-2739664/}{Pexels}  \\
    % Feel free to remove the following attribution, it is not required - still appreciated :-)
    %Style: & EPFL Report Style, with modifications by Batuhan Faik Derinbay
\end{tabular}

\end{center}

%% Insert the EPFL logo at the bottom of the page
\begin{tikzpicture}[remember picture, overlay]
    \node[above=10mm] at (current page.south) {%
         \includegraphics[width=0.5\linewidth]{figures/logos/iedes_logo.jpg}
    };
\end{tikzpicture}

\end{titlepage}
