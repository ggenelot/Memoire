\chapter*{Remerciements}
\addcontentsline{toc}{chapter}{Remerciements}

% Encadrement

Je voudrais commencer ce manuscrit en remerciant celles et ceux qui m'ont accompagné dans ce voyage initiatique à la recherche. Tout d'abord, je suis très reconnaissant envers Nadia Maïzi, directrice du Centre de Mathématiques Appliquées, qui m'en a ouvert les portes avec toute sa confiance, son exigence et sa bienveillance. Ces rendes-vous réguliers ont été autant de jalons où j'ai été impressionné par la précision des réflexions, la richesse des échanges et la qualité des conseils. Cette présence discrète et rassurante m'a permis de grandir intellectuellement et méthodologiquement; elle est rare, et je suis heureux de l'avoir trouvée ici. Je voudrais bien sûr remercier Thibaut Feix, qui m'a encadré au quotidien et avec qui j'ai partagé un bureau, des doutes et des questions. Merci de m'avoir proposé ce sujet, qui s'est révélé passionnant, et merci croire en une science plus humaine et juste. \\


% Encadrement plus lointain, inspirations

Je voudrais également remercier toutes celles et ceux qui m'ont encadré de près ou de loin, répondu à des questions ou montré des voix. Je pense bien sûr à Béatrice Cointe, qui m'a aidé à formuler des questions, à Mathias Girel, grâce à qui j'ai osé m'approprier les outils essentiels mais impressionnants de l'épistémologie, Marc Fleurbeay, qui inspire mon chemin d'étudiant en économie, Alessandra Giannini, dont la confiance m'a permis de voir de plus près les rouages du régime climatique. Je veux aussi exprimer ma reconnaissance auprès de tous les autres que je ne peux pas citer, qui par une remarque, une question ou un échange, ont fait avancer ma réflexion. \\ 

% Labo

Mais cette aventure n'est pas qu'intellectuelle. Aussi, je voudrais remercier Claire, Lucas, Alice, Victor, Grégorio, Sophie, Marie, et tous les autres, pour m'avoir accueilli à Sophia-Antipolis, et surtout pour faire résonner les couloirs du CMA d'éclats de rire et rayonner de bonne humeur. \\

% Proches

Enfin, je voudrais remercier mes proches, celles et ceux qui ont partagé des heures de bibliothèque et m'ont permis, par la diversité et la richesse de leurs points de vue, de louvoyer entre les disciplines pour trouver ma voie. Une pensée toute spéciale va à mes grands-parents, qui m'ont toujours guidé par leur curiosité. 
