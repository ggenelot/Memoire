\chapter*{Remerciements}
\addcontentsline{toc}{chapter}{Remerciements}

% Encadrement

en exprimant ma profonde gratitude à toutes celles et ceux qui m’ont accompagné dans ce voyage initiatique à la recherche. En premier lieu, je tiens à remercier Nadia Maïzi, directrice du Centre de Mathématiques Appliquées, qui m'en a ouvert les portes avec confiance et bienveillance. Les rendez-vous réguliers que nous avons eus ont été autant de jalons marquants où j’ai été constamment impressionné par la finesse de ses réflexions, la richesse de nos échanges, et la qualité de ses conseils.  Cette présence rassurante, tout en étant exigeante, m'a permis de grandir intellectuellement et méthodologiquement; elle est rare, et je suis heureux de l'avoir trouvée ici. e souhaite également remercier Thibaut Feix, qui m’a encadré au quotidien, partageant avec moi un bureau, des doutes, et de nombreuses questions. Merci de m’avoir proposé ce sujet, qui s’est révélé profondément passionnant, et merci de faire des ponts de curiosité entre des disciplines si souvent cloisonnées. \\


% Encadrement plus lointain, inspirations

Je tiens également à exprimer ma reconnaissance à toutes celles et ceux qui m’ont accompagné, de près ou de loin, en répondant à mes questions ou en m'indiquant de nouvelles voies à explorer. Je pense, bien sûr, à Béatrice Cointe, qui m’a aidé à formuler des questions fondamentales, à Mathias Girel, grâce à qui j’ai osé m’approprier les outils, parfois impressionnants, de l’épistémologie, à Marc Fleurbaey, dont le parcours continue d’inspirer ma démarche en tant qu’étudiant en économie, et à Alessandra Giannini, qui, par sa confiance, m’a permis de découvrir de plus près les rouages du régime climatique. Je veux aussi adresser ma gratitude à tous les autres que je ne peux pas citer ici, mais qui, par une remarque, une question, ou un simple échange, ont contribué à faire avancer ma réflexion. \\

% Labo

Cette aventure n’a pas été qu’intellectuelle. Je tiens à remercier chaleureusement Claire, Lucas, Alice, Victor, Grégorio, Sophie, Marie, et tous les autres, qui m'ont accueilli à bras ouverts à Sophia-Antipolis, et qui, surtout, ont fait résonner les couloirs du CMA de rires et de bonne humeur, et fait rayonner les pentes de Valbonne. \\

% Proches

Enfin, je voudrais remercier mes proches, celles et ceux qui ont partagé des heures de bibliothèque et m'ont permis, par la diversité et la richesse de leurs points de vue, de louvoyer entre les disciplines pour trouver ma voie.  Je pense tout particulièrement à mes grands-parents, qui, par leur inlassable curiosité, m'ont toujours inspiré et guidé.
