\chapter{Conclusion}
\label{chapter:conclusion}
\newrefsegment


% Rappel de l'objectif

%La problématique de ce mémoire est partie d'une première intuition, en voyant la fonction de dommage de DICE (équation \ref{eq:df_dice2023}) : celle que cette forme était absurdement simple et simpliste, et qu'on ne pouvait pas (ou même, ne devait pas) simplifier autant de paramètres, de variables, dans une formule semblable à une formule magique. Cette 

La question de recherche de ce mémoire est née de la confrontation avec la fonction de dommage de DICE (équation \ref{eq:df_dice2023}). Une question est née tout de suite : comment peut-on représenter des phénomènes aussi complexe que les dommages climatiques avec une fonction aussi simple, qui ressemble presque à une formule magique ? Viennent des conclusions hâtives, des intuitions : non, il n'est pas possible de simplifier autant, ou en tout cas ce n'est pas souhaitable, ou alors irresponsable. Pourtant, en tirant petit à petit le fil des questions, il apparait que cette approche est porteuse de sens. Revenons ici sur le fil du raisonnement, avant de tirer quelques conclusions. 


% Synthèse des chapitres

\section{Faut-il modéliser les dommages ?}

Nous avons d'abord cherché à voir les différentes formes de fonctions de dommage. La diversité avec laquelle les dommages sont représentés est importante, et croissante. De nombreuses voix s'élèvent pour remettre en question des formes de fonction de dommage top-down comme celle de DICE. Dans le même temps, celle-ci sont toujours très majoritaires, et ont inspiré de nombreuses variantes. Pourtant, elles restent souvent uniquement monétaires, ce qui risque d'exclure d'autres phénomènes du champ des représentations. Des tentatives existent de représenter les tipping points, les effets sur l'énergie, la biodiversité ou encore la productivité, mais nous n'avons pas eu le temps de les modéliser. \\

Nous avons ensuite cherché à mesurer l'effet d'hypothèses éthique, à travers l'exemple de l'équité spatiale. Il était clair que les variables physiques jouent un rôle essentiel dans le niveau de dommage. Par ailleurs, une partie importante de la littérature est consacrée à des comparaisons inter-modèles, et il est clairement établi que différents modèles auront des résultats différents, même à paramétrisation identique. Enfin, le rôle de choix éthiques est abondamment discuté, comme celui du taux d'actualisation. Notre apport ici est double. D'abord, d'avoir étendu la discussion à d'autres dimensions que l'équité temporelle, en prenant en compte l'équité spatiale. Ensuite, en quantifiant la variabilité du niveau de dommage expliquée par chacune de ces sphères. Nos résultats montrent que les considérations éthiques jouent un rôle important dans la quantification des dommages, dont l'amplitude est comparable aux (ou du moins, non négligeable devant) les variables physiques et méthodologiques.   \\

Nous avons ensuite replacé ces choix dans un contexte épistémologique qui permet de les interpréter à travers un prisme éthique. En effet, nous avons cherché à discuter le rôle des modélisateurs au prisme de cette importance des variables éthiques à l'aide d'auteurs issus de l'épistémologie. Il nous est apparu trois choses. D'abord, que les considérations éthiques liées à la modélisation dépassaient largement le fait de la \emph{bonne science}. À travers les concepts d'éthique intrinsèque et extrinsèque, nous avançons que le modélisateur à une influence bien plus large que celle du laboratoire. Ensuite, nous avons développé l'idée que ces choix façonnent les futurs possibles et cadre les discussions du régime climatique. Enfin, nous avons conclu que les modélisateurs avaient une forme de responsabilité quant aux conséquences de leurs modèles, sans que les divergences puissent être qualifiées de doutes normativement inappropriés. \\

Nous avons finalement interrogé ce lien entre la science et les utilisateurs finaux du modèle. En effet, un discours revient souvent au sein de la communauté des modélisateurs. Celle-ci ne serait pas responsable de mauvaises interprétations qui serait faite des modèles, et notamment d'une méconnaissance des hypothèses et des limites qu'elles posent. Au regard de la section précédente, une telle position semble difficilement tenable. En effet, elle impliquerait que tous les utilisateurs de modèles, c'est-à-dire à la fois les modélisateurs et les personnes amenées à prendre des décisions sur la base de ceux-ci, en aient une compréhension fine. Nous avons réalisé des entretiens semi-directifs avec des acteurs du régime climatique (modélisateurs, décideurs, journalistes, activistes) pour tester cette hypothèse de manière qualitative. Il en ressort que malgré les efforts de transparence largement déployés, l'effet \emph{boite noire} du modèle persiste. De ce fait, l'ensemble des acteurs sont dépendants de choix fait en amont, dans la modélisation. Une alternative peut être l'utilisation de modèles plus petits, construits pour une question spécifique en collaboration avec des acteurs non-experts.  Par ailleurs, il apparait que le rôle de la modélisation dans la décision publique est parfois plus faible qu'il n'y parait : plutôt que de construire celle-ci à partir de la compréhension offerte par les modèles, elle est justifiée par des résultats correspondant à une politique antérieurement décidée. \\

Il ressort de ces réflexions que la perspective d'une modélisation neutre ou objective semble être un mirage. Bien au contraire, le modèle est le reflet d'hypothèses implicites, de normes et de valeurs. Celles-ci sont transposées au sein du modèle, dont les résultats sont le reflet. Il serait bien hâtif d'en conclure que les modèles en sont inutiles. En effet, comme l'indique Hélène Longino, si l'indépendance de la science de valeurs non épistémiques est un leurre, il n'en reste pas moins une possibilité d'autonomie. \\

\section{Implications pratiques}

On peut ainsi rejeter deux comportements qui pourraient découler de ces conclusions. \\

D'une part, il ne s'agit pas de rejeter toute forme de modélisation des dommages. À l'heure où ceux-ci sont de plus en plus nombreux, où les choix d'adaptation actuels nous engagent pour des décennies, où l'action climatique peine à trouver ses marques et où d'aucun appellent à la rigueur budgétaire, il est essentiel de proposer des lectures alternatives. Modéliser des dommages permet de voir les dépenses climatiques comme un investissement d'avenir; d'orienter les politiques d'adaptation, d'aménagement du territoire; de mettre en évidence les inégalités et injustices que ces dommages vont générer ou exacerber, et ainsi permettre une transition plus juste. Malgré les (nombreuses) limites de ce genre de modélisation, la représentation des dommages est une alliée dans le contexte incertain qui nous entoure. \\

D'autre part, il ne s'agit pas non plus de chercher une fonction de dommage parfaite, qui permettrait de représenter le système terrer avec une précision infinie. À l'inverse, chaque fonction de dommage a ses atouts et ses contraintes, peut être déployée dans un domaine particulier pour représenter certains phénomènes. Un des apports que nous avons tenté de faire avec ce travail est de montrer la diversité des fonctions de dommage, non seulement pour l'étudier, mais aussi pour que chacun puisse choisir une forme adaptée à ses besoins et à son discours. Si la modélisation est un outil de discours, alors autant en étendre le vocabulaire. 

\section{Perspectives futures}

D'abord, approfondir les pistes qui ont été ouvertes ici. On peut par exemple chercher à compléter la base de donnée des fonctions de dommages; en implémenter plus dans WILIAM, pour étendre la comparaison avec d'autres fonctions de dommage; implémenter un pondérateur qui reflète l'équité sociale, comme dans \textcite{dennig_inequality_2015}, ou encore un taux d'actualisation. Ces apports permettraient de pouvoir élargir la comparaison développé ici à d'autres variables. Ensuite, on pourrait  approfondir la réflexion autour de l'éthique des fonctions de dommage, notamment en lien avec les travaux de \textcite{pacchetti_for_2024}. Enfin, cet approfondissement pourrait se faire en lien avec une pratique artistique, comme c'est le cas dans la thèse de \textcite{van_beek_persuasive_2023}. Cette thèse avait été accompagnée d'une résidence d'artiste, qui a abouti à la production d'un manuel de modélisation \autocite{noauthor_future_nodate}. \\

%Ce travail peut être vu comme un point de départ, une base à discuter et raffiner. De ce point de vue, et c'est peut

