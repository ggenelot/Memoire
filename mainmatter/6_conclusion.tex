\chapter{Conclusion}
\label{chapter:conclusion}
\newrefsegment


% Rappel de l'objectif

La problématique de ce mémoire est partie d'une première intuition, en voyant la fonction de dommage de DICE (équation \ref{eq:df_dice2023}) : celle que cette forme était absurdement simple et simpliste, et qu'on ne pouvait pas (ou même, ne devait pas) simplifier autant de paramètres, de variables, dans une formule semblale à une formule magique. Cette 




% Synthèse des chapitres

Nous avons d'abord cherché à voir les différentes formes de fonctions de dommage. \\

Nous avons ensuite cherché à mesurer l'effet d'hypothèses éthique, à travers l'exemple de l'équité spatiale. \\

Nous avons ensuite replacé ces choix dans un contexte épistémologique qui permet de les interpréter à travers un prisme éthique. \\

Nous avons finalement interrogé ce lien entre la science et les utilisateurs finaux du modèle. \\



% Reflexions ethiques 


% limites 

% implications pratiques 


% perspectives futures

% reflexions générales