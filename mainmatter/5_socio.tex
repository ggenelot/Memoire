\chapter{Interpréter le modèle dans le monde réel}
\label{chapter:socio}
\newrefsegment

\PEEL{Il est nécessaire de diversifier les représentations des dommages, mais aussi de prendre en compte les dommages non monétaires. Les decideureuses n'ont pas, n'auront pas et n'ont pas vocation à avoir de connaissances ou de compétences techniques.}{Fournissez des preuves, des données ou des citations qui soutiennent votre point.}{Expliquez en quoi les preuves que vous avez fournies sont pertinentes et comment elles appuient votre point.}{Il sont donc tributaires des modélisateur.ices pour l'interprétation des modèles.}

\begin{displayquote}
    It is the developer’s firm belief that most researchers should be locked away in an ivory tower. Models are often quite useless in unexperienced hands, and sometimes misleading. No one is smart enough to master in a short period what took someone else years to develop. Not-understood models are irrelevant, half-understood models treacherous, and mis-understood models dangerous\footnote{Le modélisateur a la conviction profonde que la plupart des chercheurs devraient être enfermés dans des tours d'ivoires. Les modèles sont souvent inutiles dans des mains inexperimentées, et parfois trompeurs. Personne n'est assez malin pour maitriser dans une courte période ce qu'une autre a développé en plusieurs années. Pas compris, les modèles ne sont pas pertinents; à moitié compris, il sont traitres; mal compris, ils sont dangereux.}. \\
    
    Richard Tol et David Anthoff, documentation du modèle FUND
\end{displayquote}
    
\chapterabstract{
 Cette partie du mémoire s'intéresse à la manière dont les résultats issus de la modélisation intégrée sont interprétés dans le débat public et permettent de prendre des décisions. Plus précisement, on s'interroge sur les utilisations des modèles par divers acteurs de la politique climatique : techniciens, décideurs, journalistes, scientifiques, activistes.  Pour cela, on réalise une série d'entretiens semis-directifs. Ils ont pour but de répondre aux questions suivantes : comment est interprétée l'incertitutde inhérente aux modèles pour la prise de décision ? 
}

Comme nous l'avons vu dans les chapitres précédents, les hypothèses jouent un rôle fondamental et déterminant dans le fonctionnement du modèle et les résultats qu'il propose. Ainsi, utiliser le modèle dans un cadre adapté requiert d'avoir une compréhension fine de ces hypothèses et de ces choix de modélisation. Cette compréhension permet de mieux articuler les différents enjeux éthiques qui sont à l'œuvre dans le processus. 

Une des limitations principales à cette compréhension est la présence de communauté distinctes, mues par des objectifs différents, et ayant des caractéristiques différentes. On définit ici la notion d'utilisateur final du modèle, qui désigne les personnes amenées à utiliser, sous une forme ou sous une autre, un modèle, ses résultats, ou les interprétations de ces résultats. Cette catégorie est extrêmement large, et regroupe entre autres les administrations publiques, les décideurs politiques, les journalistes, les membres du milieu associatif. Si ces profils sont très divers, ils se rejoignent sur deux points. D'abord, il s'agit d'individu ayant des actions liées au changement climatique, prenant et / ou influençant des décisions politiques. Ensuite, ce ne sont pas des experts de la modélisation, en ce sens qu'ils ne sont pas à l'origine de modèles. On a donc deux communautés distinctes : les modèles sont conçus et maitrisé au sein de la communauté des modélisateurs, mais leurs résultats et interprétations sont utilisés et converti en actions au par les utilisateurs finaux des modèles. 

Les membres de cette seconde partie sont ceux qui vont animer le modèle, et conduire ses conclusions dans le monde réel. D'une certaine manière, il s'agit d'une des étapes les plus importantes, ou en tout cas les plus politiques, puisqu'il s'agit d'ancrer les hypothèses normatives dans le monde réel, ce qui en fait dès lors des positions politiques. On sort alors du pur cadre théorique et conceptuel pour rejoindre celui de l'action publique. Ce passage est sensible, puisque les biais contenus dans la partie théorique deviennent autant d'inégalités ou d'injustices : de limitations scientifiques, ils deviennent de véritables choix de sociétés.

Ce raisonnement nous amène à la question suivante : dans quelle mesure les utilisateurs finaux de modèles sont en mesure de comprendre les modèles et leurs implications ? 

Nous allons répondre à cette question de manière qualitative (voir encadré méthodologique), à travers des entretiens semi-directifs avec des acteurs de la modélisation ou des utilisateurs finaux des modèles. Nous commencerons par une contextualisation théorique, puis nous détaillerons les résultats des entretiens à travers ces trois axes : d'abord, les modèles intégrés sont trop complexes pour être manipulés par des publics non avertis; ensuite, il existe d'autres méthodes de modélisation, telle que la modélisation participative, qui permettent de co-concevoir les modèles; enfin, les décideurs politiques ont un besoin fort de données quantitatives. 




\begin{methodbox}[Entretiens semi-directifs]

On réalise des entretiens semi-directifs avec des acteurs du débat public sur les questions climatiques. \\

Cette catégorie est volontairement large. On classe ensuite les enquêtés en quatre catégories d'acteurs : les scientifiques, les techniciens, les politiques et la société civile. Les \textbf{scientifiques} désignent les acteurs dont la parole est reconnue comme porteuse d'un message scientifique. Il s'agit généralement de modélisateurs, et parfois d'acteurs gravitant autour des milieux universitaires : comité d'éthique, communication scientifique non-vulgarisée. Les \textbf{techniciens} sont les acteurs d'administrations publiques n'ayant pas de mandats électifs. Il s'agit le plus souvent de spécialistes de sujets spécifiques, qui transcrivent les connaissances climatiques en plan d'action ou en texte réglementaires. Les \textbf{politiques} sont tous les acteurs qui sont dotés d'un mandat électif. Leur spécificité est d'être amené à prendre des décisions, à trancher dans des contextes où les conséquences des différentes actions sont soit empreinte d'incertitude soit de choix moraux. Enfin, la \textbf{société civile} désigne les acteurs non-institutionnels qui s'emparent de sujets climatiques. Il s'agit en particulier de journalistes, mais aussi d'activistes ou de personnes engagées dans le monde associatif. \\

Les entretiens sont du format semi-directif. Une grille d'entretien est soumise aux enquêtés (voir annexe \ref{ch:grille}). Cependant, les réponses sont longues et libres, et peuvent donner lieu à des questions inédites. Symétriquement, toutes les questions ne sont pas traitées dans tous les entretiens. Ces entretiens sont ensuite retranscrits, et des passages sont sélectionnés. \\

Notre échantillon se compose de 7 enquêtés : 3 personnes appartiennent à la communauté des scientifiques, 2 à la société civile, 2 techniciens. Plusieurs personnes appartenant à la catégorie \textit{politique} ont décommandé leur entretien en raison de la situation politique en France. Les entretiens ont duré en moyenne une heure, dans une grande flexibilité vis-à-vis de la grille de question. \\

Pour des raisons de temps, il n'a pas été possible de réaliser plus d'entretiens. Il faut donc prendre avec précaution ces conclusions, qui ne reflètent qu'une petite partie des perceptions qu'ont les acteurs de la modélisation des dommages. 

\end{methodbox}


%Beaucoup de remarques issues de \cite{intergovernmental_panel_on_climate_change_ipcc_mitigation_2023}



\section{Introduction, méthodologie et questions}

Le rôle des modèles dans le débat public est un sujet largement abordé dans la littérature de la sociologie de la modélisation. Dans les années 1970, les différentes versions du modèle World popularisent l'utilisation de systèmes dynamiques pour représenter le monde dans une approche systémique. Ils représentent le fonctionnement du monde autour de cinq grandes familles de variables, interdépendantes, et aboutissent à la conclusion qu'une croissance infinie n'est pas soutenable \cite{forrester_world_1971}.  Ces modèles, et en particulier World3, donnent naissance au rapport dit Meadows, ou encore rapport au Club de Rome, intitulé \emph{limits to growth} \cite{meadows_limits_1972}. Ce rapport a eu un fort retentissement politique et médiatique, et a contribué à la popularisation de l'idée que la croissance économique pure n'est ni souhaitable, ni soutenable \cite{edwards_global_1996}. 
Nous explorerons ici les liens qui se tissent entre la fabrique des modèles et leur utilisation, d'abord dans une perspective générale, puis dans le cas plus spécifique des fonctions de dommage. Nous questionnons ainsi le modèle linéaire de la science, développé par \cite{aykut_gouverner_nodate} \ref{model-lineaire}. Notre question de recherche est la suivante : comment les enjeux éthiques liés à la modélisation des dommages sont-ils perçus et interprétés par les utilisateurs finaux des modèles ? 

\subsection{Le rôle des modèles et des scenarios dans la prise de décisions}

Depuis les années 1990, les scénarios socio-économiques d'émissions jouent un rôle central dans les rapports du GIEC. Ils sont ainsi au cœur de ce qui est aussi qualifié de \emph{policy-relevant knowledgde}. Ils sont désormais produits de manière quasiment exclusive par les modèles intégrés \cite{cointe_organising_2019}, avec la volonté explicite de servir à la production de politique publique \cite{Weyant}. \\



%\cite{cointe_organising_2019} => le rôle des IAMs dans la prise de décision

Dès leur conception, les modèles intégrés sont orientés vers de la prise de décision. Ils sont plus simples que les modèles physiques tels que les modèles couplés, et ont pour vocation de fournir une intuition des effets des politiques climatiques. En ce sens, ils constituent un outil heuristique destiné à être utilisé soit directement par les administrations, soit au plus proche des administrations. Ils produisent des évaluations, et permettent des comparaisons, et non des prédictions \cite{edwards_global_1996}. 

\begin{displayquote}
“whether or not they are ever used directly by policymakers, these models are contributing to what I believe to be a fundamental shift in the structure of scientific work toward trans-disciplinary collaboration and communication. This means that ESMs and IAMs in fact contribute substantially to the basis of global change politics, in the important sense that they serve as one of the organizing principles of a large, growing, epistemologically coherent community.” (Edwards, 1996, p. 152)
\end{displayquote}


\cite{wynne_institutional_1984} montre que l'évaluation des modèles, y compris par les pairs, est difficile. Il développe l'exemple de l'IIASA Energy Study. Celle-ci a fait l'objet d'une attention particulière quant à la cohérence interne. Cette rigueur affichée a été perçue à l'extérieur, y compris par les pairs, comme signe d'une grande précision et objectivité du modèle.  Bien que cet exemple soit ancien, il montre comment la perception des résultats influence le regard critique qui est porté sur le modèle. 

\subsection{L'implication des choix de modélisation sur la décision}

Nous avons développé plus haut le concept de cadrage, ou de framing. Un exemple intéressant de comment ce cadrage se retranscrit dans le discours public est celui de l'absence de scénario de décroissance. \cite{cointe_understanding_2023} montrent que dans les 1,071 scenarios utilisés pour produire le 5ème rapport du GIEC (AR5) dans lesquels un taux de croissance annuel était calculable, tous avaient un taux de croissance supérieur à zéro. 
Il existe pourtant des manières de modéliser la décroissance. \cite{briens_decroissance_2015} en a fait l'objet de sa thèse. A travers différentes techniques, il propose d'explorer, dans une perspective de prospective, des scénarios de décroissance. 

Comme décrit plus haut, le rôle du GIEC est précisément de faire la synthèse des connaissances disponibles dans un format qui soit utile à la prise de décision. Exclure les scénarios de décroissance revient dès lors à les retirer de la connaissance valide, mais aussi des options de politique publique possibles. Ainsi, la variété des scénarios produits et pris en compte a des répercussions directes sur la nature des options de politique publiques envisagées. 

\subsection{La diffusion de connaissance auprès des utilisateurs finaux}

Pour les décideurs, choisir un modèle est une activité difficile. \cite{boulanger_models_2005} identifient cinq critères qui permettent d'évaluer la pertinence d'un modèle pour une politique publique : l'approche interdisciplinaire, le traitement de l'incertitude, la perspective à long-terme, la perspective global-local, et la participation. Nous reviendrons sur la notion de participation un peu plus loin. Cette approche illustre comment la difficulté d'accès pour des décideurs aux modèles intégrés. Ils doivent en effet choisir entre des options nombreuses, qui sont semblables sur de nombreux points, mais différentes sur d'autres. Ces différences, comme nous l'avons vu, peuvent parfois sembler très faible, mais peuvent avoir des conséquences importantes sur le comportement du modèle. \\


Pour répondre à cette difficulté, des auteurs proposent de créer des services scientifiques spécifiques, dont le rôle est précisement d'avoir une compétence technique au service de questions politiques. C'est le cas de \cite{auer_climate_2021}, qui proposent de créer un service des scénarios du changement climatique. Un tel service aurait pour mission de \emph{permettre à une communauté large de décideurs, acteurs économiques, financiers et régionaux d'avoir accès et d'utiliser des scénarios de changement climatique de pointe et de manière intéressante}. Il ferait le lien entre la communauté scientifique et les besoins des utilisateurs, en éclairant les questions de ces derniers au regard des possibilités et limites des modèles. \\

Un décalage qui crée de la friction entre les communautés scientifiques et le grand public est la gestion de l'incertitude. \cite{shackley_representing_1996} montre que cette tension vient d'une part de l'omniprésence de l'incertitude dans la production scientifique, et d'autre part de la demande de la part du public pour une science fiable et certaine (ou du moins perçu comme telle). En effet, l'incertitude est inhérente à la recherche, et particulièrement dans le cadre de la recherche autour du changement climatique. Elle fait partie des sujets régulièrement discutés au sein de la communauté. En revanche, exposée au grand jour, elle met en question la crédibilité de la production scientifique. Il y a dès lors un dilemme lors de la diffusion de cette incertitude : masquer cette incertitude pour avancer des résultats, qui par ailleurs peuvent être reconnus et consensuels malgré l'incertitude résiduelle; ou afficher cette incertitude, en accord avec les principes d'intégrité et de transparence, au risque que les résultats soient perçus pour moins fiables que ce qu'ils ne sont, et d'une perte de confiance ? 
Cette tension inhérente à la science est particulièrement intéressante dans le cas des fonctions de dommage. En effet, celles-ci sont très sensibles au traitement des incertitudes. Les exposer pleinement décrédibilise les modèles intégrés, bien que ce soient des outils d'aide à la décision essentiel et partiellement valable malgré ces incertitudes. Les masquer revient à imposer une vision et une interprétation. 



\subsection{La formation d'une communauté des modélisateur}

La complexité des modèle est un frein au sein même de la communauté. En effet, le nombre important de variables et d'équations, et l'immense littérature relative aux phénomènes que cherchent à modéliser les modèles intégrés font que chaque modèle doit être approprié pour lui même. Ainsi, les modélisateurs ne sont souvent pas familier d'autres modèles intégrés, pourtant proches les uns des autres. Pour faire face à une communauté fractionnée, deux initiatives ont vu le jour. \\

D'abord, l'\emph{Integrated Assesment Modeling Consortium (IAMC)}. Ce consortium a été créé en 2007 pour \emph{coordonner la production de RCP pour le GIEC} \cite{cointe_organising_2019}. A travers des colloques, le consortium vise à faciliter les échanges de point de vue, de documentation, et à coordonner les différents projets de modélisation intégrée. Nous avons d'ailleurs, dans la partie \ref{chapter:litrev}, beaucoup utilisé les données issues du wiki de l'IAMC \cite{noauthor_models_nodate}. Il est cependant important de noter que ces informations étaient souvent très lacunaires (des informations très vagues ou inexistantes) voir fausse (des équipes dont les modèles étaient indiquées comme ayant des fonctions de dommage ont été contactées, et nous ont bien confirmé l'absence de fonction de dommage dans ces modèles). \\

Ensuite, des \emph{Model Intercomparison Projects (MIPs)}. \\




%\cite{keppo_exploring_2021} => sur les limitations des IAMs liées à la communication à ce sujet

%\cite{davidson_climate_nodate} => sur l'intérêt de penser au pire scénario climatique pour limiter les dégâts de la potentielle crise


%\cite{dekker_consensus_2022} => sur les bases de données IIASA
  
%D'autres outils : la modélisation participative : 
%\cite{etienne_modelisation_2010}






\section{Résultats}

\subsection{La difficile mais nécessaire transparence}

Face à l'importance des dimensions éthiques et normatives des modèles intégrés, une attention de plus en plus soutenue a été accordée à la transparence.  Si c'est désormais un objectif affiché, c'est tout de même un véritable défi, comme nous le confie un modélisateur : 

\begin{displayquote}
    What is transparency ? Tranparency is not easy. One equation is easy to understand, 50 equations is difficult. 
\end{displayquote}
Cette remarque met en évidence une des difficultés principale du lien modélisateur - utilisateur, dont le renforcement est pourtant un des objectifs de la transparence. \cite{keppo_exploring_2021} soulignent que les hypothèses centrales des modèles sont souvent mise en avant, mais que chaque module du modèle est influencé  aussi par des hypothèses faites ailleurs dans le modèle. \\

De la même manière, le fait d'ouvrir le code et le modèle est une étape bienvenue pour augmenter la transparence; elle n'est pourtant pas suffisante. \cite{keppo_exploring_2021} avancent que peu de gens sont capables de comprendre et de faire tourner le programme, même lorsque le code est ouvert. Par ailleurs, \emph{implications of specific assumptions only become clear when one understands the model well}, ce qui maintient une grande barrière à la compréhension, quand bien même le modèle est ouvert. 

Ainsi, si l'ouverture du modèle fait sauter certaines barrières techniques, elle ne permet pas pour autant d'en diminuer tous les limites. 

\begin{displayquote}
    include lo
\end{displayquote}
Cependant, même si l'audience n'est pas très large, la transparence reste un atout. 

\begin{displayquote}
Our wiki is intented for two objective: internal, we use it; external, it's open for modellers. If any one wants to download the model and test it, they can. 
\end{displayquote}

\subsection{Seuls les modélisateur.ices comprennent leurs modèles (et encore)}

\textit{Une incompréhension de la part du grand public
=> exemple de journaliste ; SGPE; }

Les modèles sont difficiles à comprendre pour le grand public. La première barrière est souvent l'accès à la connaissance, dans le sens de faire la démarche d'accéder à la connaissance. 
Par exemple, lors d'un entretien avec une haute fonctionnaire d'une administration française, celle-ci reconnait ne pas se référer régulièrement aux rapports du GIEC, car ceux-ci sont trop volumineux. Par extension, elle ne cherche pas à avoir accès aux connaissances antérieures, et notamment aux modèles, principalement par manque de temps. On peut néanmoins remarque que c'est précisément pour synthétiser une connaissance variée et dispersée qu'est né le GIEC. \\

\textit{=> y compris chez les personnes qui ont déjà une certaine connaissance : activiste} \\


On constate le même raisonnement chez des personnes qui ont une pratique antérieure de la modélisation, de l'informatique ou des mathématiques par leur formation. Par exemple, une enquêtée active dans une association de plaidaoyer climatique déclare ne jamais avoir ouvert un modèle, alors même qu'elle en a les capacité technique et que le modèle est accessible en open-source. 

\begin{displayquote}
    Ajouter la quote de activiste
\end{displayquote}

Les limitations d'accès aux modèles sont lourdes, et dépassent des contraintes uniquement matérielles ou capacitaires. \\

\textit{Pas forcément d'envie non plus
=> journaliste ne souhaite pas franchement savoir; au contraire, c'est plutot pour lui le rôle du journaliste de connaitre ses sources} \\

Un enquêté journaliste apporte un autre point de vue sur la question. Il n'a lui non plus pas consulté directement les modèles, mais se base sur les rapports et autres analyses tirés de ces modèles, tels que les rapports du GIEC. Pour lui, cette limitation technique est inévitable, et le rôle du journalisme est précisément de choisir ses sources pour s'assurer qu'elles sont fiables. 

\begin{displayquote}
    Citation de journaliste
\end{displayquote}

Malgré les tentatives de transparence, il y aurait donc une part irréductible à la distance entre le modèle et ses utilisateurs finaux. 

\begin{displayquote}
    En fait, ce qui manque pour moi dans ces modèles, c'est des sciences sociales. C'est à dire qu'on a des modèles qui sont conçus par des gens qui sont très intelligents, qui sont des ingés super forts, qui sont capables de tout calculer et qui veulent en plus assez souvent à la fin qu'on est zéro au bout de la ligne et que le truc retombe d'une manière à peu près tranquille. Ça en fait du coup des modèles qui sont très, très théoriques dans la manière dont les choses se passent, dans l'économie et la société, la transition, elles ne se passent pas comme les modèles le dessinent pour plein de raisons.
\end{displayquote}
Par ailleurs, cette séparation entre la modélisation et l'interprétation est parfois perçue comme souhaitable. 
\begin{displayquote}
    C'est-à-dire le travail en amont, c'est le travail des modélisateurs. Ça, je pense que c'est à eux aussi d'être capables de dire, on rassemble les parties prenantes pour que nos présupposés, à nous, nos biais soient confrontés à d'autres biais et qu'ensuite on trouve le meilleur équilibrage de la manière la plus sincère possible. Évidemment, c'est difficile, c'est jamais parfait, mais qui brasse le plus de monde possible. Le travail des journalistes, il intervient quand même plutôt après. C'est-à-dire nous, derrière, on peut expliquer ce qu'il y a dans les modèles, confronter les résultats de ces modèles au débat public, aux décisions de politiques publiques des décideurs, etc. Mais ce n'est pas au même endroit, je pense.
\end{displayquote}
Distinction entre 
\begin{displayquote}
    C'est marrant parce que j'ai fait un truc devant les anciens des mines, il y a 15 jours ou 3 semaines. Et il y avait un retraité d'EDF qui était là,  qui vient me voir. Et il me dit, mais je ne comprends pas. Moi, j'ai tout à fait la solution face au changement climatique. La tonne de carbone évité est beaucoup moins chère dans les pays du Sud. Donc, en fait, il faut tout mettre, tout l'argent pour que c'est plus simple et c'est beaucoup moins cher. C'est dix fois moins cher. J'ai tout calculé moi-même.Donc là, il me montre son cahier, il y a tous ces calculs et tout ça. Donc, c'est à la fois un peu mignon, mais en même temps, pour moi, ça témoignait un peu de ce que je disais tout à l'heure, c'est à dire l'approche très, très techno, très ingé du problème. Comme si on pouvait dire, ah, ben en fait, on se fout, on fait baisser les émissions que dans les pays du Sud. Et nous, on ne le fait pas puisque ça coûte moins cher de le faire là bas. Enfin, c'est politiquement et socialement un peu impossible de tenir ce genre de discours.
\end{displayquote}


\subsection{La force de la modélisation participative}

Face à ce constat, une méthodologie de modélisation radicalement différente apparait. Plutot que des modèles importants, complexes, développé pendant des années par des équipes scientifiques de manière indépendante de l'usage finale des modèles ou de leur interprétation, la méthode de la modélisation participante prend en compte les acteurs, leurs visions et leurs valeurs dès la conception du modèle. \\

Le paradigme est tout à fait différent : plutot que de chercher à produire un modèle objectif et toujours plus réaliste, les acteurs de cette technique prennent à bras-le-corps les oppositions normative des acteurs. 

Répondre au problème par des modèles plus spécifiques
=> SGPE + biodiversité
=> enqueté modélisation participante

Une maniere de s'ouvrir à la communauté
=> Open source + tous les produits annexes chez WILIAM

\subsection{Les décideurs politiques ont besoin d'objectifs chiffrés}

A la fin, le modele doit sortir un chiffre, n'importe lequel
=> 


\section{Discussions}








