\chapter{Interprétation des fonctions de dommage dans le monde réel}
\label{chapter:socio}



Cette partie du mémoire s'intéresse à la manière dont les résultats issus de la modélisation intégrée sont interprétés dans le débat public et permettent de prendre des décisions. Plus précisement, on s'interroge sur les utilisations des modèles par divers acteurs de la politique climatique : techniciens, décideurs, journalistes, scientifiques, activistes.  Pour cela, on réalise une série d'entretiens semis-directifs. Ils ont pour but de répondre aux questions suivantes : comment est interprétée l'incertitutde inhérente aux modèles pour la prise de décision ? 



\begin{methodbox}

On réalise des entretiens semi-directifs avec des acteurs du débat public sur les questions climatiques. Cette catégorie est volontairement large. On classe ensuite les enquêtés en quatre catégories d'acteurs : les scientifiques, les techniciens, les politiques et la société civile. Les scientifiques désignent les acteurs dont la parole est reconnue comme porteuse d'un message scientifique. Il s'agit généralement de modélisateurs, et parfois d'acteurs gravitant autour des milieux universitaires : comité d'éthique, communication scientifique non-vulgarisée. Les techniciens sont les acteurs d'administrations publiques n'ayant pas de mandats électifs. Il s'agit le plus souvent de spécialistes de sujets spécifiques, qui transcrivent les connaissances climatiques en plan d'action ou en texte réglementaires. Les politiques sont tous les acteurs qui sont dotés d'un mandat électif. Leur spécificité est d'être amené à prendre des décisions, à trancher dans des contextes où les conséquences des différentes actions sont soit empreinte d'incertitude soit de choix moraux. Enfin, la société civile désigne les acteurs non-institutionnels qui s'emparent de sujets climatiques. Il s'agit en particulier de journalistes, mais aussi d'activistes ou de personnes engagées dans le monde associatif. \\

Les entretiens sont du format semi-directif. Une grille d'entretien est soumise aux enquêtés. Cependant, les réponses sont longues et libres, et peuvent donner lieu à des questions inédites. Symétriquement, toutes les questions ne sont pas traitées dans tous les entretiens. Ces entretiens sont ensuite retranscrits. Les réponses aux questions sont taguées selon les idées dominantes qu'elles abordent, ce qui permet ensuite de mieux les rapprocher dans la partie analyse. 


\end{methodbox}


Beaucoup de remarques issues de \cite{intergovernmental_panel_on_climate_change_ipcc_mitigation_2023}



\section{Introduction}




\section{Résultats}

\subsection{Des grandes parties}

\subsection{Des grandes parties}

\paragraph{Idée importante 1}

\paragraph{Idée importante 2}



\section{Discussions}








