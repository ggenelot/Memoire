


\newglossaryentry{tipping point}
{
    name=tipping point, 
    plural = tipping points, 
    description={Un \textit{tipping point} est un point où la dynamique du système change \cite{acemoglu_colonial_2001} }
}

\newglossaryentry{risk tipping point}
{
    name=risk tipping point,
    plural = risk tipping points, 
    description={kljekljkl}
}

\newglossaryentry{regime climatique}
{
    name=régime climatique, 
    description={Terme repris de \cite{aykut_gouverner_nodate}, qui le décrivent comme un \enquote{système complexe d’arènes et d’institutions qui a réuni des acteurs et des partenaires de plus en plus nombreux, a suscité de nouvelles pratiques de recherche, a instauré des procédures d’évaluation et de validation, a vu s’affronter des intérêts économiques et des enjeux politiques variés et a établi, enfin, des relations particulières entre sciences, expertise, politiques et marchés}.}
}

\newglossaryentry{cout social du carbone}
{
    name=coût social du carbone, 
    description={Un modèle est une représentation simplifiée de la réalité, qui permet de la rendre intelligible. C'est le produit de la modélisation, \textit{l'élément du dispositif scientifique qui fait médiation entre le système réel et la théorie} \cite{briens_decroissance_2015}. Il y en a deux types : des modèles appliqués et quantitatifs, et des modèles théoriques plus stylisés \cite{briens_decroissance_2015}. Le rôle central des modélisateur.ices dans la fabrication du modèle est souvent mis en avant, comme par exemple chez \cite{beck_epistemic_2016} : \textit{model building is an art and not a mechanical procedure}. \\}
}


\newglossaryentry{modele}
{
    name=modèle, 
    plural = modèles, 
    description={Un modèle est une représentation simplifiée de la réalité, qui permet de la rendre intelligible. C'est le produit de la modélisation, \textit{l'élément du dispositif scientifique qui fait médiation entre le système réel et la théorie} \cite{briens_decroissance_2015}. Il y en a deux types : des modèles appliqués et quantitatifs, et des modèles théoriques plus stylisés \cite{briens_decroissance_2015}. Le rôle central des modélisateur.ices dans la fabrication du modèle est souvent mis en avant, comme par exemple chez \cite{beck_epistemic_2016} : \textit{model building is an art and not a mechanical procedure}. \\}
}

\newglossaryentry{modelisation}
{
    name=modélisation, 
    plural = modélisation, 
    description={La modélisation est une pratique visant à produire un \gls{modele}.}
}

\newglossaryentry{iam}
{
    parent=modele,
    name=modèle intégré, 
    plural = modèles intégrés, 
    description={Les modèles intégrés sont des \textit{représentations simplifiées de systèmes sociaux et physiques complexes, qui se concentrent sur les interactions entre l'économie, la société et l'environnement} \cite{intergovernmental_panel_on_climate_change_ipcc_annex_2023}  \\
    On peut distinguer deux grandes familles de modèles intégrés : les modèles de \textit{policy optimization} et ceux de \textit{policy evaluation}. Les premiers visent à trouver le "meilleur" chemin parmis toutes les options possibles, c'est à dire celui maximisant une fonction objectif. Les seconds permetent de voir l'évolution de variables clés au fil du temps, mais ne vise pas à maximiser des variables. \\
    Malgrè de nombreuses limites, ils constituent un outil essentiel de la prise de décision climatique.
    }
}

\newglossaryentry{damage function}
{
    name=fonction de dommage, 
    plural = fonctions de dommage, 
    description={Composant du modèle permettant de représenter les dommages du changement climatique. Plus précisement, Waidelich et al \cite{waidelich_climate_2024} les définissent comme \emph{projections of economic damage from climate change are key for evaluating climate mitigation benefits, identifying effects on vulnerable communities and informing discussions around adaptation needs, as well as loss and damage financing.}}
}

\newglossaryentry{scenario}
{
    name=scénario,
    plural = scénarii, 
    description={Un scénario est une jeu de paramètre que l'on donne à un modèle}
}

\newglossaryentry{imp}
{
    name=illustrative mitigation pathways, 
    description={Trajectoires d'atténuation stéréotypée}
}

\newglossaryentry{attenuation}
{
    name=atténuation, 
    description={Réduction de l'ampleur du changement climatique par la réduction des émissions de gaz à effet de serre}
}


\newglossaryentry{ethique}
{
    name=éthique, 
    description={}
}

\newglossaryentry{procedural ethics}
{
    name=éthique procédurale, 
    description={Respect des lignes conductrices et des usages dans un travail de recherche. Correspond à ce que l'on appelle communément de la \textit{bonne recherche} (good science). \cite{tuana_leading_2010} la définit ainsi : \enquote{ethical aspects of the process of conducting scientific research, such as: falsification, fabrication, and plagiarism; care for subjects (human and non-human animal); responsible authorship issues; analysis of and care for data}.}
}

\newglossaryentry{intrinsic ethics}
{
    name= éthique intrinsèque, 
    description={Valeurs personnelles qui sont incorporées dans le travail de recherche, de manière consciente ou non. \cite{tuana_leading_2010} la définit ainsi : \enquote{ethical issues and values that are embedded in or otherwise internal to the production of scientific research and analysis. These involve ethical issues arising from, for example: the choice of certain equations, constants, and variables; analysis of data; handling of error, and degree of confidence in projections.}}
}

\newglossaryentry{extrinsic ethics}
{
    name=éthique extrinsèque, 
    description={Dimension éthique des effets que produit la recherche sur la société. \cite{tuana_leading_2010} la définit ainsi : \enquote{ethical issues that are external to the production of scientific research. These arise, for example, when considering the impact of scientific research on society; e.g., the effects of technological innovations on social ends such as health and well-being, whether pressing social and economic issues are likely to be addressed and if so, who benefits, and the role of science in policy-making.}}
}

\newglossaryentry{incertitude}
{
    name=incertitude, 
    plural = incertitudes, 
    description={Selon le Larousse, désigne les \emph{points, éléments qui, dans quelque chose, ne peuvent être connus à l'avance, sont imprévisibles}. Peut être lié soit à un déficit d'information, à un \emph{manque de données}, soit à de l'indetermination. Ainsi, Walker et al., cité par \cite{beck_epistemic_2016}, la décrit comme \enquote{any departure from the unachievable ideal of complete determinism}. Toujours selon eux, l'incertitute dans les modèles intégrés existe sous deux formes : l'incertitude scientifique, qui désigne les limites (actuelles à la connaissance), sur les causes, processus et conséquences du changement climatique; et l'incertitude éthique, qui désigne l'absence de consensus sur le cadre éthique adéquat.}
}

\newacronym{SCC}{SCC}{social cost of carbon}

\newacronym{FUND}{FUND}{Climate Framework for Uncertainty, Negotiation and Distribution model}

\newacronym{DICE}{DICE}{Dynamic Integrated Climate-Economy model}

\newacronym{PAGE}{PAGE}{Policy Analysis of Greenhouse Effect model}

\newacronym{CCNUCC}{CCNUCC}{Convention Cadre des Nations Unies sur le Changement Climatique}

\newacronym{SBSTA}{SBSTA}{Subsidiary Body on Science and Technical Advice}

\newacronym{COP}{COP}{Conference Of Parties}

\newacronym{ipcc}{IPCC}{Voir GIEC}

\newacronym{AR6}{AR6}{Sixth Assesment Report}

\newacronym{WILIAM}{WILIAM}{WIthin Limits Integrated Assesment Model}

\newacronym{GIEC}{GIEC}{Groupe Intergouvernemental d'Experts sur le Climat. Il s'agit dun groupe d'experts, qui travaillent pour faire la synthèse des connaissances disponibles en matière de climat. \cite{cointe_ar6_2024}}
