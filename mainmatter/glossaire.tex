\newglossaryentry{latex}
{
    name=latex,
    description={Is a markup language specially suited 
    for scientific documents \LaTeX}
}


\newglossaryentry{tippingpoint}
{
    name=tipping point, 
    description={Un \textit{tipping point} est un point où la dynamique du système change \cite{acemoglu_colonial_2001} }
}

\newglossaryentry{Risk tipping point}
{
    name=risk tipping points, 
    description={kljekljkl}
}

\newglossaryentry{Modèle}
{
    name=Modèle, 
    description={Un modèle est une représentation simplifiée de la réalité}
}

\newglossaryentry{iam}
{
    name=Modèle intégré, 
    description={Les modèles intégrés sont des \textit{représentations simplifiées de systèmes sociaux et physiques complexes, qui se concentrent sur les interactions entre l'économie, la société et l'environnement} \cite{intergovernmental_panel_on_climate_change_ipcc_annex_2023}  \\
    On peut distinguer deux grandes familles de modèles intégrés : les modèles de \textit{policy optimization} et ceux de \textit{policy evaluation}. Les premiers visent à trouver le "meilleur" chemin parmis toutes les options possibles, c'est à dire celui maximisant une fonction objectif. Les seconds permetent de voir l'évolution de variables clés au fil du temps, mais ne vise pas à maximiser des variables. \\
    Malgrè de nombreuses limites, ils constituent un outil essentiel de la prise de décision climatique.
    }
}

\newglossaryentry{scenario}
{
    name=Scénario, 
    description={Un scénario est une jeu de paramètre que l'on donne à un modèle}
}

\newglossaryentry{imp}
{
    name=Illustrative mitigation pathways, 
    description={Trajectoires d'atténuation stéréotypée}
}

\newglossaryentry{attenuation}
{
    name=Atténuation, 
    description={Réduction de l'ampleur du changement climatique par la réduction des émissions de gaz à effet de serre}
}


\newglossaryentry{ethique}
{
    name=éthique, 
    description={}
}

\newglossaryentry{procedural ethics}
{
    name=éthique procédurale, 
    description={Respect des lignes conductrices et des usages dans un travail de recherche. Correspond à ce que l'on appelle communément de la \textit{bonne recherche} (good science). \cite{tuana_leading_2010} la définit ainsi : \textit{ethical aspects of the process of conducting scientific research, such as: falsification, fabrication, and plagiarism; care for subjects (human and non-human animal); responsible authorship issues; analysis of and care for data}.}
}

\newglossaryentry{intrinsic ethics}
{
    name= éthique intrinsèque, 
    description={Valeurs personnelles qui sont incorporées dans le travail de recherche, de manière consciente ou non. \cite{tuana_leading_2010} la définit ainsi : \textit{ethical issues and values that are embedded in or otherwise internal to the production of scientific research and analysis. These involve ethical issues arising from, for example: the choice of certain equations, constants, and variables; analysis of data; handling of error, and degree of confidence in projections.}}
}

\newglossaryentry{extrinsic ethics}
{
    name=éthique extrinsèque, 
    description={Dimension éthique des effets que produit la recherche sur la société. \cite{tuana_leading_2010} la définit ainsi : \textit{ethical issues that are external to the production of scientific research. These arise, for example, when considering the impact of scientific research on society; e.g., the effects of technological innovations on social ends such as health and well-being, whether pressing social and economic issues are likely to be addressed and if so, who benefits, and the role of science in policy-making.}}
}




\newacronym{ipcc}{IPCC}{Voir GIEC}

\newacronym{giec}{GIEC}{Groupe Intergouvernemental d'Experts sur le Climat. Il s'agit dun groupe d'experts, qui travaillent pour faire la synthèse des connaissances disponibles en matière de climat. \cite{cointe_ar6_2024}}


