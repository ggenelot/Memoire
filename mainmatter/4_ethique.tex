\chapter{Éthique de la modélisation intégrée}
\label{chapter:ethique}

\chapterabstract{Les choix de modélisation des fonctions de dommage sont lourds de conséquences sur le message que véhiculent les résultats de ces modèles. Ces résultats alimentent le débat public sur les questions climatiques. Vu les enjeux sociaux et sociétaux qui sont à l'oeuvre, la pratique de la modélisation implique des questions éthiques importantes. Quelles sont-elles, et comment sont-elles prises en compte ? Cette partie est plus conceptuelles et épistémiques, et cherche à identifer des points d'attention de la modélisation.}



\begin{table}
    \centering
    \begin{tabular}{cccc}
         Pratique&  Enjeu&  Concept associé& Commentaire\\
         Choisir les modèles pour faire tourner les scénarios&  Framing / possibility space&  & \\
         Attribuer une valeur à un paramètre (taux d'actualisation)&  Equité intergénérationnelle&  & \\
         Fabrique du doute&  Responsabilité dans l'interprétation des résultats&  Doute normativement inappropriés& \\
    \end{tabular}
    \caption{Enjeux éthiques dans les modèles et cadre d'analyse associés}
    \label{tab:ethique}
\end{table}




\section{Les trois niveaux de l'éthique et la responsabilité du modélisateur}

Cette section pose plusieurs questions quant à la place de la technique et de la modélisation dans la cité. 

Elle s'appuie notamment sur \cite{jonas_principe_2008} et \cite{vast machine}, ainsi que sur la classification des enjeux éthiques de \cite{tuana_leading_2010}.

\subsection{L'éthique procédurale}

L'éthique procédurale est le niveau le plus facilement accessible de l'éthique du chercheur. C'est ce niveau qui fait que 

\begin{figure}
    \centering
    %\includegraphics{}
    \caption{Les trois niveaux de l'éthique dans la recherche. \textit{Reproduire la table de Tuana en l'adaptant selon mes besoins}}
    \label{fig:diag-venn}
\end{figure}

\subsection{L'éthique intrinsèque}

\subsection{L'éthique extrinsèque}


\begin{itemize}
    \item [[cadrage]], forcément normatif
« IAM analysis could focus on only a subset of relevant futures and thus push society in certain directions without sufficient scrutiny » (\href{zotero://select/library/items/2SDDNUUF}{“Annex III: Scenarios and Modelling Methods”, 2023, p. 1862}) (\href{zotero://open-pdf/library/items/CHVFSLLH?page=22&annotation=4MBM5B9Q}{pdf})

\end{itemize}

 Ces considérations sur l'éthique extrinsèque prendront plus de sens dans la section suivante, on l'on développera l'idée que la modélisation participe activement au cadrage du débat public, et donc aux choix de futurs possibles. 

\section{Construire / dessiner les futurs possibles}

Cette section montre que le savoir produit par la science est situé dans le temps et dans l'espace. Ainsi, il n'est plus positif, mais bien normatif, en ceci qu'il décrit un univers des possibles. 

Un exemple de à quel point le framing peut impacter la connaissance est la classification des pays. Dans le SPM de l'AR5, les parties n'ont pas pu s'accorder sur un type de classification des pays à adopter ; finalement toutes les figures et textes associés ont été rejetées par les gouvernements. On pourrait ici penser que présenter une information ou une autre n'a pas d'importance, tant que celle-ci a été produite dans les normes scientifiques en vigueur. Pourtant, cet exemple montre que les gouvernements considèrent que le choix d'une classification porte en soi un message trop important ; reconnaissant alors l'absence de neutralité du contenu scientifique \cite{edenhofer_mapmakers_2014}. 

Les modèles intégrés sont utilisés comme base scientifique pour les négociations climatiques. Ils permettent notamment de décrire l'espace des possibles, c'est-à-dire l'ensemble des chemins qui peuvent être pris par les sociétés. Cette relation entre le modèle et la prise de décision est décrite par une image très parlante dans \cite{edenhofer_mapmakers_2014}, où les modélisateurs sont décrits comme des cartographes, et les décideurs comme des navigateurs. Dès lors, le rôle des modélisateurs-cartographe est de décrire l'espace possible, l'ensemble des zones qui sont navigables ; et, si possible, les conditions de navigation que l'on peut rencontrer dans ces zones.  En regard de ces nouvelles connaissances, les décideurs-navigateurs doivent décider du cap à suivre aujourd'hui selon la zone de navigation voulue pour demain. Cette distinction très nette entre décideurs et modélisateurs n'est pas sans limitations. L'une d'entre elle est le cadrage, c'est à dire la manière dont le débat public est façonné par le cadre qu'on lui donne.  Plusieurs choses influencent ce cadrage. Nous verrons d'abord que le prisme technique et énergétique aggrégé de la plupart des modèles représente ce genre de contraintes, sans questionner les modèles sous-jacents. Nous verrons ensuite que le choix des variables, et notamment la monétarisation des dommages fait que de nombreuses dynamiques ne sont soit pas prises en compte, soit prises en compte d'une manière que l'on peut questionner. Enfin, nous, aborderons l'idée que la connaissance est socialement construite, en se basant notamment sur les écrits de Helène Ongino, pour questionner le caractère universel des résultats des modèles. 

\subsection{Un monde homogène, technique et neutre ?}

=> il n'y a pas de modèle qui décompose les dommages par groupe sociaux


« there are concerns that IAMs are describing transformative change on the level of energy and land use, but are largely silent about the underlying socio-cultural transitions that could imply restructuring of society and institutions » (\href{zotero://select/library/items/2SDDNUUF}{“Annex III: Scenarios and Modelling Methods”, 2023, p. 1862}) (\href{zotero://open-pdf/library/items/CHVFSLLH?page=22&annotation=JY4VBIZY}{pdf})

 \subsection{Prendre en compte le non-monétaire}


« The difficulty in fully representing the extent of climate damages in monetary terms may be the most important and challenging limitation of IAMs and it is mostly directed to costbenefit IAMs. However, all categories of IAMs present important limitations (Annex III.I.9). » (\href{zotero://select/library/items/2SDDNUUF}{“Annex III: Scenarios and Modelling Methods”, 2023, p. 1844}) (\href{zotero://open-pdf/library/items/CHVFSLLH?page=4&annotation=YT933ZM4}{pdf})
 

\begin{itemize}
    \item « there are concerns that IAMs are missing important dynamics » (\href{zotero://select/library/items/2SDDNUUF}{“Annex III: Scenarios and Modelling Methods”, 2023, p. 1861}) (\href{zotero://open-pdf/library/items/CHVFSLLH?page=21&annotation=WDMBNU3A}{pdf})
=> et donc ne sont pas vraiment précis, ou passent à coté de certaines choses

\end{itemize}
 



 \subsection{La modélisation, une connaissance construite}


\section{Les doutes normativement inappropriés, ou la limite entre mauvaise science et fabrique de l'inaction}

Cette section repose sur les approches des \textit{ignorance studies}, notamment \cite{melo-martin_fight_2018} et \cite{gross_routledge_2015}, avec comme ressources complémentaires \cite{noauthor_carnet_2024} et \cite{proctor_agnotology_2008}. Elle est tirée de réflexions tirées du cours de Mathias Girel à l'ENS \cite{girel_vertus_2023}. 



+ Stern / Nordhaus argument (c'est sûr qu'ils se sont balancé des trucs à la gueule) 

+ steve keen \cite{keen_appallingly_2021}

\subsection{Rester dans sa retenue ou s'engager ?}

\subsection{L'impossible neutralité des modèles}

\subsection{La responsabilité}




